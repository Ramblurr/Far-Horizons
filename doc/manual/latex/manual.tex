\documentclass[10pt,titlepage]{article}
\usepackage[utf8x]{inputenc}
% \usepackage{fullpage}
\usepackage{amsmath}
\usepackage{enumerate}
% \usepackage{notes}
\usepackage{tabularx}
\usepackage{longtable}
\usepackage{fullpage}
\usepackage[tt]{titlepic}
\usepackage{graphicx}
\usepackage{abstract}
\usepackage{calc}
\usepackage{multirow}
\usepackage[usenames,dvipsnames]{color}
\usepackage{colortbl}
\usepackage[bookmarks, pdftitle={Far Horizons Rules (7th ed)}, pdfauthor={Rick Morneau}]{hyperref}
\hypersetup{
    colorlinks,%
    citecolor=black,%
    filecolor=black,%
    linkcolor=black,%
    urlcolor=black
}
%opening
\title{
\textbf{{\Huge Far Horizons}}
\vspace{10pt}
%\hline
\hrulefill
\vspace{10pt}
{\small\MakeUppercase{Seventh Edition Rules}}
}
\author{Rick Morneau}
\date{January 2, 1999}
\titlepic{\includegraphics[width=\textwidth]{./pleaides_square_border.pdf}}

\renewcommand{\abstractname}{Introduction}
\renewcommand{\abstractnamefont}{\normalfont\bfseries}
\renewcommand{\abstracttextfont}{\normalfont}


\definecolor{lightblue}{rgb}{0.8,0.85,1}
\definecolor{lightyellow}{RGB}{255,255,153}
\definecolor{lightgreen}{RGB}{153,255,153}
\definecolor{lightred}{RGB}{255,153,153}
\makeatletter\newenvironment{importantnote}{%
  \begin{lrbox}{\@tempboxa}\begin{minipage}{\linewidth-15\fboxsep}}{\end{minipage}\end{lrbox}%
   \fcolorbox{black}{lightyellow}{\usebox{\@tempboxa}}
}\makeatother

\makeatletter\newenvironment{informationnote}{%
   \begin{lrbox}{\@tempboxa}\begin{minipage}{\linewidth-15\fboxsep}}{\end{minipage}\end{lrbox}%
   \fcolorbox{black}{lightgreen}{\usebox{\@tempboxa}}
}\makeatother

\makeatletter\newenvironment{warningnote}{%
   \begin{lrbox}{\@tempboxa}\begin{minipage}{\linewidth-15\fboxsep}}{\end{minipage}\end{lrbox}%
   \fcolorbox{black}{lightred}{\usebox{\@tempboxa}}
}\makeatother
\begin{document}
\maketitle
\begin{abstract}
FAR HORIZONS is a strategic role-playing game of galactic exploration, trade,
diplomacy, and conquest.  The first and second editions were designed for play
by postal mail.  Later editions were designed for play by electronic mail, and
many mistakes were made in the transition.  Hopefully, this current edition has
corrected most of those mistakes.

At the start of a game, each player controls an intelligent species and the
home planet on which it lives.  As the game progresses, you can explore nearby
regions of the galaxy and establish colonies.  As you range farther and farther
from home, you will encounter other intelligent species.  These encounters can
be hostile, neutral, or friendly, depending on the participants.  Interstellar
war is a distinct possibility.

FAR HORIZONS, unlike some similar games, has been designed to make role-
playing as easy and practical as possible.  In addition to being a rich and
realistic simulation, there are no true victory conditions---the game is played
solely for enjoyment.  However, at the end of the last turn, final statistics
for all species will be sent to all of the players so that they can compare
their relative strengths and weaknesses.  Thus, rather than requiring a massive
bloodletting as in some other similar games, it's possible for a peace-loving
species to effectively ``win''.

Still, those who enjoy a more aggressive game, or those who wish to role-play
an ``evil'' or warlike species will not be disappointed.  FAR HORIZONS does not
discriminate against anyone---it simply tries to be as realistic as possible.
\end{abstract}
\newpage
\tableofcontents
\newpage

\section{GAME BASICS}

The following sections will discuss the basic concepts used throughout the
game.


\subsection{THE GAME TURN}

A game turn is about five Earth years long.  As a result, generations can pass
in a single game.  Populations will increase and colonies will grow.  Since
quite a lot can happen in five years, FAR HORIZONS is strategic in nature,
rather than tactical.  The reason for a five year turn will be explained below.


\subsection{THE GALAXY}

In FAR HORIZONS, the galaxy is a small open star cluster.  It is approximately
spherical, but is projected onto an easy-to-use two-dimensional map.  The size
of the galaxy and the actual number of stars in it will depend on the number
of players.  As an example, for a game with about 15 players, the galaxy would
have a radius of approximately 18 parsecs and contain about 80 usable star
systems.

The basic unit of interstellar distance is the PARSEC, which is equal to 3.26
light-years.  Thus, every star system in the galaxy will have an X, Y, and Z
coordinate in parsecs, relative to the reference point at 0,0,0.  Furthermore,
all coordinate values are zero or greater.  Thus, you can picture the galaxy
as floating in a box whose lower, left, rear corner has the coordinates 0,0,0.
(Negative numbers were used in earlier games, but players sometimes forgot the
minus signs, with disastrous results.  Use of only positive numbers will help
prevent mistakes.)


\subsection{SPACE TRAVEL}

In this universe, scientists have discovered only one way to break the barrier
imposed by the speed of light.  The method is inherently risky.  Essentially,
a spaceship's engines must ``rip'' a hole through the space-time fabric in order
to travel from one point to another.  A ship does this by creating a temporary,
private ``wormhole'' that the ship can pass through.  A ship does this by
creating and manipulating a small black hole.  When a ship travels in this
way, it is said to ``jump'' to its destination.

There is no limit to how far a ship can travel in this way, but greater
distances involve greater risks.  The technology involved is called
``avitics'', and the more experience a species has in this technology, the
farther a ship can jump without risk of missing its destination or of being
destroyed.

The time needed for a ship to travel through the wormhole is independent of the
distance being traveled.  However, it IS dependent on the mass of the ship, as
shown by the following equation:

\[
\textrm{time in years} = 5\: tanh (\textrm{mass in grams})
\]


If you're not mathematically inclined, don't worry!  What the equation says is
that it takes almost exactly five years for anything larger than a pea, since
all masses used in the game will be at the extreme asymptotic limit of the
equation.  Only extremely small masses (such as the photons that make up radio
waves) can take significantly less than 5 years to travel through a wormhole.

Note also that no time goes by for the people on the ship.  For them, the move
is essentially instantaneous.  For the people who remain behind, however, the
ship will appear to ``wink out'' immediately.  Five years later, it will reappear
at its destination.

Creation of a wormhole is risky business, and there is always a chance that a
ship can be swallowed up by the wormhole it creates.  If this happens, then the
ship and everything it carries will be totally destroyed.  It is also possible
for a ship to ``mis-jump''.  If this occurs, the ship will not be destroyed, but
will simply arrive at the wrong destination.  As a species gains knowlit (or edge and
experience in gravitics technology, its ships will become more reliable and
less susceptible to sudden destruction or mis-jumps.


\subsection{COMMUNICATIONS}

Communications across interstellar distances utilizes the same technology as
space travel, but on a much smaller scale.  Any two tranceivers can be ``tuned''
to each other (if both sides cooperate), effectively creating a small wormhole
through which radio waves can pass.  Also, since radio waves have very close
to zero mass, the transmission time is close to zero.  Thus, although ships
require five years to move between star systems, radio communication is
essentially instantaneous, and home planets can always be in instant
communication with their ships and colonies.  Note, though, that ships in
transit are incapable of communicating, since, for them, no time is actually
passing.


\subsection{THE HOME PLANET}

Each player starts the game with a home planet.  This is where his species
evolved, acquired intelligence, and eventually learned how to travel among the
stars.

At the start of the game, the only material resources available to a species
are those of its home planet.  These resources can be used to build units
such as mines, factories, spaceships, planetary defenses, etc.  As the game
proceeds, a species can colonize other planets and tap them for resources
as well.


\subsection{TECH LEVELS}

A tech level is a measure of how advanced a species is in a specific field of
technology.  Six technologies are defined and used in this game.  Each one is
described below:

\begin{description}
 

\item[MINING]   Mining tech level is a measure of how proficient
			a species is at tapping a planet's natural resources.
			It includes functions such as mining and farming,
			and basic refining and food processing. As mining tech
			level increases, greater quantities of raw materials
			can be produced.

\item[MANUFACTURING]	Manufacturing tech level is a measure of proficiency
			at converting raw materials to usable, final forms.
			It is used to determine how many units, such as ships,
			can be built each turn. It also places a limit on
			the maximum size of ships that a species can build.

\item[MILITARY]	Military tech level is a measure of experience in
			warfare. It indicates the level of sophistication
			in military strategy, tactics, and weaponry.  It
			is one of the major factors used to determine the
			outcome of armed conflict.

\item[GRAVITICS]	Gravitics tech level is a measure of a species'
			knowledge of gravity control.  Gravitics allows
			the design of the engines which drive interstellar
			ships, since a black hole cannot be created and
			controlled without the ability to manipulate
			gravitational fields.

\item[LIFE SUPPORT]	Life support tech level is a measure of a species'
			experience in surviving in hostile environments.
			It is used to construct and maintain artificial
			shelters on planets with temperatures or atmospheres
			that differ from the home planet.  It also determines
			the effectiveness of defensive shields used on ships.

\item[BIOLOGY]		Biology tech level is a measure of a species'
			knowledge and experience in the life sciences.  Its
			most obvious applications are in genetic engineering,
			germ warfare, and terraforming (i.e., using specially
			designed micro-organisms to modify the atmosphere
			and micro-flora of a planet, making it more suitable
			for habitation).
 \end{description}
There are many other applications of the six basic technologies in addition to
the ones mentioned above.  These will be discussed later.  After the start of
the game, tech levels will increase primarily through research.  While there is
no limit to how high a tech level can get, in practice it is unlikely that a
tech level will ever exceed 100.


\subsection{TURN PROCESSING: Sequence of Events}

Each turn is processed in six steps, and the order form that you send to the
gamemaster has six corresponding sections.  These sections are:

\begin{enumerate}
	\item Combat orders
	\item Pre-departure orders
	\item Jump orders
	\item Production orders
	\item Post-arrival orders	
	\item Strike orders
\end{enumerate}
When your turn is processed, all combat orders are processed first, then all
pre-departure orders, then all jump orders, and so on.  After yur orders have
been processed, a special program is run that handles population growth and
interspecies transactions, and performs several other housekeeping chores.
Finally, a "report" program is run that generates summaries that will be sent
to the players.  Thus, several programs are actually used by the gamemaster
to process a turn.

[The strike phase is a limited-combat phase.  Any combat that takes place
in the strike phase generally takes the form of an initial surprise attack.
Combat that requires more time, such as bombardment and siege, will take place
in the combat phase of the following turn, and are thus continuations of the
combat that began in the strike phase.]

At the end of each report is an order form that you will need to fill out.
This form will contain all six sections, even though not all of them may be
applicable for the current turn.  For example, the jump section cannot be used
in the first turn, since you have no ships.  Simply delete the sections that
do not apply, and fill out and send in those that do.

Each section of the orders begins with a START command and ends with an END
command.  Each section should only appear ONCE.  Thus, each order form will
contain the following sections:
\begin{verbatim}
START COMBAT
;Combat orders belong here.

END

START PRE-DEPARTURE
;Pre-departure orders belong here.

END

START JUMPS
;Jump orders belong here.

END

START PRODUCTION
;Production orders belong here.

END

START POST-ARRIVAL
;Post-arrival orders belong here.

END

START STRIKES
;Strike orders belong here.

END
\end{verbatim} 

The production section will be started for you, and will have an appropriate
PRODUCTION order for each planet you control.  This will save you a little
time, and will help prevent you from accidentally forgetting to give orders
for a planet.

The six sections shown above may appear in any sequence.  However, it is
recommended that you fill out your orders using the sequence provided,
since that is the sequence in which they will be executed.  In other words,
combat orders will be executed before pre-departure orders, pre-departure
orders will be executed before jump orders, and so forth, REGARDLESS of where
each section appears in your order form.

The orders that you give in each section must be reasonable for that section.
For example, you may not give JUMP orders in any section except the jump
section.  BUILD and RESEARCH orders may only be given in the production
section.  Combat orders may only be given in the combat and strike sections.
And so on.  Here is a complete list:

\begin{description}
 \item[Combat section:]  
  \setlength{\itemsep}{1pt}
  \setlength{\parskip}{0pt}
  \setlength{\parsep}{0pt}
  \item Attack
  \item Battle
  \item Engage
  \item Haven
  \item Hide
  \item Hijack
  \item Summary
  \item Target
  \item Withdraw
\end{description}
\begin{description}
    \item[Pre-departure section:]
  \setlength{\itemsep}{1pt}
  \setlength{\parskip}{0pt}
  \setlength{\parsep}{0pt}
	\item Ally
	\item Base
	\item Deep
	\item Destroy
	\item Disband
	\item Enemy
	\item Install
	\item Land
	\item Message
	\item Name
	\item Neutral
	\item Orbit
	\item Repair
	\item Scan
	\item Send
	\item Transfer
	\item Unload
	\item Zzz
\end{description}
\begin{description}
    \item[Jump section:]
  \setlength{\itemsep}{1pt}
  \setlength{\parskip}{0pt}
  \setlength{\parsep}{0pt}
	\item Jump
	\item Move
	\item Pjump
	\item Visited
	\item Wormhole
\end{description}
\begin{description}
    \item[Production section:]
  \setlength{\itemsep}{1pt}
  \setlength{\parskip}{0pt}
  \setlength{\parsep}{0pt}
	\item Ally
	\item Ambush
	\item Build
	\item Continue
	\item Develop
	\item Enemy
	\item Estimate
	\item Hide
	\item Ibuild
	\item Icontinue
	\item Intercept
	\item Neutral
	\item Production
	\item Recycle
	\item Research
	\item Shipyard
	\item Upgrade
\end{description}
\begin{description}
 \item[Post-arrival section:]
  \setlength{\itemsep}{1pt}
  \setlength{\parskip}{0pt}
  \setlength{\parsep}{0pt}
	\item Ally
	\item Auto
	\item Deep
	\item Destroy
	\item Enemy
	\item Land
	\item Message
	\item Name
	\item Neutral
	\item Orbit
	\item Repair
	\item Scan
	\item Send
	\item Teach
	\item Telescope
	\item Terraform
	\item Transfer
	\item Zzz
\end{description}
\begin{description}
 \item [Strike section:]
  \setlength{\itemsep}{1pt}
  \setlength{\parskip}{0pt}
  \setlength{\parsep}{0pt}
\item same as combat section
\end{description}


All of these commands will be explained in later sections of this document.


\begin{importantnote}
A special note must be made about the TRANSFER command.  There is a possible
situation in which colonists and supplies could be transferred to a new colony
immediately after a jump.  If the planet is already inhabited by another
species, neither species will know about the new colony until the next turn.
To prevent this very unrealistic kind of incident, a TRANSFER to a planet may
only be made in the post-arrival phase IF the planet is already inhabited by
the species making the transfer.  Otherwise, the transfer will have to be
done in the pre-departure phase of the next turn.  Once the colony has been
established, you may TRANSFER goods to the planet in either the pre-departure
or post-arrival phases.
\end{importantnote}


\subsection{VICTORY CONDITIONS}

There are no final winners or losers in Far Horizons, just as there aren't any
in other role-playing games.  The only purpose of the game is to have fun.
However, for those who want to know how well they did relative to the other
players, the following will be done:

\begin{quotation}
	At the end of a game, a final summary report will be sent to all
	players, and will contain a list of the total revenue-generating
	capacity of each species along with their final tech levels and
	other statistics.  These values can be used to get a good idea
	of who "won" the game.\end{quotation} 

A game will last between 20 and 100 turns.  The actual final turn number will
be randomly determined by the gamemaster and will be kept secret until the end
of the game is announced.  This approach will prevent the unrealistic gameplay
that always results when players know that the game is about to end.  The
gamemaster may arbitrarily and secretly extend the game if he feels it would be
inappropriate to interrupt an ``interesting'' situation, or if he is convinced
that everyone is having a lot of fun.



\section{STARS AND PLANETS}

As mentioned earlier, the galaxy of FAR HORIZONS is a small open star cluster,
similar to the Pleiades Cluster.

In a real cluster, many of the stars would be components of binary or trinary
star systems.  In fact, such multiple star systems make up about 85\% of all
star systems in the Milky Way galaxy.  In systems such as these, planets, if
any, are likely to have very odd orbits, and if they have atmospheres, their
climates are likely to be extremely erratic.  As a result, multiple star
systems have been totally eliminated from the game.  You can assume that they
exist, but they will not be shown on star maps or be made available for use
by players.


\subsection{STARS}

The location of a star is indicated by its X, Y, and Z coordinates, which are
always positive integers greater than or equal to zero.  I assume here that the
reader has sufficient technical background to understand how to work with these
coordinates (called Cartesian Coordinates).  Just as a reminder, though, the
distance between any two stars can be calculated using the formula:

\begin{equation*}
      \textrm{distance}  =  \sqrt{ (X2 - X1)^2   +  (Y2 - Y1)^2   +  (Z2 - Z1)^2 }
\end{equation*}

Most distances can be estimated by simply looking at the map and counting
squares.  Finicky players, however, may want to calculate exact distances using
the above formula.

Any region of space defined by a set of specific X Y Z coordinates is called a
``sector''.  Thus, the number of sectors in the galaxy is simply the number of
possible combinations of X, Y, and Z.  Most sectors are effectively empty.
Only a relatively small number of sectors contain usable stars and planets.


\subsubsection{SPECTRAL CLASS}

The information in this section is for ``color'' only, and does not play an
important role in the game.  Feel free to skim through it.  Do not let the
technical jargon bother you.

In addition to its galactic coordinates, a star is identified by its spectral
class, which indicates both its color and its size.  In general, stars which
radiate more towards the red end of the spectrum are smaller than those which
radiate more towards the blue end of the spectrum.  There are, however, many
exceptions.  It is not uncommon to find red giants or blue dwarves.  Also, in
general, large stars will have more usable planets than small stars. \\

\noindent Here is a list of the most common spectral colors:
\begin{description}
	\item[O] - Blue stars, hottest and largest (eg. Lambda Orionis)
	\item[B] - Blue-white (eg. Rigel, Spica)
	\item[A] - White (eg. Sirius, Vega)
	\item[F] - Yellow-white (eg. Canopus, Procyon)
	\item[G] - Yellow (eg. Earth's sun, Capella))
	\item[K] - Orange (eg. Arcturus, Pollux)
	\item[M] - Red stars, coolest and smallest (eg. Antares, Betelgeuse) \\
 \end{description}

\noindent Here is a list of the most common spectral types:
\begin{description}
   \item[(not marked)]	- main sequence star
	\item[d]	- ordinary dwarf star
	\item[g]	- ordinary giant star
	\item[D]	- degenerate dwarf star \\
\end{description}


Each class contains ten subdivisions numbered 0 through 9.  Thus an F5 star is
approximately halfway between F0 and G0.  Zero indicates the hottest within the
spectral class, while 9 indicates the coolest with the class.

\noindent Here are some examples:
\begin{description}
	\item[O8] 	Blue
	\item[dF1]	Yellow-white dwarf
	\item[DA5]	Degenerate white dwarf
	\item[G6]	Yellow
	\item[gG9]	Yellow giant
	\item[dM5]	red dwarf
	\item[DB]	Degenerate blue-white dwarf
	\item[gK7]	Orange giant
\end{description}
\begin{informationnote}
It is customary to drop the number in the designation of degenerate dwarf
stars.  Thus, in the astronomical literature, one is more likely to see ``DA''
rather than ``DA5''.  I have left them in, however, for consistency.
\end{informationnote}

The star map which you will receive from the gamemaster will be two-
dimensional, and will show X and Y coordinates on the axes.  If a star exists
at a particular X,Y coordinate, then a number and a spectral type will be
displayed at that location.  The number will be the Z coordinate.  Thus,
if you see the following:

\begin{verbatim}
				 12
				gF6
\end{verbatim}

at the position on the map where X=5 and Y=9, it indicates that a giant yellow-
white star is located at coordinates X=5, Y=9, Z=12.


\subsection{PLANETS}

Planets are real estate, and are the ultimate source of all wealth and power in
the game.  As a result, planets are also the most common cause of interstellar
conflict.

In the following sections, the various terms used to describe a planet's
physical characteristics will be discussed.  It is by evaluating these
characteristics that a player can decide if a planet is suitable for
colonization and/or exploitation by his species.  For reference, here is
a sample of a star system scan:

\begin{verbatim}
Coordinates:	x = 7	y = 10	z = 18	stellar type =  A0   8 planets.

                Temp  Press Mining
   #  Dia  Grav Class Class  Diff  LSN  Atmosphere
  ---------------------------------------------------------------------
   1    5  0.28  23     3    0.37   42  Cl2(100%)
   2   14  0.80  23     6    0.38   39  F2(33%),H2O(67%)
   3   12  0.91  18     9    0.65   24  HCl(38%),Cl2(32%),F2(30%)
   4   21  2.00  15     6    0.38   24  CO2(29%),HCl(43%),Cl2(17%),F2(11%)
   5   14  0.96  10     9    2.25    0  N2(47%),CO2(23%),O2(30%)
   6  189  2.67   4    18    4.34   33  CH4(49%),NH3(48%),N2(3%)
   7  103  1.94   3    17    0.49   33  H2(58%),CH4(42%)
   8   34  2.46   3    11    0.77   21  He(40%),N2(60%)
\end{verbatim}

\subsubsection{PLANET NUMBER}

Each planet has a number, indicating its relative position around the sun.
Planet number 1 is closest to the sun.  The planet with the largest number is
farthest from the sun.  Actual distances are not important for game purposes.
In the above sample, there are eight planets numbered 1 through 8.


\subsubsection{PLANET DIAMETER}

A planet's diameter is listed under ``Dia'' and is the diameter of the planet in
thousands of kilometers.  Thus, in the above sample, planet \#5 has a diameter
of 14,000 kilometers.  (For comparison, Earth has a diameter of approximately
13,000 kilometers, and Jupiter has a diameter of about 143,000 kilometers.)


\subsubsection{PLANET GRAVITY}

A planet's gravity is listed in the ``Grav'' column, and is given in standard
Earth gravities.  Thus, Earth would have a value of 1.00.  In the above sample,
a person standing on the surface of planet \#4 would weigh twice as much as on
Earth.


\subsubsection{TEMPERATURE CLASS}

A planet's temperature class is listed in the ``Temp Class'' column and can have
one of the values listed in Table~\ref{tab:temp}

\begin{table}[h]
\begin{center}
\begin{tabular}{|ccl|ccl|}
\hline
\rowcolor{lightblue} \textbf{Temp Class} & \textbf{Deg. (C)} & \textbf{Examples} & \textbf{Temp. Class} & \textbf{Deg. (C)} & \textbf{Examples} \\
\hline
    1 &    -273 &  Pluto, absolute zero   &   16 &     180 & \\
    2 &    -240 &  Mercury (dark side)    &    17 &     210 & \\
    3 &    -210 &  Neptune                &   18 &     240 & \\
    4 &    -180 &  Titan (moon of Saturn) & 19 &    270 & \\
    5 &    -150 &  Uranus, Saturn         &   20 &     300 & \\
    6 &    -120 &  Jupiter                &   21 &     330 & \\
    7 &     -90 &                         &   22 &     360 & \\
    8 &     -60 &                         &   23 &     390 & \\
    9 &     -30 &  Mars                   &   24 &     420 & \\
   10 &       0 &                         &  25 &     450 &  Venus \\
   11 &      30 &  Earth                  &   26 &     480 & \\
   12 &      60 &                         &   27 &     510 &  Mercury \\
   13 &      90 &                         &   28 &     540 & \\
   14 &     120 &                         &   29 &     570 & \\
   15 &     150 &                         &   30 &     600 & \\
\hline
\end{tabular}
\caption{Planet Temperature Class}
\label{tab:temp}
\end{center}
\end{table}

The temperatures listed in the table are approximate, average temperatures that
can be experienced on the surface of the planet.  Colonies are more likely to
prosper if the temperature class of a planet is as close as possible to that of
the home planet.  If this is not the case, life support technology will have to
be applied to produce an artificial environment for the colony.


\subsubsection{PRESSURE CLASS}

A planet's pressure class is listed in the ``Press Class'' column and can have
one of the values listed in Table~\ref{tab:press}.

\begin{table}[h]
\begin{center}
\begin{tabular}{|ccl|ccl|}
\hline
\rowcolor{lightblue} \textbf{Press. Class} & \textbf{Pressure} & \textbf{Examples} & \textbf{Press. Class} & \textbf{Pressure}& \textbf{Examples} \\
\hline
        0  &     0.0000 &  Mercury,vacuum &  15  &    32  & \\
        1  &     0.0020 &                 & 16   &   64  & \\
        2  &     0.0039 &                 & 17   &   128 & \\
        3  &     0.0078 & Mars            & 18   &   256 & \\
        4  &     0.0156 &                 & 19   &   512 & \\
        5  &     0.0312 &                 & 20   &   1024  &   Saturn \\
        6  &     0.0625 &                 & 21   &   2048  & \\
        7  &     0.125  &                 & 22   &   4096  & \\
        8  &     0.25   &                 & 23   &   8192  & \\
        9  &     0.5    &                 & 24   &   16384  & \\
        10 &     1      & Earth           & 25   &   32768  & \\
        11 &     2      & Uranus          & 26   &   65536  &    Jupiter \\
        12 &     4      &                 & 27   &   131072 & \\
        13 &     8      & Neptune         & 28   &   262144 & \\
        14 &     16     & Venus           & 29   &   524288 & \\
\hline
\end{tabular}
\caption{Planet Pressure Class}
\label{tab:press}
\end{center}
\end{table}

The pressures listed in the above table are multiples of Earth ``atmospheres''
and are approximate, average values that can be experienced on the surface of
the planet.  Colonies are more likely to prosper if the pressure class of a
planet is as close as possible to that of the home planet.  If this is not the
case, life support technology will have to be applied to produce an artificial
environment for the colony.


\subsubsection{MINING DIFFICULTY}

A planet's mining difficulty is listed under ``Mining Diff''.  Mining difficulty
is a relative figure-of-merit which indicates how difficult it is to extract
or utilize a planet's natural resources.  Higher values represent greater
difficulties.  This value will be used to determine how much raw materials
can be produced on a planet during each game turn.


\subsubsection{LIFE SUPPORT NEEDED (LSN)}

The number in the column labeled ``LSN'' is the amount of Life Support technology
that your species needs to survive on the planet.  If your Life Support tech
level is lower than this value, then your species may not safely colonize the
planet, and any attempt to colonize the planet will result in the destruction
of the colony.  If your Life Support tech level is equal to or greater than
this value, then your species may safely colonize the planet.  We will discuss
later how these values are determined.

Keep in mind that these values apply only to the species that does the scan.
If you receive a scan from another player, the LSN values will probably not
apply to your species.


\subsubsection{PLANETARY ATMOSPHERE}

A planetary atmosphere will be described in terms of the gases that are its
major components.  Each gas in the atmosphere will have a percentage value
associated with it.  The gases and their symbols are used in this
game are listed in Table~\ref{tab:gas}.

\begin{table}[h]
\begin{center}
\begin{tabular}{|ll|}
\hline
\rowcolor{lightblue} \textbf{Symbol} & \textbf{Name} \\
\hline
		H2	&Hydrogen \\
		CH4	&Methane \\
		He	&Helium \\
		NH3	&Ammonia \\
		N2	&Nitrogen \\
		CO2	&Carbon Dioxide \\
		O2	&Oxygen \\
		HCl	&Hydrogen Chloride \\
		Cl2	&Chlorine \\
		F2	&Fluorine \\
		H2O	&Water Vapor or Steam \\
		SO2	&Sulfur Dioxide \\
		H2S	&Hydrogen Sulfide \\
\hline
\end{tabular}
\caption{Plantary Atmosphere Gases}
\label{tab:gas}
\end{center}
\end{table}

For example, an Earth-type atmosphere would be described as N2(78\%), O2(22\%).
This means that Earth's atmosphere consists of approximately 78\% Nitrogen and
22\% Oxygen.


\subsubsection{FURTHER NOTES ON TEMPERATURE, PRESSURE, AND ATMOSPHERE}

The temperature, pressure, and gaseous components of a planet are the prime
criteria by which you can decide if the planet is suitable for colonization by
your species.  Furthermore, there are many possible combinations, and finding
a planet that closely matches your home planet will not be easy.

For planets with a pressure class greater than about 20, the gases in the
atmosphere will usually condense into liquids, and will often even solidify as
you get closer to the surface.  On these planets, there is often no clear
distinction between the atmosphere and the solid surface.  These planets are
usually gas giants.  Only the most intrepid and advanced species would ever try
to colonize the surface of such a planet since it is so inherently hostile to
life.  Because of this, any colonies that you do establish "on" such planets
will actually be on moons orbiting the planet, artificial satellites, etc.

If you wish to colonize or exploit a planet that is unsuitable for your
species, then some form of life support must be provided.  This is where your
Life Support tech level will play an important part.  A low value for this
technology will give you few options - you will have to search longer and
farther from home to find a planet that is suitable for colonizing.  As your
Life Support tech level increases, you will have a wider range of options.

Another possibility open to a species is to actually modify or ``terraform'' the
planet.  This can be done by seeding the atmosphere with specially designed
micro-organisms, and by operating large plants on the surface that will convert
the atmosphere to something more suitable.  Both temperature and pressure
classes can also be changed in this way.  However, terraforming is available
only to species with relatively high Biology tech levels.



\section{SETTING UP FOR THE GAME}

When a player is ready to enter the game, he must fill out a Set-up Form and
send it to the gamemaster.  The form is in Appendix D.  The following sections
explain how to fill it out.


\subsection{TECH LEVEL POINT ALLOCATION}

A starting player has a total of 15 points that can be allocated to Military,
Gravitics, Life Support, and Biology tech levels.  Any combination is allowable
as long as they add up to 15.

A tech level can even be zero if you decide that your species has no knowledge
in that area.  If Gravitics tech level is zero, then you may only build sub-
light ships.  If your Life Support tech level is zero, then none of your ships
will have defensive shields.  If a tech level is zero, then it can only be
raised if another species transfers the knowledge to you.  We'll discuss how
to do this later.

All species start the game with Mining and Manufacturing equal to 10.


\subsection{SPECIES, HOME PLANET AND GOVERNMENT}

Choose a name for your species.  It can be something out of science fiction or
something you make up.  Feel free to use your imagination!  Examples: Human,
Kenda Jo, Klingon, Graxian, Jubjub Denboy, Ferengi, Mo Ja'adebi, etc.

Choose a name for your home planet.  Examples: Earth, Mars, Barsoom, Dune,
Giver of Life, Korunkorunkoruniman, Toi di Bai, etc.

Choose a name for your government or political system.  Examples: The United
States of America, The Korun Federation, The Holy Alliance of Denadan, The
Jubjub Denboy Empire, etc.

All of the above names are limited to 31 characters and will be truncated if
they are longer.  When referring to them later, case will not be significant.
Names may contain spaces and any printable characters except commas and
semi-colons.  ALL characters in a name, including spaces, are included
in the 31-character limit.  Names may NOT contain tabs!

Name the type of government or political system of your species.  (In this
game, we assume that all planets owned by a species are run by a single
government.)  Be descriptive but limit yourself to 31 characters.  Examples:
Libertarian Democracy, Communist Totalitarianism, Constitutional Monarchy,
Absolute Dictatorship, Benevolent Plutocracy, Slaver Republic, Ruthless
Oligarchy, Theocratic Monarchy, Military Republic, etc.

The political system you choose could have an impact on the game, since species
may react to each other differently, depending on ideology.  Also, since Far
Horizons is a role-playing game, the player should always operate within the
limitations imposed by the type of government he chooses.


\subsection{PROCESSING TURNS}
\label{sec:processingturns}

After the gamemaster has received your set-up information, he will either send
you a map of the galaxy or he'll tell you where you can obtain a copy via ftp.
He will also send you a status report which contains a detailed description of
your home planet and its production capabilities.  At the end of the status
report, there will be an order form that you can fill out for your first turn.

In the following sections, we will discuss these items in more detail.


\subsection{STAR SYSTEM DATA}
\label{sec:starsysdata}


The star system data sent to you for the first turn provides a detailed
description of the home star system.  Refer to Chapter 2 if you have any
problem deciphering the information.

Star system data can be provided to you for every star system that your species
visits.  However, you will not receive this information automatically, but must
specifically request a ``scan''.  This can be done in the pre-departure or post-
arrival section of the orders you send to the gamemaster.  (We'll have more to
say about the `scan' command later.)


\subsection{SPECIES STATUS REPORT}
\label{sec:speciesstatusreport}


This section of the information sent to you describes your species' current
situation.  At the start of a game, you will not have any planetary defenses,
ships, etc.  You start the game with a blank slate.  You DO, however, have
mining and manufacturing capability which you can start using immediately
to build ships and other items.

Names of items that are available for use on a planet will be printed out in
full, along with their class abbreviation, required carrying capacity, and
quantity.  For example:

\begin{equation*}
 \textrm{Raw Material Units}(\textrm{RM},\textrm{C1}) = 17
\end{equation*}
The above example indicates that the planet has 17 unused raw material units,
which require a carrying capacity of 1 each, and have the abbreviation ``RM''.
(We'll have more to say about raw material units and carrying capacity later.)

Ages of ships and their orbital status will be indicated as in the following
example:
\begin{verbatim}
	CT Derby Dan (A5,O6)
\end{verbatim}
Here, ``CT'' is the abbreviation for a corvette.  ``A5'' indicates that the age
of the ship is 5 turns.  The letter ``O6'' indicates that the ship is in orbit
around planet number 6 (as opposed to being landed on the surface).  If a ship
is on the surface (i.e., ``landed''), then ``L'' will be used.  If a ship is in
deep space, not associated with any planet, the letter ``D'' will be used.  If a
ship voluntarily withdrew from combat during the strike phase, then ``WD'' will
be used.  If a ship was forced to jump using Forced Jump Units during the
strike phase, then ``FJ'' will be used.  If a ship is still under construction,
then the complete designation will be simply ``(C)''.  (We'll have more to say
about landing, orbiting, and combat later.)

The rest of the status section is intended to be self-explanatory.  It
indicates what your current Mining and Manufacturing Bases are and how much
you can produce in the current turn.  (Abbreviations MI = Mining Tech Level,
MA = Manufacturing Tech Level, and MD = Mining Difficulty.)  How to use this
data will be described later.  The atmospheric requirements and the list of
gases poisonous and harmless to your species will be used later in the game
to decide whether or not other planets are suitable for colonization.  How
to colonize planets will be described later.


\subsection{NOMENCLATURE}
\label{sec:nomenclature}


In order to make turn-processing as easy as possible for the gamemaster, and to
allow as much processing as possible to be done by the computer, certain naming
conventions have been established.  Players should be careful to follow these
conventions carefully.

All items in the game have a 2-5 letter abbreviation.  For example, heavy
cruisers use the designation ``CA'', while starbases use ``BAS''.  These class
designations should ALWAYS be used when giving orders for the items.

Ship and planet names are limited to 31 characters and will be truncated if
they are longer.  Case is not significant.  Thus, the following names for a
heavy cruiser are all the same:
\begin{verbatim}
	CA USS Enterprise
	ca uss enterprise
	CA USS ENTERPRISE
\end{verbatim}

The particular upper/lower case combination that you use the first time you
name a ship or planet will be used in all subsequent reports.  You may use
ANY combination, however, in subsequent orders.

Names may contain spaces and any printable characters except commas and
semi-colons.  ALL characters in a name, including spaces, are included in
the 31 character limit.



\section{MINING AND MANUFACTURING}
\label{sec:miningandmanufacturing}

In order to explore the galaxy, you will need spaceships, among other things.
The following sections will discuss how mining and manufacturing are used to
produce the items you will need.


\subsection{RAW MATERIAL UNITS}
\label{sec:rawmaterialunits}


As was mentioned earlier, Mining tech level does not apply strictly to mining,
but also includes such operations as drilling for oil, refining metals,
growing, harvesting, and processing crops, etc.  In other words, it is a
generic term that covers all aspects of tapping a planet's natural resources.
This technology is used to produce all of the raw materials that, in turn, are
used for final production.  Thus, mining technology produces raw materials and
manufacturing technology consumes them.

In FAR HORIZONS, quantities of raw materials are measured in RAW MATERIAL
UNITS.  The number of raw material units that can be produced on a planet in a
single turn is:
\begin{equation*}
	\textrm{Raw Material Units}  = \dfrac{\textrm{Mining Tech Level}  \times  \textrm{Mining Base}}{\textrm{Mining Difficulty}}
\end{equation*}
MINING BASE is a relative measure of the total physical plant, acreage,
infrastructure, etc. that is available on the planet for the production of
raw materials.  Thus, it is a measure of how many mines, farms, drilling
facilities, steel mills, etc that can be used by the planet's population.

For example, if your mining tech level is 4, your mining base is 136 and the
planet's mining difficulty is 1.24, then you can produce
\begin{equation*}
\dfrac{4  \times  136}{1.24} = 438.71 = 438 \textrm{ raw material units}
\end{equation*}
in the current turn.  Note that fractions are always dropped.  Since current
values are always shown on status reports that are sent to players, it will
NOT be necessary for players to do these calculations.

The mining base on all home planets will automatically increase by about 2\% per
turn.  On colonies, however, the mining base can only be increased by shipping
in colonists and installing colonial mining units (we'll have more to say about
this later).  This, in fact, is how colonies are actually started.

When referring to raw material units in orders which you send to the
gamemaster, use the abbreviation ``RM''.  We'll have more to say about this
later.

Finally, unused raw material units may be carried over into later turns.  In
effect, such carry-over is the equivalent of long term storage and stockpiling.


\subsection{PRODUCTION CAPACITY}
\label{sec:productioncapacity}


A species' PRODUCTION CAPACITY is a measure of its ability to convert raw
material units into final products.  Specifically, it is a measure of the
number of raw material units that may be converted into usable products in
a single turn.  This value is determined as follows:

\begin{equation*}
  \textrm{Production Capacity}  =  \textrm{Manufacturing Tech Level}  \times  \textrm{Manufacturing Base}
\end{equation*}

MANUFACTURING BASE is a relative measure of the total physical plant available
on a planet for the conversion of raw material units into final products.
Thus, it is an indication of how many factories, dock yards, processing plants,
etc. that can be used by the planet's population.

For example, if manufacturing tech level is 6 and manufacturing base is 142,
then

\begin{equation*}
	\textrm{Production Capacity}  =  6  \times  142  =  852
\end{equation*}

in the current turn.  Thus, the planet has the production capacity to ``consume''
852 raw material units and ``purchase'' 852 units of final products.  Since
current values are always shown on status reports that are sent to players,
it will NOT be necessary for players to do these calculations.

The manufacturing base on all home planets will automatically increase by
about 2\% per turn.  On colonies, however, the manufacturing base can only be
increased by shipping in colonists and installing colonial manufacturing units
(we'll have more to say about this later).

Production capacity can only be utilized at its maximum if sufficient raw
material units are available.  You cannot convert what you don't have.
Furthermore, production capacity cannot be carried over into later turns.  If
you don't utilize your full production capacity, then it simply means that
your production facilities are not operating at full capacity.

It is possible to use raw materials in the same turn as they are produced.
They do not have to be stockpiled in earlier turns.  Thus, the total number of
raw material units available for manufacturing in the current turn is the sum
of what was carried over from the previous turns plus what will be produced in
the current turn.



\section{PRODUCTION}
\label{sec:production}

At the end of each status report that you receive, there will be an order form
which you must fill out and send to the gamemaster.  In it you will provide
your orders for the current turn.  The form has sections for combat orders,
pre-departure orders, jump orders, production orders, post-arrival orders,
and strike orders.

At the start of a game, you will only be able to provide production orders
for one planet - your home planet.  In later turns, you will need to fill
out orders for each planet on which you have production capacity.

Production for each planet must be preceded by the order:
\begin{verbatim}
           PRODUCTION PL name
\end{verbatim}
where \texttt{PL} is the class abbreviation for a planet and \texttt{name} is the name of the
planet.  This command indicates that the orders that follow apply only to the
indicated planet.  Orders for each planet must be preceded by a PRODUCTION
command.

Each section of the orders is started for you, and you must provide your
specific orders.  For example, if your home planet is called "Earth", and you
have a colony called \texttt{Mars}, the initial production section will look something
like this:
\begin{verbatim}
START PRODUCTION
    PRODUCTION PL Earth
    ; Enter your production orders for planet Earth here.

    PRODUCTION PL Mars
    ; Enter your production orders for planet Mars here.

END PRODUCTION
\end{verbatim}
Note that some lines begin with a semi-colon.  The semi-colon indicates that
the line is actually a comment and that the computer should ignore it.  You may
add comments of your own.  Comments may appear anywhere and always start with a
semi-colon.  Everything on a line that follows a semi-colon is ignored by the
computer.  Completely blank lines are also ignored.

Each order begins with a command word, such as \texttt{TRANSFER}, \texttt{RESEARCH}, etc.
These command words are not case sensitive and may be truncated to just the
first THREE letters.  Any letters after the first three are ignored.  Thus, all
of the following are equivalent to \texttt{RESEARCH}: \texttt{reS}, \texttt{RESEA}, \texttt{REsoQQQ}, etc.

When a name (such as the name of a ship or planet) is not the last item on a
line, it should be terminated by a comma.  This is necessary because names can
contain spaces.  Thus, in the following example, the name \texttt{Laughing Dog} is
immediately followed by a comma:
\begin{verbatim}
      JUMP	CT Laughing Dog, PL Shangri La
\end{verbatim}
Note though that \texttt{Shangri La} is NOT followed by a comma, since it is the last
item on the line.

\begin{warningnote}
	WARNING! Omitting a required comma is one of the most common
	mistakes made by players.  It is also one of the most frustrating,
	since the computer will reject the order.
\end{warningnote}

\noindent You may use as many tabs and/or spaces as you wish to separate items in a
command to make it more readable.  Tabs and spaces at the very beginning of a
command line are ignored.  For example, the following are valid orders:
\begin{verbatim}
      Orbit	FF	Thomas Edison,		PL	Mars
      JUMP	PB	Benjamin Franklin,	12	7	18
\end{verbatim}
Tabs, like commas, will terminate a name.  However, use of commas is
recommended because tabs are not always easily visible.  Any spaces that
appear in a name will become part of the name.

If a comment appears after an order on the same line, spaces and tabs that
precede the semi-colon are ignored.  Thus, there is no need to terminate a name
with a comma if the name is immediately followed by a comment.  For example:
\begin{verbatim}
	Jump	TR7 Love Dove,	PL Mars    ;Deliver new colonists.
\end{verbatim}
Note that \texttt{Mars} is not followed by a comma, and that the spaces between \texttt{Mars}
and the semi-colon will not be considered as part of the name \texttt{Mars}.

Any items that you may wish to build, such as ships, planetary defenses, etc.
will have a ``cost'' of equal amounts of raw material units and production
capacity.  Thus, if you wish to build a spaceship with a cost of 200, then
a total of 200 raw material units will be used in its construction, and a
total production capacity of 200 will be needed to actually build it.

Except for ships, all items must be built in a single turn.  You can take as
many turns as you like to build ships.  Thus, if we continue the above example,
you could build the ship in, say, 4 turns.  You could allocate 100 raw material
units and production capacity in the first turn, 70 in the second turn, 0 in
the third turn, and 30 in the fourth turn.

Finally, keep in mind that raw material units and production capacity must
always be spent in equal amounts.  Thus, for example, if your production
capacity is greater than the number of raw material units, then the excess may
not be used.  Also keep in mind that unused raw material units MAY be carried
over into later turns, but that unused production capacity may NOT.


\subsection{SPACESHIPS}
\label{sec:spaceships}


Spaceships come in all sizes as can be seen in Table~\ref{tab:ships}.

\begin{table}[h]
\begin{center}
\begin{tabular}{|crlrrrl|}
\hline
  \rowcolor{lightblue} \textbf{Minimum}&    &  &              &   \textbf{Carrying} &    \multicolumn{2}{>{\columncolor{lightblue}}c|}{ \textbf{Cost}} \\
\rowcolor{lightblue} \textbf{MA} &    \textbf{Abbr}   & \textbf{Class} &              \textbf{Tonnage} &   \textbf{Capacity} &  \textbf{FTL} &  \textbf{Sub-light} \\
\hline
      2 &    PB  &  Picketboat        &  10,000  &    1   &     100 &      75 \\
      4 &    CT  &  Corvette          &  20,000  &    2   &     200 &     150 \\
     10 &    ES  &  Escort            &  50,000  &    5   &     500 &     375 \\
     20 &    FF  &  Frigate           & 100,000  &   10   &    1000 &     750 \\
     30 &    DD  &  Destroyer         & 150,000  &   15   &    1500 &    1125 \\
     40 &    CL  &  Light Cruiser     & 200,000  &   20   &    2000 &    1500 \\
     50 &    CS  &  Strike Cruiser    & 250,000  &   25   &    2500 &    1875 \\
     60 &    CA  &  Heavy Cruiser     & 300,000  &   30   &    3000 &    2250 \\
     70 &    CC  &  Command Cruiser   & 350,000  &   35   &    3500 &    2625 \\
     80 &    BC  &  Battlecruiser     & 400,000  &   40   &    4000 &    3000 \\
     90 &    BS  &  Battleship        & 450,000  &   45   &    4500 &    3375 \\
    100 &    DN  &  Dreadnought       & 500,000  &   50   &    5000 &    3750 \\
    110 &    SD  &  Super Dreadnought & 550,000  &   55   &    5500 &    4125 \\
    120 &    BM  &  Battlemoon        & 600,000  &   60   &    6000 &    4500 \\
    130 &    BW  &  Battleworld       & 650,000  &   65   &    6500 &    4875 \\
    140 &    BR  &  Battlestar        & 700,000  &   70   &    7000 &    5250 \\
\hline
\end{tabular}
\caption{Spaceship Classes in Far Horizons}
\label{tab:ships}
\end{center}
\end{table}

\begin{verbatim}

\end{verbatim}
\texttt{Minimum MA} is the minimum Manufacturing tech level that a species must have
before it is capable of building a ship of the corresponding tonnage.  For
example, a species must have a Manufacturing tech level of 40 or higher in
order to be able to build a light cruiser.

\texttt{Abbr} is the abbreviation that will be used for the corresponding class of
ship.  The correct abbreviation must ALWAYS be used when making reference to
a specific ship.  Thus, if you build a battleship named \texttt{USS Iowa}, you must
always refer to it as BS USS Iowa.  You are free to give your ships whatever
names you wish, as long as you conform to some simple naming conventions that
will be discussed later.

\texttt{Tonnage} is the ship's fully loaded deadweight in long tons in standard
gravity.  (Note that the maximum tonnage that a species can handle is equal
to 5000 times the Manufacturing tech level.)

\texttt{Carrying Capacity} is the amount of cargo units or colonist units units that
a ship can carry.  More will be said about this later.  (Note that carrying
capacity is equal to the tonnage divided by 10,000.)

\texttt{FTL Cost} is the number of raw material units and production capacity needed
to build a ship capable of interstellar travel (FTL stands for ``Faster-than-
light'').  (Note that the FTL cost is equal to the tonnage divided by 100.)

\texttt{Sub-light Cost} is the number of raw material units and production capacity
needed to build a ship that is NOT capable of interstellar travel.  These ships
have all of the capabilities of the equivalent FTL ship, except that they must
either remain in the star system in which they were constructed, or move very
slowly from one sector to the next.  (Note that the sub-light ship cost is
equal to the FTL cost minus 25\%.)

To build ships, use the BUILD command, and be careful to use the correct class
abbreviation.  For example:
\begin{verbatim}
START PRODUCTION
    PRODUCTION PL Earth
    ; Enter your production orders for planet Earth here.

        ;Build sub-light frigate and pay for all of it now. Cost is 750.
        BUILD   FFS Farragut

        ;Build light cruiser. Pay one-quarter now (500) and the rest (1500)
        ; later...
        BUILD   CL      Guardian,       500

END PRODUCTION
\end{verbatim}

Note that if a payment amount is not specified, then the ship will be
completely built.  The computer will calculate the needed cost.  You will, of
course, have to keep track of the costs yourself to make sure that you don't
try to spend more than what you have.

To continue construction on a ship, use the CONTINUE command, as in the
following examples:
\begin{verbatim}
START PRODUCTION
    PRODUCTION PL Earth
    ; Enter your production orders for planet Earth here.

        ;Pay 1250 more on the dreadnought we started a while back...
        CONTINUE        DN Dynamite Dan,        1250

        ;Finish the light cruiser we started last turn...
        CONTIN  CL      Guardian

END PRODUCTION

\end{verbatim}
If you do not specify the amount to spend, then the computer will calculate
the cost needed to finish construction.

NEVER use the same name for two different ships, even if their classes are
different.  For example, if corvette \texttt{CT Danny Boy} already exists, then an
order to build frigate \texttt{FF Danny Boy} will fail.


\subsection{COMBAT EFFECTIVENESS}
\label{sec:combateffectiveness}


In this section, we will digress slighlty and consider just how effective the
above-listed ships would be in combat situations.

In a game where each turn is five years long, any rules regarding combat must,
of necessity, be highly abstract.  Furthermore, FAR HORIZONS is a strategic
game, as opposed to a tactical game.  And since the game must be played by
mail, it is not possible for players to be directly involved in the details of
space combat.  As a result, the outcome of all battles must be determined by
the gamemaster's computer.

Whenever combat does occur, the computer will assign probabilities defining
each side's offensive and defensive potentials.  Offensive potentials will
depend most heavily on the numbers and sizes of the ships (i.e. how much
firepower is available) and their Military tech levels (i.e. how effective
their weaponry and tactics are).  Defensive potentials will depend most heavily
on the sizes of the ships (i.e. how much armor and defensive shield generators
can be carried) and their Life Support tech level (since the design of shields
is an application of life support technology).

MOST IMPORTANTLY, a single, large ship has MUCH more offensive and defensive
capability than several smaller ships of the same total tonnage.  For example,
a single frigate (100,000 tons) could EASILY destroy a fleet of five corvettes
(20,000 tons each) or ten picketboats (10,000 tons each) if tech levels are
about the same.

Each battle will consist of one or more ``rounds'', during which ships on either
side may be damaged or destroyed.  The battle will proceed until one side is
destroyed or forced to leave.

Thus, the players will not be directly involved in determining the outcome
of a battle.  Only the results of a battle will be reported to them, on their
status reports.


\subsection{PLANETARY DEFENSES}
\label{sec:planetarydefenses}


Planetary defenses are intended to protect a planet from attack by enemy ships.
They can attack and be attacked by ships in space near the planet.  A planet
may not be controlled by an invading force until all planetary defenses have
been destroyed.  Planetary defenses can also be used in besieging a planet if
they are on the planet that is under siege.

In this game, we will not be concerned with the number, location or strength of
the individual bases and facilities that make up a planet's planetary defenses.
If a planet is attacked, only the total planetary defense strength is
important.

Specifically, each planetary defense unit will have a cost of 1, and will have
the combat ``value'' of a 50 ton FTL warship.  For example, if a planet has
produced 2000 planetary defense units (at a total cost of 2000), then it will
have the same combat effectiveness as a 100,000 ton frigate.

At first glance, it might seem that planetary defenses are not very effective
for the amount spent in their construction.  Keep in mind, though, that
planetary defenses can grow without limit, becoming more and more powerful at
each step in their growth.  Also, planetary defenses are not limited in size
by the species' Manufacturing tech level, as ships are.  And, as we will see
later, planetary defenses don't suffer from aging effects.

When referring to planetary defenses in orders, use the abbreviation "PD".
Each planetary defense unit requires a cargo capacity of 3.

To construct planetary defenses, use the BUILD command, as in the following
examples:
\begin{verbatim}
START PRODUCTION
    PRODUCTION PL Earth
    ; Enter your production orders for planet Earth here.

        Build   102 PD  ; Build 102 planetary defense units. Total cost = 102.

    PRODUCTION PL Vega VI
    ; Enter your production orders for planet Vega VI here.

        BUI     55 pd   ; Build 55 planetary defense units. Total cost = 55.

END PRODUCTION
\end{verbatim}
The units produced will remain on the producing planet unless they are
transferred elsewhere.


\subsection{TRANSPORTS}
\label{sec:transports}

Transports are ships that are specially designed to carry colonists and cargo.
They CAN take part in combat, but their offensive and defensive capabilities
are about one-tenth that of warships.  (As indicated in the warship list above,
warships also have carrying capacity, but it is much less than a transport of
the same tonnage.)

Transports can be built in any multiple of 10,000 tons.  The maximum tonnage,
as for other ships, is 5000 times the Manufacturing tech level.

The carrying capacity of a transport is the total number of colonist units or
cargo units (or combination thereof) that a transport can carry at any one
time.  Colonist units will be discussed later.  A cargo unit is the equivalent
of one raw material unit.  For example, a transport with a carrying capacity
of 200 could carry 90 colonist units and 110 raw material units.  Transports
should use the class abbreviation "TRn", where "n" is the tonnage divided by
10,000.

The carrying capacity of a transport is calculated as follows:
\begin{equation*}
	\textrm{Transport Carrying Capacity}  =  (10 + n/2) \times n
\end{equation*}

For example, a TR7 (i.e. a 70,000 ton transport) has a carrying capacity of
\[
(10 + 7/2) \times 7 = (10 + 3) \times 7 = 91 
\]
Note that fractions are dropped in the division.

\begin{table}[h]
\begin{center}
\begin{tabular}{|crrrrl|}
\hline
  \rowcolor{lightblue} \textbf{Minimum}&    &           &   \textbf{Carrying} &    \multicolumn{2}{>{\columncolor{lightblue}}c|}{ \textbf{Cost}} \\
\rowcolor{lightblue} \textbf{MA} &    \textbf{Abbr}   &              \textbf{Tonnage} &   \textbf{Capacity} &  \textbf{FTL} &  \textbf{Sub-light} \\
\hline
        2    &   TR1   &  10,000     &     10   &   100  &   75 \\
        4    &   TR2   &  20,000     &     22   &   200  &   150 \\
        10   &   TR5   &  50,000     &     60   &   500  &   375 \\
        12   &   TR6   &  60,000     &     78   &   600  &   450 \\
        20   &   TR10  &  100,000    &     150  &   1000 &   750 \\
        30   &   TR15  &  150,000    &     255  &   1500 &   1125 \\
        40   &   TR20  &  200,000    &     400  &   2000 &   1500 \\
        100  &   TR50  &  500,000    &     1750 &   5000 &   3750 \\
        120  &   TR60  &  600,000    &     2400 &   6000 &   4500 \\
  \ldots  &\ldots&\ldots&\ldots&\ldots&\ldots\\
\hline
\end{tabular}
\caption{Transport example listing (this is not exhaustive)}
\label{tab:trans}
\end{center}
\end{table}

Table~\ref{tab:trans} lists several example Transport specifications. You may build a transport of any tonnage as long as it is a multiple of 10,000,
and as long as your Manufacturing tech level is high enough.  Note that you do
NOT have to calculate the carrying capacity of ships each time you give orders
for the ships, since all capacities are listed on your status reports.

To build transports, use the BUILD command, and be careful to use the correct
class abbreviation.  For example:
\begin{verbatim}
START PRODUCTION
    PRODUCTION PL Earth
    ; Enter your production orders for planet Earth here.

        ;Build sub-light transport and pay for all of it now. Cost is 750.
        BUILD   TR10S Barrel of Monkeys

        ;Build 40,000 ton transport. Pay one-quarter now (100) and the
        ;  rest (300) later...
        BUILD   TR4     Tummy Tunes,    100

END PRODUCTION
\end{verbatim}

Note that if a payment amount is not specified, then the ship will be
completely built.  The computer will calculate the needed cost.  You will, of
course, have to keep track of the costs yourself to make sure that you don't
try to spend more than what you have.

To continue construction on a transport, use the CONTINUE command, as in the
following example:

\begin{verbatim}
START PRODUCTION
    PRODUCTION PL Earth
    ; Enter your production orders for planet Earth here.

        ;Finish the 40,000 ton transport we started in the last turn.
        ;  Since we only paid 100 then, we must now pay 300.
        Cont    TR4     Tummy Tunes

        ; Pay an additional 500 on the 200,000 ton transport we started a
        ;  few turns ago.
        CON     TR20 Tunnel of Love, 500

END PRODUCTION

\end{verbatim}

NEVER use the same name for two different ships/transports, even if their
classes are different.  For example, if frigate \texttt{FF Danny Boy} already
exists, then an order to build transport \texttt{TR8 Danny Boy} will fail.


\subsection{MORE ON SUB-LIGHT SHIPS}
\label{sec:moreonsublightships}


As mentioned above, sub-light ships have all of the capabilities of FTL ships
of the same tonnage.  However, their cost is 25\% less because they do not
have the engines that allow them to make interstellar jumps.  Thus, they
are primarily intended for local use.

Sub-light ships should add the letter \texttt{S} to their class abbreviations.  For
example, a sub-light frigate would have the class designator \texttt{FFS}, a sub-light
40,000 ton transport would use \texttt{TR4S}, etc.


\subsection{STARBASES}
\label{sec:starbases}


Starbases are essentially floating fortresses.  Unless towed, they cannot
move under their own power, but must remain in orbit around a planet.

Once construction has started, a starbase may be added to indefinitely.
However, the total tonnage cannot exceed the maximum tonnage allowed by
a species' Manufacturing tech level.  This limit is exactly the same as
for ships; i.e., the maximum tonnage of a starbase is 5000 times the
Manufacturing tech level.

Starbases are constructed in the same way as ships.  The player simply places
an order for a starbase of a specific tonnage, or for additional tonnage to
be added to an existing starbase.  The cost is the same as the equivalent FTL
ship tonnage; i.e., tonnage divided by 100.  Starbases must always be built
or incremented in multiples of 10,000 tons.  For example, it would cost
$20,000/100 = 200$ to built a starbase of 20,000 tons.  To increase its tonnage
to 50,000 tons would require an additional cost of $(50,000-20,000)/100=300$.
Thus, the amount spent must always be an exact multiple of 100.

A starbase built using the normal production capacity of a planet must be
built in orbit around that planet.  (We will discuss another way of building
starbases later.)

The carrying capacity of a starbase is determined as follows:
\[
	\textrm{Starbase Carrying Capacity}  =   \dfrac{\textrm{Tonnage}}{1000}
\]
Overall, the combat effectiveness of a starbase is exactly the same as a
warship of the same tonnage.  Thus, a 150,000 ton starbase and a 150,000 ton
destroyer would fight as equals.

\begin{informationnote}
A starbase is not maneuverable and is thus a sitting duck.  It is also limited
in its choice of targets.  This, however, is compensated for by more powerful
shield generators and weaponry.  Thus, it is probably best to think of the
tonnage of a starbase as an `effective' tonnage rather than an actual tonnage.
The actual tonnage will almost certainly be much higher.  By using an effective
tonnage in the game, we can easily compare the combat effectiveness of
starbases relative to other ships.
\end{informationnote}

\noindent Starbases in orbit around a planet may be towed into orbit around another
planet in the SAME star system.  It is not necessary to allocate individual
ships to do this - it is assumed that the starbase itself has sufficient
shuttlecraft to do it.  In the same way, starbases may also be towed up to one
parsec per turn at sub-light speeds; i.e. to an immediately adjacent sector.
We'll discuss how to do this later.

The class abbreviation "BAS" should always be used for starbases.  For example,
you could refer to a starbase as BAS Deep Space 9 or BAS High Guardian.

To build starbases, also use the BUILD command.  Since starbases must be built
in increments of 10,000 tons (which has a cost of 100), anything spent on
building a starbase must be an exact multiple of 100.  For example:

\begin{verbatim}
START PRODUCTION
    PRODUCTION PL Bakupa
    ; Enter your production orders for planet Bakupa here.

        ;Build a new 20,000 ton starbase...
        BUILD   BAS     Misty Na Goba, 200

END PRODUCTION
\end{verbatim} 

Note that a payment amount must ALWAYS be specified when constructing
starbases, and the amount must always be an exact multiple of 100.

To increase the size of an existing starbase, use the CONTINUE command, as in
the following examples:

\begin{verbatim}
START PRODUCTION
    PRODUCTION PL Bakupa
    ; Enter your production orders for planet Bakupa here.

        ; Increase size of starbase by 40,000 tons...
        CONT    BAS Misty Na Goba,      400

END PRODUCTION
\end{verbatim} 

NEVER use the same name for two different ships/starbases, even if their
classes are different.  For example, if frigate "FF Danny Boy" already
exists, then an order to build starbase "BAS Danny Boy" will fail.


\subsection{RESEARCH}
\label{sec:research}


Tech levels may be increased by spending equal amounts of raw material units
and production capacity on research.  You may spend any amount you wish on any
or all tech levels.  For example, you could spend 25 on Military tech level and
433 on Biology tech level in a particular turn.

There is no guarantee, however, that research will result in an increase in
a tech level.  The results of scientific research are never predictable.

When a tech level increases, the increased knowledge is available for use on
all of the planets owned by the species.  Thus, even though research may be
done on just one planet, its benefits are available to the entire species.

There is no limit on how high a tech level can be, but in practice tech levels
are unlikely to exceed 100.

You may spend any amount on any tech level.  The command to allocate resources
to research is RESEARCH.  For example:

\begin{verbatim}
START PRODUCTION
    PRODUCTION PL Deneb VII
    ; Enter your production orders for planet Deneb VII here.

        ; Spend 27 on Biology research...
        RESEARCH 27 BI

        Res     1255    LS      ; Spend 1255 on Life Support research.

END PRODUCTION
\end{verbatim} 

\noindent Use the abbreviations for tech levels listed in Table~\ref{tab:techabbrvs}.

\begin{table}[h]
\begin{center}
\begin{tabular}{|cl|}
\hline
\rowcolor{lightblue} \textbf{Abbr} &    \textbf{Name}   \\
\hline
	MI	&	Mining \\
	MA	&	Manufacturing \\
	ML	&	Military \\
	GV	&	Gravitics \\
	LS	&	Life Support\\
	BI	&	Biology \\
\hline
\end{tabular}
\caption{Technology abbreviations}
\label{tab:techabbrvs}
\end{center}
\end{table}

Spending on research does not guarantee success.  In general, the more you
spend, the greater your chance of success will be, and the greater the increase
is likely to be.  Keep in mind, though, that the process is very unpredictable.
Do not be disappointed if you spend a lot on research but experience no
increase in tech level, and do not be surprised if you spend very little
and experience a large increase.

It is also possible for a tech level to rise without spending funds on
research.  This increase comes from research done by the private sector.
In effect, the government gets some technology for free, just as the private
sector gets technology for free as a result of government research.  In
general, tech increases from the private sector will not be very large,
so you should not depend too much on them.

\begin{importantnote}
	You may not raise a tech level using research, nor will you
	receive free tech increases from the private sector, if your
	initial tech level is zero.  If a tech level is zero, you
	must first have the basics of the technology taught to you
	by another species using the \texttt{TEACH} command.
\end{importantnote}

\noindent We'll have more to say about the \texttt{TEACH} command later.


\subsection{ECONOMIC UNITS}
\label{sec:economicunits}


It is expected that there will eventually be a thriving galactic economy, with
lots of trade taking place between species.  This trade is transparent to
the player, who is primarily concerned with government and military matters.
Still, there must be a way to transfer wealth between planets owned by one
species and between different species, as, for example, when one nation on
Earth sends "aid" to another.  In FAR HORIZONS, this type of transfer is
done using a special type of item called an "Economic Unit".

Unlike other items, however, economic units are more like money or bank
balances, and may be transferred freely between planets without the need for
ships or cargo capacity.  Economic units may even be transferred between
species.

In Far Horizons, each species has the equivalent of a bank account which
contains zero or more economic units.

If a species has economic units, they may be spent just as if each unit were
the equivalent of 1 raw material unit and a production capacity of 1.  During
production on a planet, economic units owned by the species will be used
automatically (if available) if orders are given which require more than the
available production capacity of the planet.  For example, if a planet can
spend 500 using normal production, plus the species has 150 economic units,
then the planet can spend a total of 650 for construction of ships and other
items.  In other words, normal production capacity will be used up first, and
economic units will only be spent if there is insufficient normal capacity.

However, there has to be a limit on how many economic units a colony can spend.
For example, it makes no sense to try to spend a large sum of 'money' on a
small colony.  A small colony simply does not have an economy that is robust
enough to deal with large sums of 'money'.

So, in Far Horizons, the amount of economic units that a colony can spend in
addition to its normal production will be limited to what it can produce on its
own.  For example, if a colony can spend 850 using normal production, then it
may ALSO spend up to 850 economic units from the species' treasury, for a total
of 1700.  Thus, the robustness of the economy, measured by the amount of
economic units it can spend, will grow as the colony grows.

There is no limit to how much may be spent on a home planet.

Economic units may NOT be produced like other items using the BUILD command.
Instead, they are "produced" automatically on any planet that has unused raw
material units and an equal amount of unused production capacity.  The ``cost''
of one economic unit is 1 raw material unit and a production capacity of 1.
Thus, for example, if you need economic units to give to another species, then
simply do NOT spend an appropriate amount on one or more planets.

Note that there is no need to explicitly transfer economic units between
planets that you control.  Economic units owned by a species are available for
any planet that needs to spend them.  Later, we'll discuss how to transfer
economic units to another species.


\subsection{UPGRADES}
\label{sec:upgrades}


The successful operation of ships and other items often depends on the value of
a particular tech level.  For example, the effectiveness of a ship in combat
will depend heavily on its Military tech level.  For game purposes, everything
will function at the current tech levels for the species.  This is unrealistic,
but it makes bookkeeping much easier.

As a way of compensating for this lack of realism, ships and starbases will
also have an ``age'' associated with them.  This age will be equal to the number
of turns that have passed since construction finished.  For a starbase, the
effective age will be the weighted average of all of its contributions.  All
other items, including planetary defenses, will not experience any aging
effects.

The age of a ship will affect its operation as follows: whenever an operation
has a certain probability of success, that probability will be reduced by
a percentage equal to 2 times the effective age.  For example, if a newly
constructed ship has a 98.17\% chance of hitting an enemy target, then at the
age of 9 turns, its chance of success for the same shot would be 98.17 - (18\%
of 98.17) = 80.50\%.  When a ship reaches the ripe old age of 49, it will remain
at that age, apparently held together by spit and glue.  Obviously, a ship that
has reached the age of 49 will have a difficult time doing ANYTHING right!

In a similar way, the age will also affect the firepower of weapons and the
absorption power of shields.

Ages of ships and starbases will be listed in the status reports for the
species.

An item's age may be reduced by having it upgraded (i.e., it will undergo a
retrofit).  The amount of age reduction can be calculated with the following
formula:
\[
	\textrm{Age Reduction}  =  \dfrac{40  \times  \textrm{Amount Spent}}{\textrm{Original Cost}}
\]
Fractions will be dropped.

Or, if you'd rather start with the age reduction, then the corresponding cost
can be determined using the following formula:
\[
	\textrm{Cost of Upgrade}  =  \dfrac{\textrm{Desired Age Reduction}  \times  \textrm{Original Cost}}{40}
\]
If the result has a fraction, it should be rounded UP to the next whole number.
For a starbase, the original cost is considered to be the current tonnage
divided by 100.

For example, to completely upgrade an 80,000 ton transport that has an
effective age of 17 turns would cost $(17 \times 800) / 40 = 340$, and would reduce
its effective ``age'' to zero.  To reduce the age of a 70,000 ton starbase
from 34 to 10 would cost $(34 - 10) \times 700 / 40 = 420$.

To upgrade a ship or starbase, use the \texttt{UPGRADE} command, as in the following
examples:

\begin{verbatim}
START PRODUCTION
    PRODUCTION PL Nushki Pata Pata
    ; Enter your production orders for planet Nushki Pata Pata here.

        ;
        ; Let's keep that old light cruiser a little longer. If we spend 700,
        ;  we will reduce its age by 40 * 700 / 2000 = 14 turns.
        ;
        UPG CL Mighty Mouse, 700

        ; Rejuvenate that old destroyer from age 23 to age 0.
        upgr    DD Dawson       ; cost will be (23 x 1500)/40 = 863.

END PRODUCTION
\end{verbatim} 

If you do not specify the amount you wish to spend in the upgrade command, then
the age will be set to zero and the cost will be determined accordingly.

There is never a need to upgrade planetary defense units since they do not
experience aging effects.

A ship or starbase that is to be upgraded must be in the same sector as the
planet doing the upgrade, and the upgrade order must appear in the production
section for that planet.  A ship cannot jump and be upgraded in the same turn,
since both jumping and upgrading require a complete turn.

Finally, keep in mind that an upgraded ship will still get one year older
during the turn in which the upgrade is done.  Thus, when you receive the
status report for the turn, the age of the above destroyer will be 1, NOT 0.


\subsection{RECYCLING}
\label{sec:recycling}


Most items may be recycled.  When this is done, the item is effectively sold on
the open market, and an appropriate amount of 'money' is added to the treasury
of the species.  The amount received will depend on the item that is recycled.
Ships and starbases will have values that depend on their age, while most
other items can be cashed in for half of their original cost.

Mining and manufacturing bases may NOT be recycled.

Recycling will add economic units to the balance for the species.  For items
that do NOT suffer aging effects, the number of economic units gained will be
half the original cost (fractions will be dropped).  The only exception to
this is for raw material units, which will be cashed in at the rate 1:5 (for
example, recycling 29 RMs will generate 5 economic units).

For ships and starbases, the amount of economic units generated will be:
\[
		\dfrac{3  \times  \textrm{original cost}}{4}      \times \dfrac{(60 - \textrm{age})}{50}
\]

If a ship is still under construction, it may be recycled for half of what has
already been spent on it.  If a ship is carrying cargo, the cargo will first be
transferred to the planet before the ship is recycled (cargo, if any, is NOT
automatically recycled).

To recycle ships, starbases and other items, use the RECYCLE command, as in the
following examples:

\begin{verbatim}
START PRODUCTION
    PRODUCTION PL Knock Out
    ; Enter your production orders for planet Knock Out here.

        ; Let's recycle some stuff we don't need...
        ;
        ; Get rid of those old corvettes.
        ;
        RECYCLE         CT      Dragon
        recycle         CT      Princess
        ;
        ;Next, I don't need so many planetary defense units...
        ;
        rec     20 pd

END PRODUCTION
\end{verbatim} 

If colonist units or planetary defense units are recycled, the available
population for the planet will be increased by the number of units recycled.
This will allow you to convert \texttt{CU}s to \texttt{PD}s or vice-versa.

Recycling is a good way to get rid of old, unreliable ships and starbases.  It
is also good for getting rid of excessive amounts of raw material units.

An item, ship, or starbase that is to be recycled must be on the planet or in
orbit around the planet, and the recycle order must appear in the production
section for that planet.  Economic units generated by a RECYCLE command may be
spent in the same turn.  Make sure, though, that the recycle command precedes
any other commands that will spend the money obtained by recycling.

A ship cannot jump and be recycled in the same turn, since both jumping and
recycling require a complete turn.


\subsection{RECYCLE OR UPGRADE - WHICH IS BEST?}
\label{sec:recycleorupgrade}


An important decision that players will have to make is whether to upgrade a
ship or to recycle it.  There are several things to keep in mind when making
this decision:

\begin{enumerate}[a.]
\item A large warship is much more effective in combat than several
	smaller warships of the same total tonnage.  This is a strong
	incentive to recycle smaller warships, and use the proceeds to
	build larger ones.

\item Recycling is more cost effective as a ship gets older, but
	the aging effects could have a serious negative impact on the
	ship's operation.  In purely financial terms, the crossover age
	is about 15 turns; i.e., the fractional financial return from
	recycling is the same as the fractional remaining useful lifetime
	when the age of the ship is about 15 turns.

\item Starbases require such a long time investment, that it is
	never worthwhile to recycle them.  In general, they should always
	be upgraded.  The only time you can justify recycling a starbase
	is if you are forced to do so by an enemy, or if it was intended
	originally for only temporary use.

\item You may want to keep smaller, non-intimidating ships (such as
	small transports) for exploration or spying.  Thus, upgrading a
	small number of these could be advantageous.
\end{enumerate}


So, as a general rule-of-thumb, it's a good idea to recycle warships and
transports of 40,000 tons or more when they reach the age of about 15.  If
you're willing to sacrifice a small amount of the financial return for a little
more security, then recycle when the ship is slightly younger, say 10 or 12
turns old.

Starbases should almost always be upgraded.  Recycle them only when you have
no choice or when you no longer need them.

Small warships and transports (less than 40,000 tons) are great for exploration
and spying, and you may want to continually upgrade a few of them.  Recycle
only if you have more than you need.  If you explore or spy with anything
bigger (especially large warships), the aliens you visit may consider it a
hostile act.

Another thing to keep in mind is player tedium.  Providing orders for lots of
small ships can be a real pain in the neck.  Also, the more ships you have, the
more likely it will be that you'll make mistakes.

Finally, keep in mind that the above are just guidelines.  The ``personality''
of the species that you are role-playing can definitely impact your strategy.
And, as in any role-playing game, you should always role-play your species
correctly, even if ``correct'' means ``less efficient'' or ``less practical''.


\subsection{BUILDING OTHER ITEMS}
\label{sec:buildingotheritems}


To build items other than ships and starbases, also use the \texttt{BUILD} command, but
specify the number of items you want and their class abbreviation.  Here are
some examples:

\begin{verbatim}
START PRODUCTION
    PRODUCTION PL Earth
    ; Enter your production orders for planet Earth here.

        build   7 PD    ;Add seven planetary defense units...
        BUI 50 CU       ;Train and equip 50 colonist units...

        Build   3 jp    ; Build 3 jump portal units.

END PRODUCTION
\end{verbatim} 

We'll have more to say later about ``colonist units'' and ``jump portal units''.
Always be careful to use the correct class abbreviations in ANY orders.


\subsection{FLEET MAINTENANCE COST}
\label{sec:fleetmaintenancecost}


Manning and maintaining ships and starbases is not free, and costs can be
especially high for military vessels.  To reflect this reality, each species
will be required to pay a ``fleet maintenance cost''.

The fleet maintanance cost will be calculated by the computer and listed on
your status report.  The base cost for all military ships will be the tonnage
divided by 500, the base cost for starbases will be the tonnage divided by
1000, and the base cost for transports will be the tonnage divided by 2,500.
Sub-light ships will receive a 25\% discount.  For example, it will cost 400
per turn to maintain a 200,000 ton light cruiser, 60 per turn to maintain a
150,000 ton TR15, and 750 per turn to maintain a 500,000 ton sub-light
dreadnought.  The full cost must also be paid for ships that are still under
construction.  After calculation of the total base cost, a discount will be
applied equal to the current military tech level divided by 2, used as a
percentage (drop fractions).  For example, if your military tech level is 27,
then you will receive a 13\% discount.  In this way, those species that
``specialize'' more heavily in military technology will be able to operate their
fleets more efficiently.

The computer will calculate the percentage of the total production of all
planets that is needed to pay the fleet maintenance cost, and will subtract
that percentage from the total amount available for spending on each planet.
Thus, the player will NOT have to do any calculations at all - the cost will be
automatically deducted from the production of each planet.  Here is an example
of how the cost and deductions will appear on your status reports:

\noindent For the entire species, you will see a line like this:

	\begin{verbatim}
Fleet maintenance cost = 926 (7.34% of total production)	\end{verbatim} 

\noindent For each planet, you will see a line like this:

	\begin{verbatim}
Total available for spending this turn = 2278 - 167 = 2111	\end{verbatim} 

where the portion of the fleet maintenance cost that is being paid by this
planet is 167 (i.e. 7.34\% of 2278).

If the fleet maintenance cost is greater than the total production of all of
your planets, then the percentage will be greater than 100.  If this occurs,
then as much of the cost as possible will be paid using any economic units in
the treasury.  If the remaining cost is still greater than total production,
the amount over 100\% will be the percent chance of civil unrest, riots, and
destruction of infrastructure.  In other words, the population will get very
upset if the military budget becomes too excessive.



\section{MOVEMENT}
\label{sec:movement}

As soon as you have finished constructing at least one ship, you will be able
to give movement orders in the movement sections of the order form.  Jump
orders can be given only for ships and items that are listed in the status
report for the current turn.  You may not give jump orders for items that
will be produced in the current turn.  Other movement orders for newly
constructed ships may be given immediately after production, in the post-
arrival section of the order form.

Movement orders are of four types: jump orders, transfer orders, landing and
orbiting orders, and sub-light move orders.  These orders are described below.


\subsection{JUMP ORDERS}
\label{sec:jumporders}


Jump orders are given when a ship must be moved to a different star system.
These orders should use the \texttt{JUMP} command.  Here are some examples:

\begin{verbatim}
START JUMPS
; Place jump orders here.

        jump PB Benjamin Franklin, 12 7 18
        JUMP    FF      Thomas Edison, PL Mars

END
\end{verbatim} 

For the destination, use the name of a planet in the destination star system
whenever possible.  If you mistype a planet name, the computer will report an
error which the gamemaster may be able to fix.  However, if you mistype ``x y z''
coordinates, the ship will arrive at the wrong destination, if it arrives at
all.

If a planet name is used, then the ship will automatically go into orbit around
the planet when it arrives.  If X Y Z coordinates are used, the ship will
remain in the deep space part of the sector, even if the sector has planets.

If a ship or starbase is located at a terminus of a natural wormhole, then it
may use the wormhole to travel to the other terminus.  To do this, use the
WORMHOLE command, as in the following examples:

\begin{verbatim}
START JUMPS
; Place jump orders here.

        Wormhole        TR10 Praying Mantis
        WORM    BAS Deep Space 3, PL Danbury
        Wor     FFS Farragut
        Worm            BC Tanid's Sword, PL Vega III

END
\end{verbatim} 

Note that this is the only way that a starbase can ``jump'' to a different
sector at FTL speeds.

If an optional planet name is specified, then the ship or starbase will enter
orbit around the planet when it arrives at the other end of the wormhole.
Otherwise, it will remain in deep space.  (Obviously, if a planet is specified,
then it must be located at the same X Y Z coordinates as the other end of the
wormhole.)

When a location is scanned (discussed later), the scan will indicate if a
wormhole is present, but it will not indicate the coordinates of the other end
of the wormhole.  The only way to determine the other endpoint is to actually
use the wormhole, as described above.

Natural wormholes are absolutely stable.  There is no chance of a mis-jump or
self-destruction when using one, regardless of the distance traveled.


\subsection{TRANSFER ORDERS}
\label{sec:transferorders}


Transfer orders use the \texttt{TRANSFER} command, and are used to move items to and
from ships and planets in the same sector.  They must be given in the pre-
departure or post-arrival sections of your orders.

Transfers between planets in the same sector do not require the use of specific
ships.  It is assumed that there are sufficient shuttlecraft available.  For
example:

\begin{verbatim}
START POST-ARRIVAL
; Place post-arrival orders here.

        TRANSFER 100 RM PL Earth, PL Mars

END
\end{verbatim} 

In effect, the \texttt{TRANSFER} command is used to transfer goods between any two
entities that are capable of holding them, as long as the transfer occurs
within a sector.  Here are some more examples:

\begin{verbatim}
START PRE-DEPARTURE
; Place pre-departure orders here.

        TRAN 50 RM PL Earth, BAS Earth Orbit 3
        tra     2 RM    FF Gibbon,      CA Embassy
        TRANSFER 100 RM BAS Earth Orbit 2, BAS Mars Orbit 1
        Tran    50 PD   PL Earth, Pl Mars

END
\end{verbatim} 

If a planet is the source or destination in a transfer, the planet name MUST
be used - coordinates may NOT be used!

There is no limit to the number of \texttt{TRANSFER} commands that a ship or planet can
be given in a single turn.

A special note must be made about the \texttt{TRANSFER} command.  There is a possible
situation in which colonists and supplies could be transferred to a new colony
immediately after a jump.  If the planet is already inhabited by another
species, neither species will know about the new colony until the next turn.
To prevent this very unrealistic kind of incident, a \texttt{TRANSFER} to a planet may
only be made in the post-arrival phase IF the planet is already inhabited by
the species making the transfer.  Otherwise, the transfer will have to be
done in the pre-departure phase of the next turn.  Once the colony has been
established, you may \texttt{TRANSFER} goods to the planet in either the pre-departure
or post-arrival phases.

An optional feature of the \texttt{BUILD} command that was not discussed earlier allows
the player to provide a destination for the items that are built.  Here are
some examples:

\begin{verbatim}
START PRODUCTION
    PRODUCTION PL Earth
    ; Enter your production orders for planet Earth here.

        ; Build 120 colonist units and transfer them to a transport.
        Build   120 CU  TR6 Belly Laugh

        ; Build 150 planetary defense units and transfer them to the colony
        ;  on Mars.
        Bui     150 PD  PL Mars

END PRODUCTION
\end{verbatim} 

If the optional destination is a planet, then it must be in the same star
system as the producing planet.  If the optional destination is a ship, then
the ship must be in the system BY THE END OF THE TURN.  In other words, a \texttt{BUILD}
command with an optional destination is exactly equivalent to a \texttt{BUILD} command
in the production section of the orders, followed by a \texttt{TRANSFER} order in the
post-arrival section.  If the destination does not have sufficient cargo
capacity, then only items for which there is sufficient capacity will be
transferred.  All item transfers will be logged on the status report.  And
since this optional feature cannot have any destructive or irreversible
consequences, no error message will be posted if the transfer fails or is
incomplete.  If the transfer cannot be made, then the produced items will
simply remain on the planet.

\begin{importantnote}
	If you attempt to auto-transfer items to another planet in the
	same sector, and at least one of the planets is under siege, then
	the transfer will be ignored.  If a planet is under siege, you
	MUST use a separate \texttt{TRANSFER} command.  Also, keep in mind that
	post-arrival \texttt{TRANSFER}s will only work if the destination planet
	is already populated.
\end{importantnote}

\subsection{THE ``LAND'', ``ORBIT'', AND ``DEEP'' COMMANDS}
\label{sec:landorbitdeep}

The \texttt{LAND}, \texttt{ORBIT}, and \texttt{DEEP} commands are used for moving ships within a star
system.  The \texttt{LAND} command indicates that the ship should land on the surface of
a planet.  The \texttt{ORBIT} command indicates that the ship should enter orbit around
a planet.  The \texttt{ORBIT} command can also be used to have a starbase towed from
orbit around one planet into an orbit around another planet in the same star
system.  The \texttt{DEEP} command may be used to move a ship that is currently landed
or orbiting into deep space.  A \texttt{DEEP} order may NOT be given for a starbase.
\texttt{LAND}, \texttt{ORBIT}, and \texttt{DEEP} orders may be given in either the pre-departure or post-
arrival section of your orders.

\noindent Examples:
\begin{verbatim}
START PRE-DEPARTURE
; Place pre-departure orders here.

        Land    FF Don Quixote, PL Mars 
        ORB BAS Hurdy Gurdy,    PL Jupiter
        DEEP    DD Jeopardy
END
\end{verbatim} 

The destination in a \texttt{LAND} or \texttt{ORBIT} can be a planet number or a planet name.
(We will discuss how to name planets later.)

In FAR HORIZONS, ships that are given JUMP commands of the form \texttt{JUMP X Y Z}
do not actually land on a planet or go into orbit around a planet.  Instead,
they are simply located somewhere in the star system, and we refer to this
``somewhere'' as `deep space'.  Also, movement within a star system is considered
to be trivially easy in FAR HORIZONS.  Thus, there is no movement penalty or
advantage to being in deep space, in orbit around a particular planet, or
landed on a particular planet.

However, there is one circumstance where being landed on a planet can provide
an advantage.  If a ship owned by another species visits the system, it will
detect ships in orbit or in deep space, but NOT ships that have landed on a
planet that is populated by your species.  In effect, the population of your
colony or home planet can "hide" your ships from prying eyes.  Ships that are
under construction are always automatically hidden in this way.  However, if
another species also has a colony on the same planet, then your ships cannot be
hidden from them.

The \texttt{LAND} command will allow you to land a ship on one of your populated
planets, thus hiding it from alien view.  The \texttt{ORBIT} command will allow you to
place a landed ship in orbit, thus intentionally making it visible to others.
Here are some more examples:

\begin{verbatim}
START POST-ARRIVAL
; Place post-arrival orders here.

; Move the little corvette from the surface of Mars to Earth orbit for the
;  Klingons to see.
        orb     CTS Jiminy Cricket, PL Earth
; But, don't let them see our big ships...
        LAND    DN Faragut, PL Earth
        Lan     BS Wellington, PL Mars

END
\end{verbatim} 

A \texttt{LAND} or \texttt{ORBIT} command may only be issued to a ship that is in the same star
system as the planet.  A ship does not already have to be at the planet when a
\texttt{LAND} or \texttt{ORBIT} command is given - it just has to be in the same sector.
  
When you give a JUMP order of the form \texttt{JUMP ship, PL name}, the ship will
automatically go into orbit around the planet.

For the purposes of this game, a ship cannot land on uninhabited planets.
[Actually, it is certainly possible for ships to land on uninhabited planets,
but it does not perform a useful game function, and so it is not allowed.]
Also, in general, you may not land your ships on a planet that is not inhabited
by your species, even if it IS inhabited by one or more other species.  The
only exception to this is if a species that inhabits the planet has declared
you as an \texttt{ALLY}.  (We'll have more to say about the \texttt{ALLY} command later.)

If you want to land your ship on a planet that is inhabited by another species
that has declared you as an \texttt{ALLY}, you must use a planet number rather than a
planet name in the \texttt{LAND} order, even if you have given a name to the planet.
Here are some examples:

\begin{verbatim}
        Land    TR5 Jabberwocky, 5  ; One of our allies has a colony on
                              ;  planet 5.

        LAN     FF Kharsh Dukh, 3.  ; Let's see if they coinsider us an
                              ;  ally.
\end{verbatim} 

If there is at least one species that has population on the planet and that has
declared your species as an \texttt{ALLY}, then your ship will be allowed to land.  If
you are allowed to land by one or more species, then all of those species will
be notified that they granted you permission to land.  If you are not allowed
to land by ANY species, then all species that have population on the planet
will be notified that they denied you permission to land.

If your star system is attacked, ships on the surface of a planet will react
just as quickly as ships in orbit or in deep space.  In other words, there is
no advantage or disadvantage to being on the surface, in orbit, or in deep
space.  A ship does not have to be in orbit or on the surface to load or
unload goods.  A newly constructed ship will remain on the surface until told
otherwise.  If you land a ship on a planet that is populated by both your
species AND by one or more other species, then the other species WILL detect
your ship, even though it has landed.  In other words, if a planet is colonized
by more than one species, then ALL ships on the planet will be detected by ALL
species that populate the planet, even ships that are still under construction.
The same is true for planetary defenses.

Finally, if you do not specify a planet name or number in a \texttt{LAND} or \texttt{ORBIT}
command, the computer will check if the ship is already orbiting or landed
on a planet.  If so, it will use that planet.  For example:

\begin{verbatim}
START JUMPS
; Place jump orders here.

        Jump    DD Defiant, PL Earth
END

START POST-ARRIVAL
; Place post-arrival orders here.

        Land    DD Defiant
END
\end{verbatim} 

The above \texttt{LAND} command will land the ship on PL Earth.


\subsection{MISHAP PROBABILITIES}
\label{sec:mishapprobabilities}


Whenever a ship jumps from one star system to another, there is always a chance
of a mishap.  This section is provided for those players who would like to know
the actual probabilities involved.

The percentage probability that something will go wrong is:
\[
	\textrm{Mishap Probability}  =  \dfrac{\textrm{Distance}^2}{\textrm{Gravitics Tech Level}}
\]
The result is in percent.  For example, if the distance is 7 parsecs and the
Gravitics tech level is 4, then the mishap probability is $(7 x 7)/4 = 12.25\%$.
Note that the result is significant to two places after the decimal point.

If a mishap does occur, then the result will be either a mis-jump or self-
destruction.  When a mishap does occur, a second check is made using the same
probability.  If the second mishap also occurs, then the ship self-destructs.

Finally, don't forget that all success/failure probabilities are further
affected by the age of a ship.  The probability calculated above is for a ship
whose effective age is zero.


\subsection{THE MOVE COMMAND}
\label{sec:movecommand}


It is possible for a ship or a starbase to travel up to one parsec per turn at
sub-light speeds.  The ship can be either sub-light or FTL.  It is assumed that
a starbase is towed by its own shuttlecraft.

To do this, use the \texttt{MOVE} command.  \texttt{MOVE} orders may only be issued in the
jump section of your orders.  Here are some examples:

\begin{verbatim}
START JUMPS
; Place jump orders here.

        Move    BAS Sneakers, 5 12 17
        MOV     CC Ornery, 17 16 11
        mov     DDS Victory, 22 31 15

END
\end{verbatim} 

Only one coordinate (X, Y, or Z) may change, and it may not change by more than
+/-1.  Here are some examples:

\begin{verbatim}
        Okay:   from 15 16 21 to 15 16 22       Z increased by 1
                from 21 5 7 to 20 5 7           X decreased by 1
                from 31 15 15 to 31 16 15       Y increased by 1

        Wrong:  from 15 16 21 to 15 17 22       Y and Z both increased by 1
                from 21 5 7 to 19 5 7           X decreased by 2
                from 31 15 15 to 30 16 15       X decreased by 1 and Y
                                                 increased by 1
\end{verbatim} 

The move requires a full turn.  Thus, a ship or starbase can only be given one
MOVE order per turn.

Since sub-light travel does not involve use of a wormhole, there is no danger
of mis-jumps or self-destruction.



\section{COLONIZATION}
\label{sec:colonization}

One of the goals of most players will be to create colonies in other star
systems as well as in their home system.  How this is done will be explained
in this chapter.


\subsection{GENERAL CONSIDERATIONS}
\label{sec:generalconsiderations}


Before a colony can be established, a suitable planet must be found.  The
three major criteria used to determine the suitability of a prospective planet
are its temperature class, its pressure class and the constituents of its
atmosphere.  If any of these three criteria differ considerably from those of
the home planet, then a considerable amount of life support expertise will be
required if the colony is to survive.

Once a suitable planet has been found, the colony can be started by shipping in
colonists and the supplies and equipment they will need to set up the colony.
In its early stages, a colony will grow mainly by constant infusions from the
home planet or other larger colonies, since its population will be too low to
grow much on its own.  Eventually, though, the colony's population will become
large enough that additional people and materials will no longer have to be
shipped in from elsewhere.

\begin{importantnote}
	In general, you may NOT set up a colony on your own home planet
	or on the home planet of another species.  However, you MAY
	colonize the home planet of another species after you have
	completely destroyed the population by means of germ warfare or
	orbital bombardment (discussed later).  You may also re-colonize
	your home planet up to its former highest economic base if it
	was reduced by bombardment of germ warfare.
\end{importantnote}

This rule is designed to prevent three unrealistic situations:  1. Building
a colony on your own home planet so that you can increase your mining and
manufacturing bases above the 2\% limit per turn;  2. Doing the same on someone
else's home planet;  3. Sneaking onto someone else's home planet and installing
a colony before they can stop you.

I doubt if anyone would ever want to do number 3, unless it was an act of pure
mischief, but numbers 1 and 2 are ways of getting around the 2\% growth limit,
and might appeal to players who like to 'cheat' by taking advantage of
loopholes in the rules.


\subsection{DETERMINING A PLANET'S SUITABILITY FOR COLONIZATION}
\label{sec:determiningsuitability}


A colony may only be started if the Life Support tech level of the species is
high enough to handle the prevailing conditions at the planet.  If the Life
Support tech level is not high enough, then a colony may not be started.  Use
the following guidelines to determine how much life support is actually needed:
\begin{enumerate}[a.] 
     \item If the single gas required by the species is not present in
	the required range, 3 points of life support will be needed.

     \item If the atmosphere has any gases poisonous to the species,
	3 points of life support will be needed for EACH poisonous gas.

     \item For every point of difference between the home planet's temperature
	class and the colony's temperature class, 3 points of life support
	will be needed.

     \item For every point of difference between the home planet's pressure
	class and the colony's pressure class, 3 points of life support
	will be needed.
\end{enumerate}
The minimum Life Support tech level needed to allow creation of the colony is
the sum of all the above contributions.  For example, consider the following
data:

\begin{verbatim}
        Home:   tc=10 pc=10     NH3(29%),N2(47%),O2(24%)
        Colony: tc=9  pc=12     H2S(46%),O2(54%)

        Atmospheric Requirement: 14%-54% O2
        Gases Poisonous to Species: HCl,Cl2,SO2,H2S,Fl2,CH4
        Gases Harmless to Species: He,H2,H2O,NH3,N2,CO2
\end{verbatim} 

The requirement for O2 is just barely met, so no life support is needed for it.
However, the colony has one poisonous gas, H2S, so 3 points of life support
will be needed.  The temperature class difference is 1, so 3 points of life
support are needed.  Finally, the pressure class difference is 2, so 6 points
of life support will be needed.  Thus, a total of $0+3+3+6 = 12$ points of life
support are needed.  If the Life Support tech level of the species is 12 or
higher, then the colony may be started.

Finally, extremely large planets, such as gas giants, are so hostile to life
that nothing can survive on the surface.  However, these planets typically have
large numbers of satellites (i.e. ``moons'') which can be colonized instead.
Thus, for planets such as these, we can think of the life support requirements
as 'averages' for the entire planet and its system of satellites.  It is for
this reason also that the gravity of a planet is not a consideration.


\subsection{STARTING THE COLONY}
\label{sec:startingthecolony}


In order for a colony to be of any use, it must have people, raw materials,
and production capacity.  In FAR HORIZONS, these needs have been met by
implementing the items in Table~\ref{tab:coloitems}.

\begin{table}[h]
\begin{center}
\begin{tabular}{|clcc|}
\hline
\rowcolor{lightblue} \textbf{Abbr} &    \textbf{Name}  & Cost & Carrying Capacity \\
\hline
        CU  &    Colonist Units                &  1    &   1 \\
        IU  &    Colonial Mining Units         &  1    &   1 \\
        AU  &    Colonial Manufacturing Units  &  1    &   1 \\
\hline
\end{tabular}
\caption{Required items for colonization}
\label{tab:coloitems}
\end{center}
\end{table}

A colonist unit is the approximate equivalent of 1000 humans.

Colonist units, colonial mining units, and colonial manufacturing units are
used to establish the initial mining and manufacturing bases on a colony
planet.  Units are installed as follows:
\[
    1 \text{ colonist unit} + 1 \textrm{ colonial mining unit} = 0.1 \textrm{ mining base}
\]
\[
    1 \textrm{ colonist unit} + 1 \textrm{ colonial manufacturing unit} = 0.1 \textrm{ manufacturing base}
\]
In other words, a colonist unit consists of trained people, ready and willing
to work.  A colonial mining unit or a colonial manufacturing unit contains the
supplies and equipment they will need to do the particular job.

After colonial mining and manufacturing units have been transported to a new
colony, they must be installed; i.e., they are used to create new mining and
manufacturing bases, or to increase existing bases.  The installation is
started in the pre-departure section of the orders, but requires the entire
turn to complete.

Note that colonist units, mining units, and manufacturing units must be ON the
planet before they can be installed.  Thus, they must be transferred from the
ships that brought them to the planet's surface before they can be installed.
Furthermore, items may only be transferred to a planet that has a name.  For
this, use the NAME command.  For example:

	\begin{verbatim}
NAME 12 3 9 4	PL Epsilon Eridani IV	\end{verbatim} 

The above example will give the name ``Epsilon Eridani IV'' to the fourth planet
of the star system at coordinates X=12, Y=3, Z=9.  Note that the abbreviation
for planet \texttt{PL} is required.  (In FAR HORIZONS, names ALWAYS require use of
the appropriate class abbreviation.  There are no exceptions.)

\begin{warningnote}
	WARNING! One of the most common player mistakes is to accidentally
	omit a required abbreviation.  In Far Horizons, ALL ships and items
	have abbreviations, and ALL orders that refer to them must use the
	required abbreviation.  If not, the computer will reject the order.
\end{warningnote}

\noindent After colonial units have been transferred to the planet using the \texttt{TRANSFER}
command, they may be installed with the \texttt{INSTALL} command.  Here's a complete
example involving the \texttt{NAME}, \texttt{TRANSFER}, and \texttt{INSTALL} commands:

\begin{verbatim}
START PRE-DEPARTURE
; Place pre-departure orders here.

        Name    13 24 7 3       PL Dickory Dock

        Tra     50 cu   TR5 No-one Here, PL Dickory Dock
        Tra     22 iu   TR5 No-one Here, PL Dickory Dock
        Tra     28 au   TR5 No-one Here, PL Dickory Dock
        Inst    22 iu   PL Dickory Dock    ; Mining base will be 2.2
        INst    28 au   PL Dickory Dock    ; Manufacturing base will be 2.8
END\end{verbatim} 

Make sure that sufficient colonist units (\texttt{CU}s) are present on the planet before
installing the mining units (\texttt{IU}s) and manufacturing units (\texttt{AU}s) that will need
them.  Also, orders to transfer units to the planet will NOT automatically
install them.  If you do not give specific installation orders, the units
will simply sit on the planet's surface.

In other words, when you give an order to install 22 \texttt{IU}s, you are telling the
computer to combine 22 \texttt{CU}s and 22 \texttt{IU}s and increase the mining base of the
planet by exactly 2.2.  When the order is executed, the computer will reduce
the number of \texttt{CU}s and \texttt{IU}s by 22 each, and will increase the mining base by
exactly 2.2.

Alternatively, you can use the \texttt{UNLOAD} command, as in the following example:

\begin{verbatim}
START PRE-DEPARTURE
; Place pre-departure orders here.

        Name    13 24 7 3       PL Dickory Dock

        Orbit   TR5 No-one Here, PL Dickory Dock

        Unload  TR5 No-one Here
END\end{verbatim} 

This command will transfer all \texttt{CU}s, \texttt{IU}s, and \texttt{AU}s on the ship to whatever planet
it is located at (orbiting or landed).  After the transfer, it will then
automatically install as many mining and manufacturing units as it can,
starting with mining units, and including any colonist units, mining units,
and manufacturing units that were already on the planet.

The \texttt{ORBIT} command was required in the above example because the planet was just
named, and there was no way that the transport could have already been in orbit
around the planet.  If the planet had been named in an earlier turn, and if the
transport had jumped directly to the planet, then it would have automatically
orbited the planet and the above \texttt{ORBIT} command would not have been necessary.

A special note must be made about the \texttt{TRANSFER} command.  There is a possible
situation in which colonists and supplies could be transferred to a new colony
immediately after a jump.  If the planet is already inhabited by another
species, neither species will know about the new colony until the next turn.
To prevent this very unrealistic kind of incident, a \texttt{TRANSFER} to a planet may
only be made in the post-arrival phase IF the planet is already inhabited by
the species making the transfer.  Otherwise, the transfer will have to be
done in the pre-departure phase of the next turn.  Once the colony has been
established, you may \texttt{TRANSFER} people and goods to the planet in either the pre-
departure or post-arrival phases.


\subsection{AVAILABLE POPULATION}
\label{sec:availablepop}


As a colony grows, its population will increase by normal means.  This
population will be listed on status reports as a number of `available
population units'.  This population can then be used as follows:

\begin{quotation}
	If a colony builds planetary defense units or additional colonist
	units, they will also have an equivalent cost in 'available
	population units'.  For example, to create 17 planetary defense
	units on a colony will reduce the number of available population
	units by 17.\end{quotation} 

In other words, before you can ``build'' colonist units or planetary defense
units, you must first have enough people to hire and train for the job.
Thus, you can think of 'available population' as equivalent to the number of
population units that are available for hire.  This number will be relatively
low on colonies, but will be much higher on the home planet.

The number of available population units that are currently 'for hire' on each
planet will be listed in your status reports.

Note that the above rules do NOT apply to ships or other items built on a
colony planet.  In other words, building ships does not have an equivalent
``cost'' in colonist units.  (Crewing requirements are considered to be
insignificant.  Very large ships may require large crews, but small colonies
will not have the resources and production capacity to build such ships.)


\subsection{POPULATION GROWTH ON COLONY PLANETS}
\label{sec:populationgrowth}


In general, colonial populations increase at a much higher rate than on the
home planet.  In this game, we will use a base figure of 10\% per turn.  This
value will be modified downwards depending on how hostile the planet is to
your species.

If your Life Support tech level is exactly equal to the Life Support needed to
colonize the planet, then the colony will experience no growth.

At the opposite extreme, if no life support is needed, then population growth
will be 10\% (plus or minus small random fluctuations).

Any new growth is converted to `available population units', which you can then
use to BUILD colonist units (\texttt{CU}s) or planetary defense units (\texttt{PD}s).

The actual growth will be calculated by the gamemaster's computer, and the
current total population and available population will be listed in the status
report for each planet.

Finally, unused available population will NOT carry over into later turns if
they are not used to \texttt{BUILD} \texttt{CU}s or \texttt{PD}s.  Instead, the people will either find
other jobs locally, or will give up in disgust (because they can't find jobs)
and will look for better opportunities off-planet.  This approach will prevent
unrealistic accumulations of "idle" population.


\subsection{MINING COLONIES}
\label{sec:miningcolonies}


If a colony has a mining base that is greater than zero, but its manufacturing
base is exactly zero, then it will be considered a `mining colony'.  A mining
colony has the following special features:
\begin{enumerate}
	\item A mining colony will never produce `available population'.  The
	only way to increase the population is by bringing in colonists and
	installing colonial mining units.

	\item Raw material units produced each turn on the mining colony will be
	automatically ``sold'' and converted to economic units (3 raw material
	units = 2 economic units, fractions dropped).  In other words, the
	number of economic units generated will be two-thirds of the number
	of raw material units that are ``mined''.

	\item As raw materials are extracted from a mining colony, the mining
	difficulty of the planet will gradually rise.  The rise in mining
	difficulty will be proportional to the amount of material mined.
	Note that this increase in mining difficulty occurs ONLY on mining
	colonies.
\end{enumerate}
Economic units generated by a mining colony cannot be spent on the mining
colony itself (except for the \texttt{HIDE} command, discussed later).  Instead, they
are automatically added to the balance for the species.  They may be spent in
the same turn on other planets only if the \texttt{PRODUCTION} order for the mining
colony appears before \texttt{PRODUCTION} orders for the planets where the economic
units will be spent.

Thus, mining colonies allow a species to take advantage of planets that are
rich in resources but which are not suitable for normal life.  The raw material
units that they generate are automatically converted to cash which can then be
spent on producing planets.

A mining colony is a better investment than a normal colony as long as the
mining difficulty is less than 1.50.  When the mining difficulty becomes
greater than 1.50, the return on investment becomes less than for a normal
colony.  Thus, it's a good idea to convert a mining colony to a normal colony
(by installing colonial manufacturing units) when the mining difficulty
approaches 1.50.


\subsection{RESORT COLONIES}
\label{sec:resotcolonies}


If a colony has a manufacturing base that is greater than zero, a mining base
that is exactly zero, a gravity less than or equal to the home planet, and
requires less than 6 points of life support technology, then it will be
considered a `resort colony'.  A resort colony has the following special
features:
\begin{enumerate}
	\item  A resort colony will never produce 'available population'.  The
	only way to increase the population is by bringing in colonists
	and installing colonial manufacturing units.

	\item Each turn, the production capacity of a resort colony will be
	automatically converted to economic units at a rate of three-to-two.
	In other words, the number of economic units generated will be two-
	thirds of the production capacity.  For example, if the production
	capacity is 40, then it will be automatically converted to 26
	economic units.
\end{enumerate}

Economic units generated by a resort colony cannot be spent on the resort
colony itself.  Instead, they are automatically added to the balance for the
species.  They may be spent in the same turn on other planets only if the
\texttt{PRODUCTION} order for the resort colony appears before \texttt{PRODUCTION} orders for
the planets where the economic units will be spent.

Thus, resort colonies allow a species to take advantage of planets that are
especially suitable for life but which may have a high mining difficulty.
Their effective production capacity is automatically converted to cash which
can then be spent on producing planets.


\begin{importantnote}
	One of the most common mistakes that players make is to
	ignore the gravity of a planet when trying to create a
	resort colony.  The gravity MUST be less than or equal
	to the gravity of the home planet.
\end{importantnote}

\subsection{HIDDEN COLONIES}
\label{sec:hiddencolonies}


It is possible to hide a colony from detection by aliens.  This is done by
using special shielding, conducting as much activity as possible underground,
delaying or reducing activity when aliens are in the system, using evasive
tactics, and so on.  However, hiding a colony is expensive and it reduces the
overall efficiency of the colony.

In Far Horizons, you may indicate that a colony is to be hidden by placing a
\texttt{HIDE} order in the production section of your orders, as follows:

\begin{verbatim}
HIDE	\end{verbatim} 

The command takes no arguments.

The \texttt{HIDE} command will last only for the current turn.  If you wish to
permanently hide a colony, then you must issue the \texttt{HIDE} command each turn.

The cost of hiding is the sum of the mining and manufacturing bases of the
colony at the beginning of the turn.  For example, if the most recent status
report shows that a colony has a mining base of 25.7 and a manufacturing base
of 29.2, then the cost of hiding the colony will be:
\[
		25.7 + 29.2  =  54.9  =  54
\]
Note that fractions are dropped.

You may not hide a resort colony or a home planet.  You may only hide normal
colonies and mining colonies.  (In fact, a \texttt{HIDE} order is the only production
order that may be given to a mining colony.)  Also, you may not hide a planet
that is under siege (discussed later).

When a colony is hidden, it will not be listed on the status reports of other
species that are in the same star system.  Ships under construction and any
ships that have landed on the planet will also be undetected.  However,
starbases and ships in orbit WILL be detected.  Also, the colony WILL be
detected by a species that has population on the same planet.

If you suspect that a planet has a hidden colony, you can detect the colony by
transferring planetary defense units or colonist units to the surface during
the pre-departure phase of the turn.  However, they must remain on the surface
until at least the next turn in order to detect the hidden colony.

You MAY hide a colony that has population on it (i.e. colonist units and/or
planetary defense units) but which does not have any installed mining or
manufacturing bases.  Since the total installed base is zero, the cost is also
zero.  However, you will have to provide your own \texttt{PRODUCTION} order for the
planet.


\subsection{COLONY ATTRITION AND LOSSES}
\label{sec:colorattritionlosses}


When a new colony is started, it is normal to expect losses to be relatively
heavy compared to later in its development.  This higher-than-normal casualty
rate will have several causes: unfamiliar flora and fauna, unpredictable
climate, unexpected geological phenomena such as earthquakes and volcanoes, and
so on.  In addition, very small colonies cannot replace these losses through
new births because their population is too small.  For example, extremely small
colonies must avoid the potentially harmful effects of inbreeding, which can
greatly restrict their growth rate.

In Far Horizons, we will simulate the losses that can be expected in very small
colonies as follows:

\begin{quotation}
	If the total population of a colony is less than 50 population
	units (about 50,000 people), then there will be a fixed loss of
	exactly 1 population unit (about 1000 people) per turn.  If the
	total population is at least 50 population units, then no loss
	will occur.\end{quotation} 

This reduction in population will occur at the end of every turn, immediately
after normal population growth.  Thus, if the colony experiences a natural
increase in population, there will be no net loss.

The reduction will take place as follows:
\begin{description}

	 \item[]If the number of available population units is greater than zero, \\
		then it will be reduced by one;
 \item[]	otherwise, if the number of colonist units on the planet is greater than zero, \\
		then it will be reduced by one;
	 \item[]otherwise, if the number of planetary defense units on the planet is greater than zero, \\ 
		then it will be reduced by one;
	 \item[]otherwise, if the manufacturing base of the planet is greater than zero, \\ 
		then it will be reduced by 0.1;
	 \item[]otherwise, the mining base of the planet will be reduced by 0.1.
\end{description}
Obviously, if a planet has no population at all, then it will not experience
any reduction.

A reduction in colonist units, planetary defense units, mining base, or
manufacturing base will be indicated on the status report.  Reduction in
available population units will not be mentioned.

\subsection{THE PRODUCTION PENALTY}
\label{sec:productionpenalty}

A colony that has a Life Support tech level that is barely sufficient to
survive will not be useful for much else.  If, for example, the tech level
needed to survive is 27 and the actual tech level is also 27, then the colony
is on the borderline between extinction and survival.  We certainly cannot
expect such a colony to be very productive, no matter how large its mining and
manufacturing bases.  Instead, the entire output of the planet's mining and
manufacturing bases would be needed just to survive.  If, however, the tech
level needed is much less than the actual tech level, then the colony would
be able to thrive and be very productive.

In Far Horizons, we simulate this reality with what is called a ``production
penalty'' that will apply to all planets.  The penalty will depend on the actual
Life Support tech level of the species (\texttt{LS}) and the tech level needed to
survive on the planet (\texttt{LSN}).  The production penalty is calculated as follows:
\[
\textrm{Production Penalty}  =  \dfrac{100  \times  \textrm{Life support tech level needed}}{\textrm{Life support tech level}}\textrm{percent}
\]

For example, if the \texttt{LSN} value for the planet is 9 and the \texttt{LS} value for the
species is 36, then production capacity and raw material production will each
be reduced by 25\%.  If \texttt{LS} equals \texttt{LSN}, then the production penalty is 100\%,
and nothing at all will be available for spending - the entire output of the
planet's mining and manufacturing bases will be needed just to survive.

The penalty will apply to all planets, including mining colonies, manufacturing
colonies, and even the home planet.  However, since the life support tech level
needed on the home planet is zero, the penalty will also be zero.  (The only
way this could change is if the home planet is terraformed by another species.
We'll have more to say about terraforming later.)

The penalty will be calculated by the computer and printed on your status
reports, along with the net production values.  Although you will not have to
do any additional calculations, you should always keep the penalty in mind when
deciding on which planets to colonize.

\section{WARFARE}
\label{sec:warfare}

Since FAR HORIZONS is a role-playing game, it is quite possible for a species
to choose to be peaceful, and to live harmoniously with all of its neighbors.
Unfortunately, its neighbors may not be similarly inclined...


\subsection{PREPARING FOR BATTLE}
\label{sec:preparingforbattle}


Before a battle can begin, you must specify its location.  This is done with
the \texttt{BATTLE} command.  For example:

\begin{verbatim}
START COMBAT
; Place combat orders here.

    ;The following combat orders will be for the attack on the Romulan home
    ; planet in their star system at 12 4 3...
        BATTLE  12 4 3

        additional combat orders follow the BATTLE order

END\end{verbatim} 

Note that only the X, Y and Z coordinates are given.  Any planets that are to
be involved will depend on later commands.  A \texttt{BATTLE} command MUST be specified
before any other combat orders that will apply to the specified location.  In
effect, a \texttt{BATTLE} command states that a battle MAY take place at the specified
location, and that all of your forces at that location are on alert.  A single
\texttt{BATTLE} command may be followed by other combat commands, and all such commands
will apply to the location specified in the most recently executed \texttt{BATTLE}
command.  Thus, if you are attacking more than one star system in a turn, then
each set of combat orders will be preceded by an appropriate \texttt{BATTLE} order.


\subsection{COMBAT}
\label{sec:combat}


To indicate your intentions at a battle, use one or more \texttt{ENGAGE} commands.  Each
\texttt{ENGAGE} command requires an ``option'' argument to indicate the nature of the
engagement you are willing to take part in.  Some also require an additional
argument which specifies the planet where the option will apply.  You may
specify more than one \texttt{ENGAGE} command per battle.  Here are your options:

\begin{description}
 	\item[0] -	Defense in-place.  Do not attack anyone unless they are
		clearly hostile.  If fighting does start, fight only if
		the battle moves to your current location.  If battle
		moves to a new location, do not move with it.  This
		is the default option if you do not provide any \texttt{ENGAGE}
		orders.

	\item[1] -	Deep space defense.  Do not fight unless the opponent is
		clearly hostile.  If a fight is inevitable, try to keep the
		battle away from your planets.  This option should be used
		if you have planets you wish to defend and you wish to keep
		the battle away from them to avoid damage to civilians and
		civilian structures.  (Remember, ``deep space'' is anywhere in
		a star system that is not within fighting range of a planet.)

	\item[2] -	Planet defense.  This option requires an additional argument
		indicating the planet you are willing to defend.  This
		command should be given for each planet you wish to defend.

	\item[3] -	Deep space fight.  Attack the enemy in deep space and destroy
		as many ships as possible.  If the enemy is not willing to meet
		you in deep space, then no battle will take place.  (Remember,
		``deep space'' is anywhere in a star system that is not within
		fighting range of a planet.)

	\item[4] -	Planet attack.  This option requires a second argument to
		indicate the number of the planet that you wish to attack.
		This type of attack will do as much damage as possible to
		military targets, while minimizing damage to civilians and
		infrastructure.  With this option, an attacker will have
		to contend with starbases, planetary defenses, and ships
		on the ground or in orbit.

	\item[5] -	Planet bombardment.  This option requires an additional
		argument to indicate the number of the planet that you wish
		to attack.  Start with a planetary attack.  After you have
		successfully destroyed all of the enemy's ships, starbases,
		and planetary defenses, bombard the specified planet doing as
		much damage as possible to the population and infrastructure.

	\item[6] -	Germ warfare.  This option requires an additional argument to
		indicate the number of the planet that you wish to attack.
		Start with a planetary attack.  After you have successfully
		destroyed all of the enemy's ships, starbases, and planetary
		defenses, attempt to destroy all enemy life on the planet
		using germ warfare.

	\item[7] -	Besiege planet.  This option requires an additional argument
		to indicate the number of the planet that you wish to attack.
		Start with a planetary attack.  After you have successfully
		destroyed all of the enemy's ships, starbases, and planetary
		defenses, besiege the planet and extort 'taxes'.
\end{description}

If you give an \texttt{ENGAGE} order whose option is 3 or greater, you must also specify
the name or names of the enemy you want to fight with.  For this, use the
\texttt{ATTACK} command.  For example:

\begin{verbatim}
     Attack  SP Klingon
     ATT SP Romulan\end{verbatim} 

The above says that you will attack two species: the Klingons and the Romulans.
Note that you must issue a separate \texttt{ATTACK} command for each species that you
wish to attack.

You also have the option of attacking all species that you have declared as
enemies using the \texttt{ENEMY} command (discussed later).  To do this, give the \texttt{ATTACK}
command a zero argument, as in the following example:

\begin{verbatim}
ATTACK   0    ; Attack all declared enemies at this battle location.	\end{verbatim} 

If you want to attack both enemies and non-enemies, then give the above order
plus specific \texttt{ATTACK} orders for the non-enemies.

If you do not give specific orders to attack other species, then you will not
attack them unless they attack you.

\begin{importantnote}
    It is not possible for a battle to take place if only transports are being
    engaged.  In order for a battle to occur, at least one participant must
    have at least one warship, starbase, or planetary defense unit at the
    engagement location.
\end{importantnote}

Now, if there are other species at the battle location that you are NOT
attacking, but they have declared as \texttt{ALLY} a species that you ARE attacking,
then they will fight in the battle on the side of their \texttt{ALLY}.  In other
words, the effect will be the same as if you had given \texttt{ATTACK} orders for
them and they for you.

EXAMPLE \#1: Several Romulan ships appear in sector 4 12 3 where you (the
Humans) have two colonies (planet \#4 and \#7).  You're not sure if they will
attack, and you would prefer not to fight.  Also, if they do attack, you want
to keep them away from your colonies to avoid civilian deaths and destruction
of mining and manufacturing capacity.  Here are the combat orders you should
give:

\begin{verbatim}
START COMBAT
; Place combat orders here.

        BATTLE  4 12 3
          ENGAGE        1       ;Try to keep fight, if any, away from
                                ; colonies.
          ENGAGE        2 4     ;Protect planet \#4.
          ENGAGE        2 7     ;Protect planet \#7.

END\end{verbatim} 

EXAMPLE \#2: You are the Romulan visitors of the preceding example.  Your goal
is to attack the Humans at planet 4 and destroy all defenses there.  If this
succeeds, you then want to attack planet 7, and destroy its defenses.  Finally,
if everything goes well up to this point, bombard planet 7 (but NOT planet 4).
Here are the orders you should give:

\begin{verbatim}
START COMBAT
; Place combat orders here.

        Battle  4 12 3

          Attack        SP Human

          Engage        4 4     ;Attack planet \#4 and destroy its defenses,
                                ; but do NOT try to destroy civilians and
                                ; infrastructure.

          Engage        5 7     ;Attack planet \#7 and destroy its defenses,
                                ; then try to wipe out the population and
                                ; infrastructure by bombarding the planet.
END\end{verbatim} 

If you want to attack the enemy in deep space and still protect your planets,
then provide an \texttt{ENGAGE 2} command for each planet you wish to defend, and also
provide an \texttt{ENGAGE 3} command.  In other words, you definitely want to fight,
but you also want to keep the battle away from your planets for as long as
possible.

If a defending species provides orders to engage an attacker in deep space, but
the attacker has orders to attack a planet, then the following will apply:

\begin{quotation}
	If the defender has a higher military tech level than the attacker,
	then N rounds of combat will occur in deep space before the battle
	moves to the planet, where N is the difference in Military tech
	levels.  Otherwise, only one round will occur in deep space.\end{quotation} 

For example, if the Fizians (Military tech level = 22) wish to attack the Jubwa
(Military tech level = 27) home planet, but the Jubwas want to keep the battle
away from the planet for as long as possible, then $27-22 = 5$ rounds of combat
will occur in deep space.  Starting with the 6th round, the battle will be at
the Jubwa home planet (assuming the Fizians can still fight).

When a planet is attacked, considerable damage can also be done to non-
combatants and to surface structures.  This damage will be indicated on
status reports by reductions in mining and manufacturing bases.

In the following sections, we will discuss some of the above options in more
detail.


\subsection{ANNIHILATION}
\label{sec:annihilation}


After an attacker has destroyed all of the defender's ships, starbases, and
planetary defenses, he may choose to attempt to annihilate the population.
This can be accomplished in two ways: by heavy bombardment from space, or
by the use of germ warfare.

When attacking a planet by bombarding it from orbit, the actual damage done
will depend on the number and combat capability of the attacking ships, and on
the population and infrastructure of the now-defenseless planet.  If sufficient
power is applied, then the defenders can be completely destroyed.  As a rough
rule-of-thumb, a fleet of ten unenhanced Strike Cruisers (250,000 tons each)
with a military tech level of 50 will have sufficient firepower to wipe out
the entire population and infrastructure of a heavily populated planet.

Germ warfare is used when the attacker wishes to ensure complete annihilation
of the defender.  It is most useful if the attacker does not have or may not
have sufficient firepower to annihilate the population and infrastructure by
means of bombardment.  It is also useful if the planet is rich and the attacker
wants to loot it after destroying the inhabitants.

Germ warfare is carried out by the use of high tech devices called ``Germ
Warfare Bombs'', and the results will depend on the relative Biology tech levels
of the attacker and defender.  Germ warfare bombs will be discussed in more
detail later.

If germ warfare is used to successfully wipe out the population of a planet,
then the planet may be colonized by someone else (such as the attacker) in
the next turn.  However, the original mining and manufacturing bases are lost.
Instead, the planet is looted by the attackers and the resulting economic units
are added automatically to the balance for the attacking species.  (It is
assumed that the direct transfer of the planet itself from the defender to the
attacker is impractical because they are completely different species; that
is, mining and manufacturing bases for the two species are not compatible.
However, some of the original wealth of the defenders is obtained by looting.)

Note that germ warfare may not be successful, and the outcome will depend on
the relative Biology tech levels of the two species.  Bombardment, however,
will always do some damage.

If a home planet is bombed, it may be recolonized by the original inhabitants,
even if the damage was only minor.  However, the final economic base that
results from any installations cannot exceed the original base.  (In the case
of multiple bombings, the highest base at any time during the bombings will be
considered the original base.)  When the base eventually reaches its original
value, the home planet will be considered fully recovered.  During recovery,
the amount that may be spent on a bombed home planet will be limited, and the
available population will be less than normal.


\subsection{SIEGE}
\label{sec:siege}


After an attacker has destroyed all of the defender's ships, starbases,
and planetary defenses, he may choose to besiege the planet and make the
inhabitants pay `protection money' or 'taxes'.  In effect, the attacker is
blackmailing the planet, saying ``give us money or we will destroy you''.

The effectiveness of a siege will depend on the size of the planet's economy
and the number and sizes of ships that take part in the siege.  And since siege
is an inherently military operation, it will also depend on the relative
Military tech levels of the opponents.  An effectiveness rating will be
calculated as follows:

\[
	\textrm{Effectiveness}  =  \dfrac{\textrm{Total ship tonnage}  \times  \textrm{ML of attackers}}{\textrm{Production base}  \times  (\textrm{ML of defenders} + 1)}
\]
\[
	\textrm{Production base}  =  (\textrm{MI} \times \textrm{Mining Base})  +  (\textrm{MA} \times \textrm{Manufacturing Base})
\]

If a ship besieges more than one planet in a sector, then its effectiveness
will be divided by the number of planets that it is besieging.

Since a starbase does not have the maneuverability of a normal ship, it
will have an effective tonnage that is one quarter of its actual tonnage.
Transports will have NO effectiveness at all in a siege.  Also, if the
besieging species has planetary defense units on the planet that is being
besieged, then they will contribute to the effectiveness of the siege in
proportion to the equivalent `tonnage' of the defenses.  However, planetary
defense units on the surface of a planet are MUCH more effective against a
defeated population than against attacking enemy ships above the planet.  To
reflect this, each planetary defense unit will have an equivalent tonnage of
2000 tons FOR SIEGE PURPOSES ONLY; i.e. forty times more effective than when
attacking enemy ships.  Finally, planetary defenses may not conduct a siege
by themselves.  The besiegers must also have at least one warship or starbase
taking part in the siege to provide orbital support.

The effectiveness will be used as a percentage to determine the amount of
production lost by the besieged planet.  The maximum effectiveness is 95\%.
Also, 25\% of the lost production will be converted to economic units and
transferred to the besieging species.  (The remaining 75\% is considered lost
due to inefficiency, civil unrest, sabotage, reduced interplanetary and
interstellar trade, the cost of maintaining the siege, etc.)

For example, if the effectiveness is 37 and the planet can spend 1205 in
production for the turn that it is under siege, then the planet will only be
able to spend $1205 - 37\%$ of $1205 = 760$.  In addition, 25\% of the amount lost
will be automatically transferred to the species that is conducting the siege.
In the above example, 37\% of 1205 is 445, and 25\% of 445 is 111.  Thus, the
besieger will receive 111 economic units.  If more than one species besieges
the planet, then the amount will be divided among them according to the
relative effectiveness of each species.

The effectiveness will also be used to determine if the besieger detects and
prevents the construction of ships and planetary defense units by the besieged
species.  For example, if the effectiveness is 37\%, then there is a 37\% chance
that the besieger will detect if any ship construction occurs.  If so, the ship
will be destroyed.  Similarly for planetary defense units.

A species that is under siege maintains complete control of his planet, and may
give any orders that make sense.  However, some orders may not succeed because
of the siege.  Here are the basic rules that apply to sieges:

\begin{enumerate}
 \item Any attempt to build a starbase on a beseiged planet will ALWAYS
	be detected, and the starbase WILL be destroyed.

	\item Any construction on a ship, even partial construction, MAY be
	detected as described above.  If so, the ship will be destroyed
	before it can be used.

	\item Any construction of planetary defense units MAY be detected as
	described above.  If so, all units on the planet will be destroyed
	before they can be used.

     \item Any items other than ships, starbases, and planetary defense units
	can be built without fear of detection.

	\item Any attempt to transfer items between the planet under siege and a
	different planet in the same star system MAY be detected as described
	above.  If so, the items will be destroyed.

	\item Besiegers will not detect transfers of economic units to other
	species.

	\item Any attempt to secretly land a ship on a besieged planet MAY
	be detected as described above.  If a landing is detected, the
	besiegers will be notified of the landing, but will not otherwise
	see the ship on their status reports.  The player that owns the
	ship will be told one of the following:  a. the landing WAS detected;
	b. the landing was NOT detected;  or c. the landing MAY have been
	detected.
\end{enumerate}

Besieging ships must remain in the system for the entire turn.  Any ships that
jump out of the system voluntarily or which are forced to jump away during
combat will not take part in the siege.


\begin{importantnote}
	Keep in mind that you may not \texttt{BUILD} and auto-transfer items
	to another planet in the same sector if either planet is under
	siege.  Instead, you must use the \texttt{TRANSFER} command.  Also keep
	in mind that post-arrival \texttt{TRANSFER}s will only work if the
	destination planet is already populated.
\end{importantnote}

A player that controls a planet that is under siege or which may be placed
under siege should keep in mind that some of his production may be stolen by
the enemy.  Thus, production orders should be listed in order of importance.
Items will not be built if there are insufficient funds.  For example, if your
planet is successfully attacked and besieged at the beginning of the turn,
then a percentage of your production will be lost.  Only the remainder will be
available for you to spend.  Keep this in mind if your system is 'visited' by
a potential enemy.

A player that wishes to maintain a siege for more than one turn MUST provide
appropriate battle orders EVERY turn, just as if the siege were being attempted
each turn.  At the beginning of each combat phase, the gamemaster's computer
effectively `forgets' about any previous sieges.


\subsubsection{OCCUPATION}
\label{sec:occupation}


Even though Far Horizons does not have any special rules that deal exclusively
with `occupying' a planet of another species, it IS possible to achieve a very
realistic kind of occupation using the existing rules.  Consider the following
scenario:

\begin{quotation}
	SP Good Guys and SP Bad Guys each have a colony on the 2nd planet
	of the star system at 7 14 23.  At the beginning of the turn,
	the Bad Guys have several warships in the system, plus planetary
	defense units at their colony.  If the Bad Guys want to 'occupy'
	the colony of the Good Guys, they could give the following orders:

\begin{verbatim}
            START COMBAT
            ; Place combat orders here.
                Battle  7 14 23
                Attack  SP Good Guys
                Engage  7 2 ; Attack and attempt to besiege
                            ;  planet 2.
            END\end{verbatim} 
\end{quotation} 

If the Bad Guys are successful at eliminating the ships and defenses of the
Good Guys, then the siege will commence.  Furthermore, the planetary defense
units in the Bad Guy colony will add to the overall effectiveness the siege.
This is especially important since planetary defense units are much more
effective at siege than at fighting ships above the planet.

Now, since at least one ship or starbase is needed to maintain a siege, the
Bad Guys can build a small starbase at the colony during the production phase
of the turn (if they do not already have one).  They can also build lots of
additional planetary defense units to help maintain the siege.  In effect, they
have 'occupied' the colony of the Good Guys.  Still another option would be to
build sub-light warships.  This approach will release the FTL warships for duty
in other star systems.

Note that it's perfectably acceptable for the Bad Guys to specify planet \#2
in an ATTACK order, even though there is a Bad Guy colony on the planet.  The
computer will never allow a species to attack itself (unless, of course, you
decide to commit suicide and give an ATTACK order against your own species).

Now, consider another scenario:
\begin{quotation}
	SP Good Guys have a colony on the second planet of the star system
	at 7 14 23.  SP Bad Guys do NOT have a colony on the planet, but
	they would like to conquer and 'occupy' the Good Guy colony.  So,
	in the previous turn, several warships AND TRANSPORTS jumped to the
	colony.  The transports contain colonist units, mining units, and
	manufacturing units.  The Bad Guys can now give the following orders:

\begin{verbatim}
            START COMBAT
            ; Place combat orders here.
                Battle  7 14 23
                Attack  SP Good Guys
                Engage  7 2             ; Attack and attempt to besiege
                                        ;  planet 2.
            END


            START PRE-DEPARTURE
            ; Place pre-departure orders here.
                Name 7 14 23 2  PL Revenge
                Transfer CUs, IUs, and AUs from transports to PL Revenge
                Install IUs and AUs on PL Revenge
            END
\end{verbatim} 
\end{quotation}
The above orders will start a new Bad Guy colony if the Bad Guys win the
battle.  If the bad guys lose, then the \texttt{TRANSFER} and \texttt{INSTALL} orders will be
ignored by the computer, since the transports will have been destroyed.

Now, consider a third scenario:
\begin{quotation}
	SP Good Guys have a colony on the second planet of the star system
	at 7 14 23.  SP Bad Guys do NOT have a colony on the planet, but
	they would like to conquer and 'occupy' the Good Guy colony.  So,
	in the previous turn, several warships AND TRANSPORTS jumped to the
	colony.  The transports contain lots of planetary defense units.
	The Bad Guys can now give the following orders:
\begin{verbatim}
            START COMBAT
            ; Place combat orders here.
                Battle  7 14 23
                Attack  SP Good Guys
                Engage  7 2             ; Attack and attempt to besiege
                                        ;  planet 2.
            END

            START PRE-DEPARTURE
            ; Place pre-departure orders here.
                Name 7 14 23 2  PL Revenge
                Transfer PDs from transports to PL Revenge
            END
\end{verbatim}
\end{quotation}

If the Bad Guys win the battle, the above orders will start a new Bad Guy
colony that contains ONLY planetary defense units.  If the Bad Guys bring
enough \texttt{PD}s, then the \texttt{PD}s can contribute significantly during the subsequent
siege.

The above approaches are very realistic ways of effectively `occupying' a
planet inhabited by another species.  The planetary defenses of the besieger
play the role of an occupying army, while at the same time defending the colony
of the occupier.  If the siege has a high effectiveness rating, then the
besieged species is in a quandry, especially if it has no hope of outside help.
Also, keep in mind that it's very difficult to hide a partially constructed
ship from the besieger.  Since the besieger has people on the same planet, the
chance of detecting ship and PD construction will be 95\%, regardless of the
actual siege effectiveness.


\subsubsection{ASSIMILATION}
\label{sec:assimilation}


Another very real kind of occupation can occur if the besieger actually lives
on the besieged planet, shipping more and more of their own colonists there as
time goes on.  Eventually, the original population becomes 'assimilated'.

It is not realistic for a species to assimilate an entire home planet, since
the population to be assimilated is simply too large.  (Consider the failed
Cardassian occupation of the Bajoran home world in the Star Trek stories.)
However, it should be possible to assimilate the population of a colony.
Something similar is actually happening today, here on Earth.  For example, the
Indonesians have essentially occupied and are now assimilating the people of
East Timor, while the Chinese are doing the same in Tibet.  In situations such
as these, a relatively small occupying population is in complete control of a
much larger native population because the occupiers control all of the weapons,
the government, industry, communications, the economy, and so on, while
continually shipping in more and more of their own people.

In Far Horizons, a colony will be considered `assimilated' if the besieging
species has a colony on the same planet, if the total population of the
besieging species is more than 20 percent of the population of the besieged
colony, and if the effectiveness of the siege is exactly 95 percent (i.e. the
highest possible value).  Also, only population due to the installed mining and
manufacturing bases will be counted.  (Population due to available population
units, planetary defense units and uninstalled colonist units will NOT be
counted, since they can be removed from the planet after assimilation takes
place.)  Also, only the effective economic bases will be used; i.e., any base
over 200.0 will have only 5\% of its actual value.  (The purpose of this rule
is to prevent the unrealistic situation in which a besieged colony artificially
pumps up its economic base to prevent assimilation.)

When these conditions have been met, HALF of the mining and manufacturing
base of the besieged colony will simply be transferred to the colony of the
besieger, and the besieged colony will, in effect, cease to exist.  (The
remaining half is considered lost due to conversion inefficiency, low morale,
guerrilla resistance, costs of dealing with a hostile population, etc.)

For example, the Human colony on Vega III is besieged by the Klingons, and the
overall effectiveness of the siege is 95 percent.  On Vega III, the installed
population is 1,180,000 (mining base is 61.8 and the manufacturing base is
56.2).  The Klingons also have a colony on the same planet, named Kitomer, with
a total installed population of 250,000.  Since the conditions for assimilation
have been met, the computer will increase the mining base of Kitomer by 30.9
and the manufacturing base by 28.1.  In addition, any inventory on Vega III
will be transferred to Kitomer.  The Human colony Vega III will effectively
disappear.  From that point on, the Klingon player has complete control of
the Human population on the planet.  In effect, the Humans have become
`assimilated'.

If the siege is being conducted by more than one species that have colonies
on the planet, then the mining and manufacturing base will be allocated in
proportion to the relative siege effectiveness of each species.

Some players may object to the 20\% population ratio, saying that 20\% is too
large, and that a value such as 10\% or even smaller would be more realistic.
Keep in mind, though, that here on Earth we are all humans on our home planet.
Assimilating an alien species on an alien planet should be much more difficult.


\subsection{SURPRISE}
\label{sec:surprise}


In some situations, it may be possible for one species to take another
completely by surprise.  For example, an ally that has ships in your system
may attack unexpectedly.  If surprise occurs, the attacker will be given one
free round of attacks.  Also, the defender's shields will be down during the
surprise round.  (In Far Horizons, it is assumed that shields are never in use
unless absolutely necessary.  Thus, when a ship is taken by surprise, it will
always be assumed that the shields are down.  Note that this is similar to the
way shields are used in the Star Trek stories.)

Surprise can only occur if you are attacked by a species that you have declared
to be an \texttt{ALLY}.

If you suspect potential treachery, then give a \texttt{BATTLE} order for the location,
and also give any appropriate defensive \texttt{ENGAGE} orders.  When this is done,
it will be assumed that you are on full alert, and will not allow any allies
to have a free round of surprise.  Everyone at the location, however, will
definitely be aware of your heightened state of alert.


\subsection{BETRAYAL}
\label{sec:betrayal}


If you are attacked by a species that you have declared \texttt{ALLY}, the status of
the attacker will automatically change to \texttt{ENEMY} after the combat or strike
phase in which the betrayal occurred.  In effect, an \texttt{ENEMY SP Attacker} order
will be issued for you.

If an \texttt{ALLY} is attacked by another \texttt{ALLY}, the status of the attacker will
automatically change from \texttt{ALLY} to \texttt{ENEMY}.  In effect, an \texttt{ENEMY SP Attacker}
order will be issued for you.

[We'll discuss how to declare enmity or alliance towards other species later.]


\subsection{COMBAT LOG}
\label{sec:combatlog}


A complete log of the the fighting that takes place during a battle will be
sent to the players of the species involved.  The log describes who fires on
whom, and whether or not the shot was successful.  Since a battle contains
many misses, these logs can become very long and very boring.  If you want to
receive only a brief summary of the results, then provide a \texttt{SUMMARY} command
after the \texttt{BATTLE} command.  The \texttt{SUMMARY} command has no arguments.  Here is an
example:

\begin{verbatim}
START COMBAT
; Place combat orders here.

        BATTLE  4 12 3
          SUMMARY               ;Log only the important stuff.
          ENGAGE        1       ;Try to keep fight, if any, away from
                                ; colonies.
          ENGAGE        2 4     ;Protect planet \#4.
          etc.

END\end{verbatim} 

If a \texttt{SUMMARY} command is used, then only the important combat news will be sent
to the player.  Various failures, misses, and hits that do not completely
destroy a ship or planetary defenses will not be reported.

The gamemaster always has the option of turning on summary mode if he feels
that the logs are too long.

During a battle, it's possible for cargo to be destroyed if damage passes
through the shields of a ship.  Since this happens quite frequently, it is
NEVER reported in the combat log.  You will only know about it after the battle
when you notice that some or all of the cargo is missing.


\subsection{DAMAGE}
\label{sec:damage}


In FAR HORIZONS, damage done in combat is indicated by increasing a ship's age.
Thus, when a ship's age reaches 50, it is considered destroyed, since this is
the age at which a ship becomes 100\% disfunctional.

Ships which survive a battle, but which are badly damaged, can be repaired by
using the \texttt{UPGRADE} command, or can be ``cashed in'' by using the \texttt{RECYCLE} command.
Keep in mind, though, that these commands can only be used on planets that have
production capacity.  If the ships are badly damaged and are not in the same
system as a producing planet that you control, they may have to jump to such
a planet, and may self-destruct in the process.


\subsection{WITHDRAWING FROM COMBAT}
\label{sec:withdrawingfromcombat}


During the battle, your ships will have the option of withdrawing based on
conditions that you set using the \texttt{WITHDRAW} command.  If ships do withdraw, they
will jump out of the sector to safety.  You can also specify a rendezvous
point using the \texttt{HAVEN} command.  The \texttt{HAVEN} command has the following format:

\begin{verbatim}
	Haven	x y z\end{verbatim} 

where \texttt{x}, \texttt{y}, and \texttt{z} are coordinates where a ship should jump to if it
decides to withdraw from battle.  If no \texttt{HAVEN} order is given, then the ship
will jump to a randomly selected location very close to the battle sector.
If a ship withdraws, then any explicit \texttt{JUMP} orders that appear in the jump
section of your orders will be ignored.  (It is assumed that a ship that
withdraws from combat is likely to be damaged and that a normal, longer jump
will be too dangerous.)

You may specify the conditions of withdrawal using the \texttt{WITHDRAW} command.  Here
is the format:

\begin{verbatim}
	Withdraw	n1 n2 n3
\end{verbatim} 

where:
\begin{quotation}
		\noindent n1 = maximum acceptable transport age \\
		n2 = maximum acceptable warship age \\
		n3 = maximum acceptable fleet loss percentage \\
\end{quotation} 

The ``fleet loss percentage'' includes ships that have been destroyed or that
have already withdrawn.

Here's an example:

\begin{verbatim}
START COMBAT
; Place combat orders here.

    ;The following combat orders will be for the attack on the Romulan home
    ; planet in their star system at 12 4 3...

        Battle          12 4 3
        Haven           10 3 3
        Withdraw        10 25 50

END\end{verbatim} 

In the above example, any ships that decide to withdraw will attempt to jump to
sector 10 3 3.  An individual transport will withdraw if its age is greater
than 10.  An individual warship will withdraw if its age is greater than 25.
The entire fleet will withdraw if at least 50\% of the ships in the fleet have
either been destroyed or have already withdrawn.

For the special case where the transport value is EXACTLY zero, transports will
only withdraw if the entire fleet withdraws.  For the special case where the
fleet value is EXACTLY zero, the fleet will attempt to withdraw as soon as
possible after ANY battle begins.

If no \texttt{WITHDRAW} order is given, the default will be:

\begin{verbatim}
	Withdraw	0 50 50\end{verbatim} 

Thus, the default is to fight to the death.

If a ship withdraws during the strike phase, then the ship name will contain
the designation "WD" in the status report.  These ships will jump automatically
in the jump phase of the next turn, and any explicit jump orders will be
ignored.


\subsection{AMBUSH}
\label{sec:ambush}


If you suspect that one or more other species may jump into your system to do
battle, you may make advance preparations to `ambush' them.  You can do this by
providing an \texttt{AMBUSH} order.  However, unlike other combat commands, the \texttt{AMBUSH}
command is given in the \texttt{PRODUCTION} phase of your orders - NOT in the combat
phase.  The \texttt{AMBUSH} command takes a single argument which indicates how much you
will spend to prepare for the ambush.  For example, if you suspect that the
Cardassians and their allies will jump to your Vega IV colony system during the
current turn in order to attack you in either the strike phase of the current
turn or in the combat phase of the next turn, you can issue the following order
in the production section for PL Vega IV:

\begin{verbatim}
START PRODUCTION
    PRODUCTION PL Vega IV
    ; Place production orders here for planet Vega IV.

        Ambush  250

END\end{verbatim} 


This indicates that you expect enemies to arrive in your sector before the end
of the turn.

To carry out the ambush, you should give appropriate combat orders in the
strike phase of the turn, as in the following example:

\begin{verbatim}
START STRIKES
    ; Place strike orders here.

        Battle  12 9 23
        Attack  0       ; Attack any declared enemies that arrive.
END\end{verbatim} 


You may also give appropriate \texttt{ENGAGE} orders if you wish to continue fighting
after the initial ambush.

An ambush will provide your species with a number of `free' attacks that will
damage and perhaps destroy enemy ships before any battle even begins.  These
attacks will take place in deep space as the enemy ships arrive in your system.
The amount of damage done to the enemy will depend on how much you spent for
the ambush; that is, how well you prepared.

Damage to the enemy will be measured in `Effective Aging', which is calculated
as follows:
\[
	\textrm{Effective Aging}   =   \dfrac{10,000 \times \textrm{Amount Spent}}{\textrm{WT}  +  \textrm{TT}/10}
\]
where	
\begin{quotation}
     \noindent \texttt{WT} = The total enemy warship tonnage in the star system immediately before combat begins. \\
     \texttt{TT} = The total enemy transport tonnage in the star system immediately before combat begins.
\end{quotation}
The `Effective Aging' value will be added to the actual age of each enemy ship
immediately before any battle begins.  If the resulting age is greater than 49,
then the ship is destroyed.
\begin{quotation}
	Example:  You expect that the Klingons will jump to the Earth system
	during the current turn.  In your production orders for PL Earth, you
	provide the order \texttt{AMBUSH} 710.  The Klingons arrive with three warships:
	an \texttt{FF} (100,000 tons), a \texttt{DD} (150,000 tons), and a \texttt{TR7} (70,000 tons).
	Thus, the `Effective Aging' value will be:

\[
		\dfrac{10,000  \times  710}{100,000  +  150,000  +  70,000/10} =  27.63  = 27
\]

\end{quotation}

Note that fractions are dropped.

Thus, the age of each enemy ship will be increased by 27 turns immediately
before the battle, if any, begins.  In a situation such as the above, the
ambush could be absolutely crippling.

In order for the ambush to take effect, you must issue an appropriate \texttt{BATTLE}
order and one or more \texttt{ATTACK} orders during the strike phase, so that the
computer will know who you consider an enemy.  You may also provide any
appropriate offensive and/or defensive \texttt{ENGAGE} options.  Any normal combat
will occur \texttt{AFTER} the ambush.

If the enemy does not arrive or if you do not issue any \texttt{ATTACK} orders, then the
amount spent for the ambush will have been wasted.

A planet that is under siege may NOT issue an \texttt{AMBUSH} order.

Amounts spent on an ambush are cumulative.  Thus, if two or more planets in the
same star system issue \texttt{AMBUSH} orders, the effectiveness will be based on the
total amount spent.  Results are also cumulative if two or more species
collaborate in an ambush.

The actual details of the ambush are not important, since Far Horizons is a
strategic game - not a tactical one.  However, an ambush could involve ships
waiting in hiding among asteroids, use of space mines, subterfuge, and so on.

Since an ambush is essentially a way to `purchase victory', there must be a
realistic limitation on how much can be achieved.  It makes no sense, for
example, for a rich but poorly defended colony to be able to wipe out a
powerful invasion force.  In Far Horizons, we will simulate this reality as
follows:

\begin{quotation}
	The amount of effective aging that results from an ambush will
	be limited by the ratio of the warship tonnages present at the
	ambush.
\end{quotation} 

For example, if the warship tonnage of the ambushing forces is on par with
the warship tonnage of the ambushed forces, then the full ambushing effect
described above will take place.  If, however, the ambushing forces are
considerably weaker than the ambushed forces, then the net effective aging
will be reduced.  And if the ambushing forces are considerably stronger than
the ambushed forces, then the effective aging will be increased.

Finally, it is much easier to ambush enemy ships if they enter the sector via a
natural wormhole, since the ambushing species knows the precise exit point and
can concentrate more of its forces at that point.

To simulate this reality, a ship that is ambushed as it exits a natural
wormhole will experience DOUBLE the calculated age increase.  Thus, if any of
the ships in the above example had arrived via a natural wormhole, then their
ages would have been increased by $2 \times 27 = 54$, which would have immediately
destroyed them.


\subsection{INTERCEPTION}
\label{sec:interception}


There may be times when you wish to intercept and destroy a ship that jumps
into one of your star systems before it can scan or learn anything about your
inhabited planets.  For this purpose, use the \texttt{INTERCEPT} command.

The intercept command is like the \texttt{AMBUSH} command because it appears in the
production section of your orders and because you must specify the amount you
are willing to spend for the interception.  However, it is different because
the results are applied immediately --- you will attempt to destroy any enemy
ship that jumps into your system the moment it arrives, rather than wait until
the combat phase of the next turn.

It is also different because an ambush implies advance knowledge of who is
coming and when they are coming.  Interception implies a general state of alert
without any advance knowledge.  Thus, interception is inherently more difficult
than ambushing, and has a higher cost for the same results.

When an \texttt{INTERCEPT} order is given in the production phase of your orders, it
will apply immediately to any enemy ships that enter the system during the
current turn.  You can specify which species are enemies using the \texttt{ENEMY}
command.  The \texttt{ENEMY} command can appear in either the pre-departure, production,
or post-arrival section of your orders.  Here are some examples:

\begin{verbatim}
START PRE-DEPARTURE
; Place pre-departure orders here.

        Enemy   SP Klingon
        ENEMY   SP Romulan
        enem    999             ; ALL species are my enemies.

END\end{verbatim} 

The \texttt{ENEMY} command takes a single argument, either a name of a species or a
number.  If it is a number, it states that all species in the game (even those
you have not yet met) are your enemies.  Otherwise it states that the indicated
species is your enemy.  You may specify an enemy command for as many species as
you want.  When the argument is a number, any number may be used --- the actual
value is not important.

An \texttt{ENEMY} command remains in effect until cancelled.  You can cancel an \texttt{ENEMY}
command with either the \texttt{NEUTRAL} or \texttt{ALLY} commands.  These also can appear in
either the pre-departure, production, or post-arrival phase of the turn:

\begin{verbatim}
START POST-ARRIVAL
; Place pre-departure orders here.

        Neutral SP Klingon
        Neutral 7               ; I am neutral towards everyone.
        Ally    sp klingon      ; The Klingons are now my allies.

END\end{verbatim} 

Again, a numeric argument (regardless of value) may be used to indicate
neutrality towards ALL species in the game, even those you have not yet met.
However, a numeric argument may NOT be used with the \texttt{ALLY} command.  You can
name one and only one species in an \texttt{ALLY} command, and you must have already
met the species before you may declare alliance.

If you wish to declare that everyone in the game is an enemy EXCEPT one or
more other species, declare everyone an enemy and then use the \texttt{NEUTRAL} or
\texttt{ALLY} commands for the non-enemies.  For example:

\begin{verbatim}
START PRE-DEPARTURE
; Place pre-departure orders here.

        Enem    111          ; Everyone is my enemy...
        Ally    SP Vulcan    ;  ...except the Vulcans (our allies)
        NEUTRAL SP Bajoran   ;  ...and the Bajorans (who I'm not sure of yet)

END\end{verbatim} 

\texttt{ENEMY}, \texttt{ALLY}, and \texttt{NEUTRAL} orders will remain in effect until cancelled.

Unless you specify otherwise, the computer will assume that you are neutral
towards all species in the game.  When you meet a species for the first time,
the computer will assume that you are neutral towards it, unless you have
previously issued an ENEMY order with a numeric argument.

When an \texttt{INTERCEPT} order is executed, the computer will check if any enemy
ships have successfully jumped into the star system of the producing planet
during the current turn.  If so, the total amount spent for interception on
all planets in the star system will be applied to the enemy ships that just
arrived, one at a time, until all ships are destroyed or until funds run out,
whichever comes first.  Ships will be chosen in random order, so as not to
bias the results against any one enemy species.

The amount needed to destroy an enemy ship can be calculated as follows:
\[
   \textrm{Cost} =  \dfrac{100  \times  \textrm{warship tonnage}}{10,000}
\]

The cost needed to intercept and destroy a transport will be one-tenth of the
cost for a warship of the same tonnage.

Note that, unlike an ambush, the ships are not aged - they are either
completely destroyed or completely unharmed.  If there are not enough funds
to destroy the current randomly selected ship, then the interception process
ends, and all remaining funds, if any, are wasted.

For example, let's say that three enemy ships successfully jump into the Earth
solar system.  The first is a 150,000 ton Klingon destroyer, the second is a
50,000 ton Romulan escort, and the third is a 50,000 ton Romulan transport.
In the same turn, the production orders for Earth and Mars (in the same star
system) contain the orders:

\begin{verbatim}
START PRODUCTION
    PRODUCTION PL Earth
    ; Place production orders here for planet Earth.

        Intercept       450

    PRODUCTION PL Mars
    ; Place production orders here for planet Mars.

        Int             100

END\end{verbatim}

Thus, the total amount spent on interception is 550.

At the end of production for your species, the computer randomly selects the
50,000 ton Romulan escort as the first ship for interception.  The cost is:

\[
			\dfrac{100  \times  50,000}{10,000}  =  500
\]

Since the current balance is 550, the escort is destroyed and the new balance
is $550 - 500 = 50$.  If the computer then randomly selects the destroyer for
destruction, the interception process immediately ends because the balance of
50 is not enough to destroy the destroyer (1500 needed), EVEN THOUGH IT IS
ENOUGH TO DESTROY THE TRANSPORT!  Thus, both the destroyer and the transport
will be unharmed.  If the computer had randomly selected the transport before
the destroyer, then the transport would have been destroyed.

This element of chance is intentional, since an interception is inherently less
reliable and more `chancy' than an ambush.

Note that the three enemy ships do not have to be owned by the same species.

All species involved in an intercept will be notified of the results in their
status reports.  The species that owns the ships that were destroyed will only
be informed that the ships disappeared without a trace, cause unknown.  Thus,
the intercepted species will not know if the ship was destroyed by an enemy,
or if it self-destructed on its own.

You may give both \texttt{INTERCEPT} and \texttt{AMBUSH} orders in your production orders. 
Obviously, \texttt{INTERCEPT} orders will apply immediately, while \texttt{AMBUSH} orders will
take effect at the very beginning of the combat phase of the next turn.

Anything spent on interception is wasted if no enemy ships jump into the
system; i.e., the amount spent does NOT carry over into the next turn.  A
planet that is under siege cannot issue \texttt{INTERCEPT} orders.  Interceptions by
more than one species in the same star system are NOT combined --- they are
each handled separately.

If an interception fails completely, the intended victim will not be told that
an interception had been planned, and will remain completely in the dark.

Interception will only be used against ships that enter the sector via a JUMP,
\texttt{PJUMP}, or \texttt{WORMHOLE} command.  Interception cannot be used against ships that
\texttt{MOVE} into a sector.

\begin{informationnote}
A ship that \texttt{MOVE}s into a system is moving at sub-light speeds and will have
telemetry of the space ahead, as well as the time to analyze the telemetry.
An \texttt{INTERCEPT} takes advantage of the ``timelessness'' of the jump and the complete
lack of data about the destination until the moment of emergence from the
wormhole. 
\end{informationnote}

The main purpose of an interception is to prevent exploratory vessels from
learning about the inhabitants of a star system.  It is NOT intended as
a replacement for normal combat.  To reflect this reality, the \texttt{INTERCEPT}
command will be limited as follows:

\begin{quotation}
	The maximum size of a ship that may be successfully intercepted
	is 50,000 tons for warships and 200,000 tons for transports.  If
	the random selection process described above selects a larger
	ship, then the selection process will abruptly end.\end{quotation} 

For example, if three ships are randomly selected in the order <picketboat,
battlecruiser, picketboat>, then only the first picketboat will be destroyed,
even if there were sufficient funds allocated to destroy all three ships.

There is one major exception to the above rules regarding interception.
Interception of a ship is much easier when the ship is exiting a natural
wormhole, since the intercepting species knows the precise exit point and
can concentrate more of its forces at that point.

To simulate this reality, the above tonnage limitations will NOT apply if the
intercepted ship entered the sector via a natural wormhole.  In other words,
any ship can be intercepted and destroyed, REGARDLESS OF SIZE, if it enters the
sector via a natural wormhole and if sufficient funds have been allocated for
the interception.


\subsection{STRATEGIC TARGETS}
\label{sec:strategictargets}


In general, an attacker will try to concentrate fire on the most powerful
targets in order to eliminate them as quickly as possible.  Smaller targets
will generally be ignored until all of the more dangerous opponents are
eliminated.  Thus, if you have a large number of small ships (i.e. `cannon
fodder') and hope that they'll draw the fire of more powerful opponents, you'll
be disappointed.  Attackers will rarely waste shots on puny targets as long as
larger prey are available.

In some battles, however, you may want to concentrate your fire on targets of a
specific type.  For example, if you suspect that enemy transports are carrying
germ warfare bombs, you may want to concentrate your fire on transports as long
as they are within range.  To specify a favorite target for a battle, use the
\texttt{TARGET} command, as follows:

\begin{verbatim}
	Target	1	; Concentrate fire on warships.
	Target	2	; Concentrate fire on transports.
	Target	3	; Concentrate fire on starbases.
	Target	4	; Concentrate fire on planetary defense units.
\end{verbatim} 

Only one \texttt{TARGET} order may be given per battle.  If more than one is given then
only the last one will be effective.

The \texttt{TARGET} command simply indicates that you want to concentrate fire on units
of the selected type.  However, it does not guarantee that you will ONLY fire
on the selected targets.  The tactical situation may either prevent you from
firing on a selected target or may force you to attack a different target.  For
example, if there are no more enemy units of the selected type, then your units
will fire on any available enemy units.


\subsection{THE STRIKE PHASE}
\label{sec:strikephase}


Combat can take place either in the combat phase or in the strike phase.

All combat orders and engagement options are allowed in the combat phase.

All combat orders are allowed in the strike phase.  However, engagement options
for planet bombardment, germ warfare, and sieges are not allowed in the strike
phase.

In effect, the strike phase is a limited-combat phase, And any combat that does
take place in the strike phase generally takes the form of an initial surprise
attack.  Combat that requires more time, such as bombardment and siege,
will take place in the combat phase of the following turn, and is thus
a continuation of the combat that began in the strike phase.

Note that, even though a turn break occurs between the strike phase of one turn
and the combat phase of the following turn, no actual game time passes.


\subsection{PICKET DUTY}
\label{sec:picketduty}


There may be times when you need to intercept enemy ships, but the INTERCEPT
command will not work.  For example, the ships may be too large, or they may be
entering the sector via the \texttt{MOVE} command.  In situations like these, you can
effectively intercept enemy ships in the strike phase - IF you have the
firepower.  Here's how to do it:

\begin{verbatim}
START STRIKES
; Place strike orders here.

        Battle  x y z
        Attack  SP name
        Engage  3               ; Deep space attack.
END\end{verbatim} 

If you win the battle, the intruding ships will be destroyed before they can
learn much about the sector.

Note that the above battle will not prevent the intruder from doing a scan
during the post-arrival phase, but it WILL keep it from learning about who has
colonies in the sector, what their economic bases are, and which non-attacking
species, ships, and starbases are also in the sector.  The intruder will only
know that the attacker probably (but not necessarily!) inhabits the sector.

However, there is still one potential problem.  If the intruder \texttt{MOVE}s to your
sector and then \texttt{ORBIT}s a planet, the above orders would not have the desired
effect.  This is because an \texttt{ENGAGE 3} order only tells your ships to attack
enemy ships that are in deep space.  You can, of course, give \texttt{ENGAGE 4 n}
orders for all of the planets in the system, but this is a nuisance.

Instead, a special rule will apply in this kind of situation: the computer will
reject a post-arrival \texttt{ORBIT} order for a ship that has just \texttt{MOVE}d into a sector.
If necessary, the \texttt{ORBIT} order can be given in the pre-departure phase of the
next turn.


\subsection{HIDING YOUR STRENGTH}
\label{sec:hidingstrength}


There may be times when you wish to engage in combat with an enemy, but you do
not want to show your full strength.  For example, let's say that you have a
large fleet landed on a planet where the enemy cannot detect it.  If you want
to attack a single, small, enemy ship, you can do so using the commands
described earlier.  However, ALL of your ships will take part in the battle,
and the enemy will learn your total strength in that sector.

To deal with this problem, you can use the \texttt{HIDE} command to tell the computer
that specific ships are NOT to take part in combat unless absolutely necessary.
Here's an example:

\begin{verbatim}
START COMBAT
; Place combat orders here.

        Battle  23 17 3
        Attack  SP Klingon
        Attack  SP Romulan
        Engage  3               ; Deep space attack.
        Hide    DN Omigosh      ; These ships should not fight unless we're
        Hide    BS Thunderclap  ;  losing the battle.
        Hide    BCS Whopper

END\end{verbatim} 

In the above example, all of your warships in the sector will take part in the
deep space attack except DN Omigosh, BS Thunderclap, and BCS Whopper.  They
will only join the battle if you are losing.

You may only give a \texttt{HIDE} order to ships that are landed on the surface of a
planet.  \texttt{HIDE} orders given to ships in orbit or in deep space will be ignored.

If you give a \texttt{HIDE} order to a transport, the enemy will only be allowed to
attack it if you start losing the battle.

A ship that is given a \texttt{HIDE} order will not take part in any subsequent
bombardment or siege, unless it also took part in the battle.

Keep in mind that there may be a risk in using the \texttt{HIDE} command.  If you
misjudge the enemy's strength, they may be able to destroy your front guard
before the reserves join the fight.  On the other hand, if all ships had been
in the battle from the beginning, enemy fire could have been spread so thin
that none of them would have been damaged.


\subsection{HIJACKING}
\label{sec:hijacking}


The \texttt{HIJACK} command should be used if you do NOT want to destroy enemy ships,
but instead want to capture them.  The captured ships and cargo can then be
sold for profit, and the crew and passengers can then be ransomed or sold into
slavery.  The \texttt{HIJACK} command should be used instead of the \texttt{ATTACK} command, as
in the following example:

\begin{verbatim}
START COMBAT
; Place combat orders here.

        Battle  10 25 12
        Attack  SP Klingon      ; Attack the Klingons and attempt to destroy
                                ;  their ships.
        Hijack  SP Romulan      ; Attack the Romulans and attempt to capture
                                ;  their ships.
        Engage  3               ; Attack them in deep space...
        Engage  5 2             ;  and at planet 2, and bombard the planet if
                                ;    possible afterwards, ...
        Engage  7 3             ;  and at planet 3, and besiege the planet if
                                ;    possible afterwards.
END
\end{verbatim} 

Hijacking may also be attempted in the strike phase.

When attempting to \texttt{HIJACK} the ships of another species, combat will occur in
the usual way, but the hijacking ships will have only 25\% of their normal
offensive and defensive combat capabilities.  If an enemy ship is then
"destroyed", it will be considered effectively ``hijacked''.  If you \texttt{ATTACK} one
or more species while \texttt{HIJACK}ing one or more other species in the same battle,
then your ships will have the reduced combat capabilities whether they are
\texttt{ATTACK}ing or \texttt{HIJACK}ing.  Because of this, it may not be a good idea to do both
unless you have vastly superior forces.

Hijacked ships and cargo will be automatically sold, and economic units will be
added to the treasury of the hijacking species.  The number of economic units
received will be equal to the normal recycle value of the hijacked ships and
their cargo.

You may NOT hijack planetary defense units.  If successfully attacked, they
will be destroyed.

If additional combat operations take place after the hijacking (such as
bombardment or siege), the attacking ships will have their full, normal
combat capability.

\section{MISCELLANEOUS OPERATIONS}
\label{sec:miscoperations}


Commands and other operations that do not belong in any of the previous
chapters will be discussed here.


\subsection{SCAN ORDERS}
\label{sec:scanorders}


If you wish to receive information about a star system, you must order a ship
to do a sensor scan.  Use the \texttt{SCAN} command to do this.  For example:

\begin{verbatim}
START POST-ARRIVAL
; Place post-arrival orders here.

     SCAN TR1 Intrepid

END\end{verbatim} 

The \texttt{SCAN} command may only be issued in the pre-departure or post-arrival
sections of your orders.

If the ship mis-jumped to the scan location during the jump phase, then a scan
order issued in the post-arrival phase will still be executed, but it won't
detect anything unless it mis-jumped to coordinates that contain a star system,
which is highly unlikely.

A scan only tells you about the physical characteristics of the planets in
the sector.  It does NOT tell you which planets are populated, or if there are
other ships in the sector.  This additional information is always automatically
provided on your status report, whether or not you give a \texttt{SCAN} order.
Remember, though, that you will never be able to detect ships that have landed
on a populated planet unless you also have population on the same planet.

If you have a ship or a populated planet in a star system that has one or more
alien inhabited planets, your status report will also provide an approximate
indication of the amount of commercial and industrial activity on those
planets.  The activity will be indicated as in the following example:

\begin{verbatim}
  Colony planet PL Dizzy (planet #2)                 SP Goofballs
      (Economic base is approximately 210.)\end{verbatim} 

This simply indicates that a planet named \texttt{Dizzy} is colonized by a species
named \texttt{Goofballs}, and that the sum of mining base plus manufacturing base
on the planet is approximately 210.  Again, this information is provided
automatically on your status report --- there is no need to provide a \texttt{SCAN}
order.


\subsection{MESSAGES}
\label{sec:messages}


You can send a message to any species you know of with the \texttt{MESSAGE} command.
Messages may be sent in either the pre-departure or post-arrival phase of
the turn.  Here are some examples:

\begin{verbatim}
START PRE-DEPARTURE
; Place pre-departure orders here.

  MESSAGE	SP Klingons

    My email address is goofy@bubblegum.gov. Please contact me ASAP!

  ZZZ

END

START POST-ARRIVAL
; Place post-arrival orders here.

Mes SP Romulans
Please do not be alarmed at the size of the fleet in your home system. We're
simply using your home system as a rest stop on our way to another location.
Rest assured that we have no ill intentions towards you.
zzz

END
\end{verbatim} 

Messages can be as long as you wish.  Make sure to end the message with a line
containing the command \texttt{ZZZ}.  Also make sure that none of the lines in the
message itself start with the letters `zzz'.


\subsection{TRANSFERRING ECONOMIC UNITS}
\label{sec:transferringeconomicunits}


To transfer economic units to another species, use the \texttt{SEND} command.  For
example, if you wish to transfer 250 economic units to the Klingons, you
should give the following order:

\begin{verbatim}
    Send 250 SP Klingon\end{verbatim} 

You may not \texttt{SEND} economic units to a species that you have declared to be an
\texttt{ENEMY}.  If you wish to \texttt{SEND} economic units to an \texttt{ENEMY}, make a temporary
declaration of neutrality, as in the following example:

\begin{verbatim}
START POST-ARRIVAL
; Place post-arrival orders here.

	Neutral		SP Tholian
	Send	500	SP Tholian
	Enemy		SP Tholian
END\end{verbatim} 

A \texttt{SEND} order must appear in the pre-departure or post-arrival section of your
orders --- NOT during production!


\subsection{KNOWLEDGE TRANSFERS}
\label{sec:knowledgetransfers}


In Far Horizons, a donor species may transfer knowledge to a recipient species
using the \texttt{TEACH} command, as in the following examples:

\begin{verbatim}
         Teach	GV	SP Klingon
         Teach	BI 17	SP Klingon
\end{verbatim} 

where the maximum tech level to be taught is optional.  If not specified, then
the actual tech level of the teacher will be used.

\texttt{TEACH} orders may only appear in the post-arrival section of your orders.

You may not \texttt{TEACH} a species you haven't yet met.

When a \texttt{TEACH} order is executed, the computer simply notes that you have
obtained knowledge associated with a tech level.  If you then \texttt{RESEARCH} the
technology in a later turn, there will be no randomness in achieving the new
tech level, since you already have the needed knowledge.  Instead, the funds
will be used to perform the necessary engineering and to spread the new
technology throughout the species.  In effect, the research funds will be
used directly to `purchase' the new tech level.

Since the basic research is eliminated, the randomness is also eliminated
and the cost is reduced.  However, the recipient must still do the initial
engineering and development work.  Thus, the cost of a tech increase will be
less than the cost of solitary research.

On average, if you wish to increase a tech level by one point using normal,
unaided research, then the average cost will be the current tech level times
itself.  For example, if the current Biology tech level is 9, then the basic
cost to increase the tech level from 9 to 10 is $9 \times 9 = 81$.  Keep in mind,
though, that this is an AVERAGE value.  Basic research is highly random.

Now, if another species first transfers the knowledge to you via the \texttt{TEACH}
command, then there will be a 25\% discount on the basic cost of research, and
there will be no randomness at all.

For example, SP Human gives the following post-arrival order:

\begin{verbatim}
     Teach  GV 20  SP Klingon\end{verbatim} 

In the next turn, the Klingons have a gravitics tech level of 18 and give the
following production order:

\begin{verbatim}
    Research 750 GV\end{verbatim} 

When this order is processed, the Klingon gravitics tech level will immediately
rise as follows:
\[
	\textrm{cost for } 18 \textrm{ to } 19 = 18 \times 18 - 25\% \textrm{ discount } = 324 - 81 = 243 
\]
\[
	\textrm{cost for } 19 \textrm{ to } 20 = 19 \times 19 - 25\% \textrm{ discount } = 361 - 90 = 271
\]
\[
	\textrm{total cost using transferred knowledge} = 243 + 271 = 514
\]
The remaining funds $(750 - 514 = 236)$ will be allocated to additional, normal
research in gravitics.

If knowledge has been transferred to you but has not yet been applied, the
usable knowledge level will appear in your status report as in the following
example: \\

\noindent Tech Levels:
\begin{flalign*}
   \text{Mining} &= 17/21 &\\
   \text{Manufacturing} &= 17 &\\
   \text{Military} &= 9 &\\
   \text{Gravitics} &= 8 &\\
   \text{Life Support} &= 13/14  &\\
   \text{Biology} &= 4 &\\
\end{flalign*} 

The above indicates that your mining tech level is 17, but that you have
knowledge up to 21.  Similarly for life support.

There is no limit to how much technical knowledge may be transferred in a
single turn.  As long as the donor's tech level is high enough, the transfer
will take place.

You may NOT transfer knowledge to a species that you have declared to be
an \texttt{ENEMY}.  If you wish to transfer knowledge to an \texttt{ENEMY}, make a temporary
declaration of neutrality, as in the following example:

\begin{verbatim}
START POST-ARRIVAL
; Place post-arrival orders here.

        Neutral         SP Klingon
        Teach   GV 20   SP Klingon
        Enemy           SP Klingon
END\end{verbatim} 


TEACH orders will not be logged in the donor's status report.  They will be
logged in the recipient's report ONLY if they succeed in raising the knowledge
level of the recipient.


\begin{importantnote}
	It is strictly forbidden to attempt to transfer knowledge to
	a species that does not want it or which is not expecting it.
	This is necessary because some species may not want certain
	types of knowledge for role-playing reasons.  Any player that
	intentionally breaks this rule will be evicted from the game.
\end{importantnote}

\subsection{ESTIMATING THE TECH LEVELS OF OTHER SPECIES}
\label{sec:estimatetechlevels}


At any time after meeting another species, you many analyze all of the
information you have about the species and estimate their tech levels.  To do
this, issue the \texttt{ESTIMATE} command in the production section of your orders for
a planet.  Here is an example:

\begin{verbatim}
START PRODUCTION
    PRODUCTION PL Dagwood
    ; Place production orders here for planet Dagwood.

     Estimate SP Klingon

END\end{verbatim} 

Each estimate has a cost of 25.  You will receive an estimate of all six tech
levels for the species.

The accuracy of your estimate will depend on how high your tech level is
compared to the same technology for the other species.  If your tech level is
significantly higher than the level for the other species, then the estimate
will be very accurate.  At the opposite extreme, if your tech level is much
lower than the level of the other species, then the estimate may not be very
accurate.


\subsection{DISBANDING A COLONY}
\label{sec:disbandcolony}


There may be situations when it is desirable to disband a colony.  For example,
you may be ordered to leave by a more powerful enemy who will destroy the
colony if you don't obey.  In this case, it would be better to remove the
colony peacefully, and perhaps salvage some of your investment, than to fight
a battle that you know you cannot win.

If you decide to disband a colony, for whatever reason, you should use the
\texttt{DISBAND} command in the pre-departure section of your orders.  For example,

\begin{verbatim}
START PRE-DEPARTURE
; Place pre-departure orders here.

        Disband         PL Vega III

END\end{verbatim} 


\noindent You may NOT disband your home planet.

\noindent You may NOT disband a colony that is under siege.

When processing a \texttt{DISBAND} order, the computer will do the following:
\begin{enumerate}
	\item Any mining base and manufacturing base will be converted
	to Colonist Units, Colonial Mining Units, and Colonial
	Manufacturing Units; i.e. exactly the reverse of the \texttt{INSTALL}
	command.  However, \texttt{IU}s and \texttt{AU}s will only be converted at 50\%
	efficiency.  For example, a mining base of 21.3 will produce
	213 \texttt{CU}s, but only 106 \texttt{IU}s.

	\item The computer will mark the colony as ``disbanded''.

	\item At the end of the turn (after the post-arrival phase) the
	computer will convert anything of value that is left on the
	"disbanded" planet to economic units, and will automatically
	transfer the economic units to the balance for the species.  Any
	ships that are landed on the planet and any starbases orbiting
	the planet plus any cargo that they carry will be salvaged, even
	ships that were under construction.  The amount of economic units
	obtained by salvaging will be ONE-HALF of their total \texttt{RECYCLE}
	value (see section 5.10 for information about recycling).
\end{enumerate}
When a colony is marked as ``disbanded'', any ships on the surface, including
ships that are still under construction, and any starbases in orbit are marked
as ``salvage''.  If you \texttt{LAND} a ship on the planet after the DISBAND order, it
will also be marked as ``salvage''.  If you \texttt{ORBIT} a starbase around the colony
after the \texttt{DISBAND} order, it will also be marked as ``salvage''.  You may not give
any \texttt{MOVE}, \texttt{ORBIT}, \texttt{JUMP}, \texttt{PJUMP}, or \texttt{WORMHOLE} orders to a ship or starbase that has
been marked as "salvage".  Thus, it is important to issue any \texttt{LAND} or \texttt{ORBIT}
orders \texttt{BEFORE} giving the \texttt{DISBAND} order.

In the pre-departure section of your orders, immediately after the \texttt{DISBAND}
order, you can fill up any other ships and transports with the \texttt{CU}s, \texttt{IU}s, \texttt{AU}s,
and anything else that is on the planet, and give these ships orders to leave
in the jump section of your orders.  In other words, the ``disbanded'' colony and
``salvage'' ships and starbases may still take part in \texttt{TRANSFER} commands after
the \texttt{DISBAND} order has been given.  In this way, you can \texttt{TRANSFER} any items of
value that are on the planet or on the ``salvage'' ships to other ships or even
to other planets in the star system.  If you have any jump portals in the star
system, you may use them as well, including any jump portal starbases that have
been marked as ``salvage'' (see section on high tech items for more information
about jump portals).

After the post-arrival phase, any items and ships that are on a ``disbanded''
planet and starbases that are in orbit around the planet will be salvaged,
even if they were not there when the \texttt{DISBAND} order was executed.  If these
ships or starbases have cargo, then the cargo will also be salvaged.  The
actual details of the salvaging operation are not important.  (Possible
scenarios: 1. The people on the planet commandeer all commercial vessals in the
system, load them up, and escape.  2. The people on the planet negotiate with
the enemy for a little more time to disband.  3. If there is time, the people
and the government can hire commercial spacelines, or even professional
salvaging companies to disband the colony.  And so on.)

The DISBAND command will completely delete all information about the planet,
even its name.  Thus, you may also use this command to delete a planet that
you have named but have not yet colonized.


\subsection{DESTROYING SHIPS AND STARBASES}
\label{sec:destroyingships}


If you want to destroy a ship or starbase (for whatever reason), use the
\texttt{DESTROY} command, as in the following examples:

\begin{verbatim}
START PRE-DEPARTURE
; Place pre-departure orders here.

        Destroy         BAS One Shot
        DES             CTS Billy Joe Bob
        dest            TR4 Lollipop
END\end{verbatim} 


When the \texttt{DESTROY} order is executed, the ship or starbase and all of its
contents are blown to smithereens.  Nothing is salvaged.

The \texttt{DESTROY} command may be used in either the pre-departure or the post-arrival
phase of the turn.

Unlike the \texttt{RECYCLE} command, the \texttt{DESTROY} command may be used anywhere - the ship
or starbase does not have to be at a planet or in a star system.

The fate of the crew of the destroyed ship or starbase depends on the species.
Some will commit suicide, some will choose to die with the ship, others may
escape in lifeboats, and so on.  It's entirely up to the player.


\subsection{THE ``DEVELOP'' COMMAND}
\label{sec:developcommand}


It's easy to make mistakes when calculating the number of \texttt{IU}s and \texttt{AU}s needed
for a colony.  If the number of units is not properly balanced (based on the
mining difficulty), then you'll either have excessive production capacity or
excessive raw material units.

The \texttt{DEVELOP} command was implemented to do these calculations for you, and to
save you some time in other ways as well.

The \texttt{DEVELOP} command has three modes of operation.  In the first mode, the
command has two arguments: the name of the colony and the name of the transport
that will carry the \texttt{CU}s, \texttt{IU}s, and \texttt{AU}s.  Here is an example:

\begin{verbatim}
START PRODUCTION
    PRODUCTION PL Earth
    ; Place production orders here for planet Earth.

        Develop PL Orion 7, TR20 Tubby

END\end{verbatim} 


For the above order, the computer will first determine the current cargo
capacity of the transport.  If the colony is a normal colony (as opposed to a
resort or mining colony) and has the resources to build its own \texttt{IU}s and \texttt{AU}s,
then the computer will build only \texttt{CU}s to fill the transport; otherwise it will
build the correct balance of \texttt{CU}s, \texttt{IU}s, and \texttt{AU}s based on the mining difficulty
of the planet.  If the colony is a resort colony, then only \texttt{CU}s and \texttt{AU}s will be
built.  If the colony is a mining colony, then only \texttt{CU}s and \texttt{IU}s will be built.
If only \texttt{CU}s are being sent to the colony, then the computer will also generate
orders for the colony to build the correct amount of \texttt{IU}s and \texttt{AU}s based on the
mining difficulty.  (Why waste cargo space for \texttt{IU}s and \texttt{AU}s?  Let the colony
build them!)  These orders will be placed in the orders section at the end
of the status report.

\begin{informationnote}
Incidentally, when the \texttt{DEVELOP} command is processed, the computer can not
predict exactly what the production will be on the colony during the next turn.
For example, the computer does not know what the fleet maintenance cost will
be, and cannot predict if the player will be doing other production on the
colony.  Because of this, it makes a very conservative estimate of the next
turn's production.  The result may be that the transport carries \texttt{IU}s and \texttt{AU}s
that the colony is capable of building itself.  However, this will not be a
serious problem because it will only occur once or twice, when the colony is
still relatively small.
\end{informationnote}

If the colony is not yet colonized, or if it does not have sufficient
production capacity to build its own \texttt{IU}s and \texttt{AU}s, then the computer will fill
the remaining cargo space of the transport with a correct balance of \texttt{CU}s, \texttt{IU}s,
and \texttt{AU}s based on the planet's mining difficulty.  Note that this will only work
for normal colonies.  If you want to start a mining or resort colony, do not
use the \texttt{DEVELOP} command.  However, the \texttt{DEVELOP} command may be used AFTER the
mining or resort colony has been started.

At the end of production, any \texttt{CU}s, \texttt{IU}s, and \texttt{AU}s will be loaded onto the
transport.  Also, an order will be automatically generated for the transport to
jump to the colony. This order will be placed in the order section at the end
of your status report.  This assumes, of course, that the ship is in the same
sector as the producing planet.  If not, the items will remain on the surface
of the planet and no \texttt{JUMP} order will be given for the ship.

If you want to load other cargo onto the transport, make sure to do so BEFORE
issuing a \texttt{DEVELOP} order.  A \texttt{DEVELOP} order will use all remaining unused cargo
space.

Note that the \texttt{DEVELOP} order must be given in the production phase of your
orders.

In the second mode of operation, no transport is mentioned.  Instead, it is
assumed that the colony is in the same sector as the producing planet.  Here
is an example:

\begin{verbatim}
START PRODUCTION
    PRODUCTION PL Earth
    ; Place production orders here for planet Earth.

        Dev     PL Venus

END\end{verbatim} 


The above order is similar to the previous one, except that the remaining
available population of the PRODUCING planet will be used, and the items will
be automatically transferred to the colony instead of to a transport.  In
addition, an order will be automatically generated for the next turn to
install the units on the colony.

Since this mode uses the available population of the PRODUCING planet, this
order should be placed after any other orders are given to build \texttt{CU}s or \texttt{PD}s.

In the third mode of operation, the command should be given on the colony
itself and has no arguments.  Here is an example:

\begin{verbatim}
START PRODUCTION
    PRODUCTION PL Venus
    ; Place production orders here for planet Venus.

        DEV

END\end{verbatim} 


For the above order, the computer will build $X$ \texttt{CU}s, where $X$ is equal to the
remaining available population, and a correct balance of \texttt{IU}s and \texttt{AU}s based on
the mining difficulty of the planet.  It will also automatically generate
\texttt{INSTALL} orders for the next turn to install the units.  Note that this command
cannot be given on a mining or resort colony, since it has no production
capability.

If you do not have sufficient funds to utilize all of the available population
or transport cargo capacity, then the computer will use the available funds as
the limiting factor.  Note that the computer WILL spend economic units from the
treasury, if needed.

In all of the above modes, you may optionally specify a spending limit.  Here's
an example:

\begin{verbatim}
START PRODUCTION
    PRODUCTION PL Earth
    ; Place production orders here for planet Earth.

        Dev     450     PL Venus

END\end{verbatim} 


In the above example, no more than 450 will be spent to develop the planet
Venus even if you have more available funds or available population.

Normally, you may not \texttt{DEVELOP} a home planet.  However, if the home planet has
been bombed, then you may use the \texttt{DEVELOP} command to bring it back to its
original value.


\subsection{THE ``AUTO'' COMMAND}
\label{sec:autocommand}


Far Horizons is a complicated simulation, and providing orders for all of your
ships and planets can be both tedious and time-consuming, especially in later
stages of the game when empires become very large.  The \texttt{AUTO} command is
intended to reduce the tedium and make it less likely that mistakes or
omissions will occur.

The \texttt{AUTO} command should be given in the post-arrival phase of your orders. It
has no arguments:

\begin{verbatim}
START POST-ARRIVAL
; Place post-arrival orders here.

        Auto

END\end{verbatim} 


The AUTO command instructs the computer to automatically generate reasonable
orders for your ships and planets FOR THE NEXT TURN and to place these orders
in the appropriate places in the orders section of the status report.  The
AUTO command will generate orders to accomplish the following:
\begin{enumerate}
 	\item If a transport has colonist units in it, is not in the home
	sector, and is orbiting or landed on a planet whose economic
	base is less than 200, then an \texttt{UNLOAD} order will be generated
	for the transport, and it will be given an order to \texttt{JUMP} to the
	planet where it was most recently loaded with \texttt{CU}s (typically the
	homeworld).  Also, a \texttt{DEVELOP} command will be generated for the
	destination planet to prepare colonists for the same colony
	using the same transport.

	\item Since \texttt{TR1}s are normally used for scouting, \texttt{JUMP} orders will
	be started for all \texttt{TR1}s.  The destination will be the closest star
	that has not yet been visited by your species.  If you've already
	visited all stars in the galaxy, then the destination will be ``???''.
	The order will be commented with the age and current location of
	the ship, plus the mishap chance for the jump.

	\item All other ships will be issued \texttt{JUMP} orders with the destination
	``???''.  The order will be commented with the age and current
	location of the ship.

	\item If a planet will have excess raw material units in the next
	turn, an order will be generated to recycle them.

	\item If a colony has an economic base less than 200 and if it has
	an available population greater than zero, then a \texttt{DEVELOP} order
	will be generated for it.

	\item For the home planet and any colonies that have an economic
	base of 200 or more, a check will be made to see if any other
	colonies in the same sector have an economic base less than 200.
	If so, then orders will be generated to \texttt{DEVELOP} those colonies.

	\item Starbases will be given orders to increase their sizes up
	to the MA tech limit.  A comment will indicate the current size.

	\item Ships that are still under construction will be given orders
	to complete their construction.  A comment will indicate the amount
	left to pay.

	\item If a \texttt{TR1} is in a sector that you do not inhabit, the computer
	will assume that it is a scouting vessel and will generate a
	\texttt{SCAN} order for it in the post-arrival phase.

	\item An \texttt{AUTO} command will be generated for the next turn.
\end{enumerate}
Obviously, if you do not agree with a particular order, you can simply delete
it or change it to something more desirable.   For example, if an order is
automatically generated to develop your Rigel colony, you can manually change
it to develop your Arcturus colony instead.


\subsection{THE ``VISITED'' COMMAND}
\label{sec:visitedcommand}


The \texttt{AUTO} command generates orders for \texttt{TR1}s to \texttt{JUMP} to the nearest sector that
your species has not yet visited.  However, the computer has no way of knowing
if another player gives you a scan of an unvisited sector.  The VISITED command
tells the computer that you already have a scan of a sector and that you do not
want it to generate \texttt{JUMP} orders to go there for your TR1s.  Here are some
examples:

\begin{verbatim}
START JUMPS
; Place jump orders here.

        Vis     31 12 18        ; We just received scans of these sectors
        VISIT   33 9 10         ;  from our allies.

END\end{verbatim} 


A \texttt{VISITED} order must appear in the \texttt{JUMP} section of your orders, and the x, y,
and z coordinates must apply to a sector that has a star.  If the sector does
not have a star, then the order will be ignored.

Also, whenever you \texttt{NAME} a planet, the sector will me marked as visited by your
species even if you have not actually been there.



\section{MISCELLANY}
\label{sec:miscellany}


This chapter will cover those aspects of the game that do not quite belong
in any of the other chapters.  Since it is also the last chapter, it is an
ideal place to make additions, corrections, threats, etc. without throwing
everything else out of kilter.  Thus, this chapter is probably the most
important one in the rules.


\subsection{HIGH-TECH ITEMS}
\label{sec:hightechitems}


Appendix A of this manual contains a list of high-tech items that your species
will be able to build when it achieves certain minimum tech levels.  You will
be notified on your status report whenever this occurs.  The descriptions of
the high-tech items are self-explanatory.

All high-tech items must be built in a single turn, unless specified otherwise.


\subsection{COMMUNICATION BETWEEN PLAYERS}
\label{sec:communicationbetweenplayers}


Players who wish to keep their real identities anonymous may do so.  You may
send a message to another species via the \texttt{MESSAGE} command.  The recipient
will only know the name of the species who sent the message.  If you wish to
communicate directly with other players, you can send them your email address
with the \texttt{MESSAGE} command.

In addition, some gamemasters may provide a list server at their email site
that will allow players to send messages to other players using species names
rather than player addresses.  It will also allow you to broadcast messages to
all other players, anonymously or non-anonymously.


\subsection{INTERSPECIES TRANSFERS}
\label{sec:interspeciestransfers}


It is not practical or realistic to allow items or ships to be transferred from
one species to another.  Items built for one species are not likely to be
suitable for use by a different species.  The easiest and least expensive way
to give something to another species is to give them economic units, discussed
earlier.  You may also build items and ships SPECIFICALLY for another species.
This will be discussed later in the section on `interspecies construction'.


\subsection{QUESTIONS}
\label{sec:questions}


Questions for the gamemaster should NOT be sent with your orders for the turn.
You should send any questions to the gamemaster in a separate email message.
It is quite possible that the gamemaster will never even see your orders, but
will simply feed them to the computer.


\subsection{USE PLANET NAMES, NOT NUMBERS!}
\label{sec:planetnamesnotnumbers}


The precise designation for a planet is \texttt{X Y Z N}, where \texttt{N} is the number of
the planet in the star system.  This complete designation is only allowed in
the \texttt{NAME} command, discussed earlier.  Planet numbers are also needed in some
combat commands, and may be used in the \texttt{LAND} and \texttt{ORBIT} commands.  In all other
commands, planet names MUST be used if you need to refer to a specific planet,
such as \texttt{PL Earth}.  You may NEVER use planet names assigned by other species.
When processing YOUR orders, the computer is only aware of names that YOU have
assigned.  (There is one exception to this - the \texttt{PJUMP} command - which will be
discussed later in the section on Jump Portals.)

If a star system has one or more planets with names, you may still refer to the
star system by using the X, Y, and Z coordinates, but this is not recommended.
If you mis-type a planet name, the gamemaster will be notified by the computer
and can correct the mis-spelling.  If you mis-type a coordinate, however, the
gamemaster is not likely to know your intentions, even if an error is reported
by the computer.  Furthermore, if it is a \texttt{JUMP} order, your ship may mis-jump or
even self-destruct.  However, you should avoid giving names to planets that you
have not colonized, unless you travel there frequently.

When jumping to meet another species, there is never any need to specify a
particular planet.  As long as members of two species are in the same sector,
they may interact.  There are no combat advantages or disadvantages to being
on or near a particular planet in an encounter.

Finally, it is possible to \texttt{NAME} planets in a sector that you have not yet
visited.  If you name a planet that does not exist, then the computer will
simply reject the order.  Unfortunately, this allows cheaters to attempt to
name up to nine planets in a sector that they know nothing about and, thus, to
find out how many planets are in the sector.  To protect against this kind of
cheating, the computer will notify the gamemaster of any such attempts.  Also,
if your gamemaster is running the Far Horizons list server, then this will also
occur when using the FHTest feature of the list server.


\subsection{SPECIFYING ZERO QUANTITY}
\label{sec:specifyingzero}


Whenever it makes sense, you can specify ``0'' quantity in an order to indicate
that ALL available quantity should be used.  Here are some examples:

\begin{verbatim}
        Build   0 PD    ; Spend all remaining funds on planetary defenses.

        CONT    BAS Deep Space 9, 0     ; Increase size of starbase as much as
                                        ;  remaining funds allow, but do not
                                        ;  exceed tonnage limit.

        Cont    DD Dagger, 0    ; Spend all remaining funds (up to what is
                                ;  actually needed) on DD.

        Recy    0 RM    ; Recycle all excess RMs.

        Install 0 IU, PL Mars   ; Install all \texttt{IU}s that are on Mars.

        SEN 0 SP Klingon        ; Send all economic units we have to the
                                ;       Klingons.

        UPGRADE BAS Wobbly, 0   ; Spend all remaining production (or as much
                                ;       as is needed) upgrading the starbase.\end{verbatim} 



A zero argument used in a production command will NOT automatically spend
economic units owned by the species.  It will only spend what is available from
production capacity and raw material units.  If you wish to spend economic
units, you must provide a non-zero argument.  For the \texttt{DEVELOP} command, a zero
argument will work as expected.  Thus, if you specify ``0'', it will spend the
remaining balance for the planet, and it will not spend any economic units
(e.g. \texttt{DEV 0 PL Mars}).  However, if you do NOT specify a spending limit (e.g.
\texttt{DEV PL Mars}), then the entire balance, if needed, including economic units,
will be used.


\subsection{THE EFFECT OF COMBAT ON THE RELIABILITY OF STATUS REPORTS}
\label{sec:effectofcombat}


A status report provides information about the status of your species IF COMBAT
DOES NOT OCCUR AT THE VERY BEGINNING OF THE NEXT TURN.

If combat DOES occur during the combat phase, then the information on the
status report may be unreliable.  Here are a few examples:
\begin{enumerate}[a.]
 \item  If combat destroys some or all of the mining and manufacturing
	bases on a planet, then production for the planet will be less
	than indicated on the status report, and some production orders
	may be rejected by the computer.

	\item If ships are destroyed during combat, then commands that involve
	these ships (e.g. \texttt{JUMP}, \texttt{ORBIT}, \texttt{TRANSFER}, etc.) will be rejected by
	the computer.

	\item Ships that jumped into the star system in the previous turn and
	\texttt{LAND}ed or \texttt{ORBIT}ed a planet will not actually be able to \texttt{LAND} or \texttt{ORBIT}
	if they are destroyed in combat.
\end{enumerate}
In other words, the information in your status report assumes that combat at
the very beginning of the next turn will not change the situation.  It is
important to keep this in mind when reading and interpreting your status
reports.  For example, if your status report shows that a large enemy fleet
has just entered orbit around your home planet, IT MAY NOT BE TRUE.  If you
issue combat orders to fight the enemy in deep space, you may be able to
prevent them from reaching your home world.

Thus, in effect, the status report shows you what the situation will be if
combat does not occur at the very beginning of the next turn.  If combat DOES
occur, then the status report may be wrong.


\subsection{SHIPYARD CAPACITY}
\label{sec:shipyardcapacity}


Each producing planet will have a shipyard capacity which will limit how many
ships can be built in a single turn.  The shipyard capacity of the home planet
will be one when the game begins.  The shipyard capacity of a normal colony
will start at zero.

The number of production orders that you issue to \texttt{BUILD}, \texttt{CONTINUE}, \texttt{IBUILD}, or
\texttt{ICONTINUE} ships or starbases may not exceed the shipyard capacity.  (The \texttt{IBUILD}
and \texttt{ICONTINUE} commands will be discussed later.)  For example, if you wish
to issue three \texttt{BUILD} commands for new warships, a \texttt{CONTINUE} command for a
transport, and a \texttt{CONTINUE} command for a starbase, then the planet must have a
shipyard capacity of at least five.  Orders that exceed the shipyard capacity
will be ignored.  Orders to build other items such as colonist units, planetary
defense units, and so on, do NOT require shipyard capacity.  Only construction
of warships, transports, and starbases requires shipyard capacity.

If you wish to increase the shipyard capacity of a planet, you must issue a
\texttt{SHIPYARD} command during the production phase.  This command will increase the
number of shipyards on the planet by one.  You may only issue one \texttt{SHIPYARD}
order per planet per turn.  The cost to build a shipyard is ten times the
current manufacturing tech level.  You may not build shipyards on mining or
resort colonies.


\subsection{LIMITATIONS ON COLONY SIZE}
\label{sec:limitsoncolonysize}


In Far Horizons, colonies grow mainly by installing mining and manufacturing
units, and there is no limit to how many units may be installed.  In fact, it's
even possible for a colony to eventually produce more spendable income than the
home planet.  While this is not necessarily unrealistic, there must be a limit
to how large and productive a colony can become.  Otherwise, runaway growth
would make the game unplayable.

We will deal with this potential problem by defining a new concept called
"economic efficiency" which will apply to colonies, but not to the home planet.
The economic efficiency of a colony will be 100\% if its economic base (i.e.,
the sum of its mining and manufacturing bases) is not more than 200.0.  If the
sum is greater than 200.0, then the economic efficiency will be a percentage
smaller than 100\%.

When the computer calculates the production for a colony it will multiply the
result by the economic efficiency to determine the actual amount that will be
available for spending.  For example, if the original amount is 2107, and the
economic efficiency is 82\%, then the actual amount available to spend will
be $82\%$ of $2107 = 1727$.  (Note that fractions are dropped.  This calculation
will be done after the calculation of the production penalty but before the
calculation of the fleet maintenance cost.)

Here is how the economic efficiency (\texttt{EE}) will be calculated:

\[
     \textrm{EE}  =  100  \times  \dfrac{200.0 + (\textrm{total economic base} - 200.0) / 20}{\textrm{total economic base}}\textrm{percent}
  \]                              

In effect, any installation of \texttt{IU}s and \texttt{AU}s beyond the 200.0 value will have
only 5\% of its normal effect.

\begin{informationnote}
Keep in mind that the above applies only to colonies, since it is not
possible to install \texttt{IU}s and \texttt{AU}s on a home planet, and since it is not
possible to colonize the home planet of another species.
\end{informationnote}

Note that players NEVER have to do any of the above calculations! All of the
calculations will be done by the computer and the results will be printed on
your status reports.  It is very important, though, that you keep the 200.0
limit in mind, since installation of mining and manufacturing units after the
200.0 limit has been reached will not be very effective.

In Far Horizons, it's also possible for more than one species to colonize the
same planet.  Thus, when doing the above calculation, the computer will use
the total economic base for ALL species on the planet.  In other words, the
economic efficiency will apply to an entire planet, NOT just to a single colony
on the planet, and will be the same for all colonies on the planet.

Now, even with the above rules, you will discover that many colonies generate
more income than the homeworld, even though the homeworld has a much greater
population.

There's nothing unrealistic about this if you keep in mind that homeworld
production is the result of taxation.  A colony (at least in its earlier
stages) is completely owned by the government.  In other words, the government
owns all the resources, employs all of the population, keeps all the profits,
and does not pay taxes.  It can even tax the colonists (what one hand gives,
the other takes away :-).  Also, the total production on the home planet that
is available for the player to use is only a small fraction of the total taxes
actually collected by the government.  Most taxes will be spent on other
things, such as schools, police, health care, social security, boondoggles,
and so on, and will not be available to the player.

In time, the same thing will happen to colonies, although not to the same
extent as on the home planet, and definitely not within the course of a game.
As the colony becomes more and more self-sufficient, a smaller and smaller
fraction of its TOTAL production is owned directly by the government while a
larger and larger fraction is due to taxes on the private sector.  The 200.0
base limit is simply a game device that allows us to simulate the transition
from a government-owned corporation to a more normal economy.

So, keep the 200.0 limit in mind.  You may continue to install \texttt{IU}s and \texttt{AU}s
beyond the limit, but it will take much longer to pay off the investment, and,
in general, you will be much better off if you install the units elsewhere.




\appendix
\section{HIGH-TECH ITEMS AND CAPABILITIES}

A high-tech item is different from other items already discussed in that it has
a ``critical'' technology that is needed in its construction.  For example, if a
species wants to build Starbase Units, it must have a Manufacturing tech level
of at least 20.  Whenever your species achieves a tech level needed to build
one of these special items, you will be notified on the status report.  Unless
stated otherwise, all items can be built using the \texttt{BUILD} or \texttt{IBUILD} command
(the \texttt{IBUILD} command is described later).

\subsection{STARBASE UNITS}

A species with a Manufacturing tech level of at least 20 may build Starbase
Units.  These units are modular units that can be transported to remote
locations where there is no production capability, and used to build starbases.
For example, using starbase units, you can build a starbase at a resort or
mining colony, or even at a location where there is no star system.

Starbase units have a cost of 110 each, and require a cargo capacity of 20
each.  Use the abbreviation \texttt{SU} for starbase units.  When installed, each
unit will contribute 10,000 tons to the total mass of the starbase.

Starbase units are installed using the \texttt{BASE} command, which has the same format
as the TRANSFER command.  The first argument of the command is the number of
starbase units that are to be installed.  The second argument is the name of
the ship or planet that will provide the starbase units.  The third argument
is the name of the starbase that is to be built or increased in size.

The first argument is optional.  If it is missing or zero, then all available
starbase units will be used.  The \texttt{BASE} command may only be used in the pre-
departure section of your orders.  Here are some examples:
\begin{quotation}
\noindent The Human transport TR16 Barrel of Monkeys jumps to the location x = 7,
y = 12, z = 14.  There is no star system at this location.  The transport is
carrying 7 starbase units.  The order:

\begin{verbatim}
START PRE-DEPARTURE
; Place pre-departure orders here.

        Base    TR16 Barrel of Monkeys, BAS Deep Space 3

END\end{verbatim} 
\end{quotation}

will unload all seven starbase units and construct the 70,000 ton starbase
named BAS Deep Space 3.

In the next turn, the Human transport TR20 Tub of Lard jumps to coordinates
7 12 14 carrying 10 more starbase units.  The order:

\begin{verbatim}
START PRE-DEPARTURE
; Place pre-departure orders here.

     BAS	4	TR20 Tub of Lard, BAS Deep Space 3

END\end{verbatim} 

will install 4 of the units, increasing the size of the starbase from 70,000 to
110,000 tons.  The six unused starbase units will remain onboard the transport.

A starbase built using the \texttt{BASE} command will always be built at the same
location as the ship or planet that provides the units.  For example, if a
transport carrying the starbase units is in orbit or landed on a planet, then
the starbase will be built in orbit around the same planet.  If the transport
is at an X Y Z location that does not contain a star system, then the starbase
will be in 'deep space' at that location.  However, if an X Y Z location
contains a star system, then the starbase MUST be built in orbit around one of
the planets in the star system.  It is not practical (for game purposes) to
allow a starbase to be located in the 'deep space' section of a star system.

If an existing starbase is being increased in size, it will remain in its
original location.  Obviously, the ship or planet providing the units must be
at the same X Y Z coordinates as the starbase, although they do not have to be
at the same planet, if any.

Since starbases built using starbase units are not likely to be near a planet
where they can be upgraded, a normal upgrade is not possible.  However, an
effective upgrade can be accomplished by using Damage Repair Units, which are
discussed in the next section.

\newpage
\subsection{DAMAGE REPAIR UNITS}

A species with a Manufacturing tech level of at least 30 may build Damage
Repair Units.  These units are used to repair or upgrade ships in the field,
where it is not possible to do a normal upgrade using the \texttt{UPGRADE} command.

Damage repair units are not simply spare parts that can be used to repair a
ship.  Instead, they are a combination of special `matrix' materials that
undergo mass/energy/mass conversion to produce the needed replacement parts
(similar to the 'replicator' technology of Star Trek).

Damage repair units cost 50 each and require a cargo capacity of 1.  Use the
abbreviation \texttt{DR} for damage repair units.

Damage repair units effectively repair or upgrade a ship by reducing the age
of the ship.  The age reduction is determined as follows:


\[
	\textrm{Age reduction}    =    \dfrac{160,000  \times  \textrm{N}}{\textrm{ship tonnage}}
\]


where N is the number of damage repair units that are used to do the repair
or upgrade.  For example, if 7 damage repair units are used on a 60,000 ton
transport, then the age reduction will be $(160,000 \times 7)/60,000 = 18.67 = 18$.
Note that fractions are dropped.

The age reduction will never reduce the age to less than zero.  The damage
repair units must be carried as cargo by the ship that is being repaired or
upgraded, and will be consumed by the operation.

The repair/upgrade may be done in either the pre-departure or post-arrival
phase of the turn using the \texttt{REPAIR} command.  Here are some examples:

\begin{verbatim}
START PRE-DEPARTURE
; Place pre-departure orders here.

        Repair  BC Big Bend, 25 ; Use 25 damage repair units to reduce
                                ;  the age of the battlecruiser by 10.

        REP     BAS Strong Arm, 17      ; Use 17 damage repair units
                                        ;  to reduce the age of the
                                        ;  90,000 ton starbase by 30.

        repa    FF Gorby Too, 0 ; Use all 5 units onboard to reduce the
                                ;  age of the frigate by 8.

        rep     DD Dagger       ; Use as many units as are needed to reduce
                                ;  the age to zero.

END\end{verbatim} 


If N is zero, then all units that are onboard the ship will be used (but not
more than are necessary to reduce the age to zero).  If N is not specified,
then as many units as are needed to reduce the age to zero will be used and
MUST be onboard.

It is also possible to have all ships at a particular location (including
starbases and transports) pool their damage repair units, using them on the
ships that need them the most.  To do this, the \texttt{REPAIR} command should be given
only the X Y Z coordinates of the location where the repairs are to be done,
as in the following example:


\begin{verbatim}
START POST-ARRIVAL
; Place post-arrival orders here.

        Repair  23 12 9         ; Have all ships in the sector pool their DRs
                                ;  and do as much repair work as possible.
END\end{verbatim} 



In the above example, all of the ships in the sector will pool their damage
repair units and repair only those ships that need repairing, starting with the
most heavily damaged, and continuing until all ships are completely repaired or
until you run out of \texttt{DR}s.

You may also specify a ``desired age'' as in the following example:

\begin{verbatim}
START POST-ARRIVAL
; Place post-arrival orders here.

        Repair  23 12 9 7       ; Have all ships in the sector pool their DRs
                                ;  and do as much repair work as possible down
                                ;  to age 7 but no lower.  Do not repair ships
                                ;  if their age is already 7 or less.
END\end{verbatim} 


In the above example, all of the ships in the sector will pool their damage
repair units and repair only those ships that need repairing, starting with the
most heavily damaged, and continuing until all ships are completely repaired to
age 7 or until you run out of DRs.

If there are still DRs left after all repairs have been made, there is no way
to predict which ships will be carrying them.


\begin{importantnote}
	When a ship is damaged in combat, it is possible that some or
	all of its cargo will be destroyed, including damage repair
	units.  Since this happens quite often, it is not logged.  Thus,
	it's possible for REPAIR commands to fail or to not have the
	desired or expected result.
\end{importantnote}

\newpage
\subsection{INTERSPECIES CONSTRUCTION}

A species with a Manufacturing tech level of at least 25 may build items,
ships, and starbases for other species.  This type of construction will be
referred to as `interspecies construction'.

You may construct things for another species using the \texttt{IBUILD} and \texttt{ICONTINUE}
commands.  These commands are similar to the \texttt{BUILD} and \texttt{CONTINUE} commands with
the following exceptions:
\begin{enumerate}[a.]
 \item An additional argument indicating the name of the recipient
	species must appear immediately after the \texttt{IBUILD} or \texttt{ICONTINUE}
	command word.

	\item The \texttt{IBUILD} and \texttt{ICONTINUE} commands must always complete
	construction of the item in the current turn.  Thus, they must
	NOT indicate an amount to spend unless the item is a starbase.
	Obviously, they must have a final argument indicating the amount
	to spend if the item IS a starbase.
\end{enumerate}
If you wish, you may use ``zero'' arguments if the item being built is a starbase
or anything other than a ship.

When ships or starbases are built, they will be put in orbit around the planet
which produced them, and ownership will then be transferred to the new owner.

Items other than ships and starbases will be added to the inventory of the
colony of the new owner that is on the same planet as the planet that built
them.  Thus, the species that will receive the items must have a colony on the
same planet.  If not, the computer will automatically execute a \texttt{NAME} command
for the recipient species using the same name as the producing planet.

If you want to build a ship or starbase over more than one turn, use the normal
\texttt{BUILD} and \texttt{CONTINUE} commands for the initial stages of construction, and use the
\texttt{ICONTINUE} command for the final stage of construction.  If the item is to be
built in one turn, use the \texttt{IBUILD} command.  In other words, the IBUILD and
\texttt{ICONTINUE} commands always terminate construction and transfer the result to
the recipient species.

When building something for another species, a premium of ten percent will be
added to the cost.  (If the premium is not a whole number, it will be rounded
UP to the next whole number.)  When the \texttt{ICONTINUE} command is used for the final
stage of construction, the premium will be based on the TOTAL cost of the ship
or starbase - NOT on just the remaining cost.

You may only use the \texttt{ICONTINUE} command on a ship or starbase that YOU own.  You
may not use the \texttt{ICONTINUE} command on a ship or starbase that is already owned
by another species.

You MAY use the \texttt{ICONTINUE} command to increase the size of a starbase that you
received earlier from someone else and transfer it back to the original species
or even to a third species (or fourth, or fifth, etc).  In each case, however,
the premium will be based on the TOTAL value of the starbase - not just on what
was added.

Here are some examples:
\begin{quotation}
	\noindent The Humans and Klingons both have colonies on the same planet.
	The Human colony is called PL Big Deal and the Klingon colony is
	called PL Khaarsh Dukh.  The Humans give the production order:

\begin{verbatim}
		Ibuild	SP Klingon, 21 IU
\end{verbatim} 

	When the order is executed, 21 colonial mining units will be added
	to the inventory of PL Khaarsh Dukh.  The total cost to the Humans
	will be $21 + 10\% (\textrm{rounded up}) = 21 + 3 = 24$.  If the Klingons did
	not already have a name for the planet, then the name ``PL Big Deal''
	would have been created for them and would appear on the Klingon
	status report.

	Later, the Klingons issue the following orders during the production
	phases of turns 16, 17, and 18:
\begin{verbatim}
    Turn 16:        Build           BAS Dagger, 200
    Turn 17:        Continue        BAS Dagger, 300
                    Build           DD Hammer, 250

    Turn 18:        Icontinue       SP Human, BAS Dagger, 200
                    Icontinue       SP Human, DD Hammer
\end{verbatim} 

	At the end of turn 18, the Humans would be the new owners of a
	destroyer and a 70,000 ton starbase.  The Klingons would have paid
	the normal costs of 200 during turn 16, and 550 during turn 17.
	However, the amount paid during turn 18 would be the normal cost,
	plus a premium based on the TOTAL value of the items, as follows:

		\noindent Destroyer:  $(1500 - 250) + (10\%$ of $1500) = 1250 + 150 = 1400$ \\
		Starbase: $200 + (10\%$ of $700) = 200 + 70 = 270$
\end{quotation}
The \texttt{IBUILD} command may not be used to build colonist units or planetary defense
units.

You may not build something for a species you haven't met.

You may not build items for a species that you have declared to be an \texttt{ENEMY}.
If you attempt to build something for an \texttt{ENEMY} species, the computer will
reject your order.  If you wish to build something for an \texttt{ENEMY}, then issue
a temporary \texttt{NEUTRAL} order and cancel it immediately afterwards, as in the
following example:

\begin{verbatim}
	Neutral	SP Klingon
	Ibuild	SP Klingon, 21 IU
	Enemy	SP Klingon\end{verbatim} 

All ships and items produced with the \texttt{IBUILD} and \texttt{ICONTINUE} commands will
operate at the tech levels of the recipient species.

If the cost of an item is based on a tech level (such as terraforming plants),
the cost will be based on the tech level of the builder.  However, the item
will always operate at the tech level of whoever is using it.

\newpage
\subsection{AUXILIARY GUN UNITS}

A species with a Military tech level of at least 10 may build Auxiliary Gun
Units.  These units are carried as cargo and add to a warship's existing
offensive power.  Basically, these units can be used to convert unused cargo
space into additional offensive capacity.

Auxilairy gun units are designed to enhance the normal firepower of warships
only.  They may not be used on transports or starbases, although they may be
carried as cargo.

Auxiliary gun units come in the sizes listed in Table~\ref{tab:auxguns}.

\begin{table}[h]
\begin{center}
\begin{tabular}{|cccrc|}
\hline
  \rowcolor{lightblue} \textbf{Unit}&  \textbf{Equivalent}  & \textbf{Needed Carrying}  &                & \textbf{Minimum Military}  \\
\rowcolor{lightblue} \textbf{Abbreviation} &    \textbf{Tonnage}   & \textbf{Capacity} &   \textbf{Unit Cost}&   \textbf{Tech Level}\\
\hline
        GU1       &       50,000      &     5       &      250    &    10 \\
        GU2       &      100,000      &     10      &      500    &    20 \\
        GU3       &      150,000      &     15      &      750    &    30 \\
        GU4       &      200,000      &     20      &      1000   &    40 \\
        GU5       &      250,000      &     25      &      1250   &    50 \\
        GU6       &      300,000      &     30      &      1500   &    60 \\
        GU7       &      350,000      &     35      &      1750   &    70 \\
        GU8       &      400,000      &     40      &      2000   &    80 \\
        GU9       &      450,000      &     45      &      2250   &    90 \\
\hline
\end{tabular}
\caption{Auxiliary gun unit sizes}
\label{tab:auxguns}
\end{center}
\end{table}

The `Equivalent tonnage' indicates the amount of additional offensive capacity
that the unit provides.  For example, a 300,000 ton heavy cruiser carrying two
GU3s will have the normal firepower of a heavy cruiser PLUS the firepower of
two 150,000 ton destroyers.

\begin{informationnote}
Note that the FIREPOWER is additive --- NOT the tonnage.  Thus, the above
heavy cruiser does NOT have the offensive power of a single 600,000 ton
ship!  Keep in mind that the defensive and offensive power of a ship is NOT
proportional to the tonnage.  Larger ships are more powerful than several
smaller ships of the same total tonnage.  For example, the firepower of a
single 400,000 ton battlecruiser is significantly greater than the combined
firepower of four 100,000 ton frigates.  Also, greater firepower does not
necessarily mean that a ship will fire more often per round, since shots are
often combined.  It DOES mean, however, that the same number of shots will
do more damage.
\end{informationnote}

Use the \texttt{BUILD} command to build auxiliary gun units.  For example, to build
3-100,000 ton auxiliary gun units, give the production order:

\begin{verbatim}
	BUILD 3 GU2\end{verbatim} 

The total cost will be 1500.

On the status reports, auxiliary gun units will be referred to as ``Mark-1 Gun
Units'', ``Mark-2 Gun Units'', etc. corresponding respectively to GU1, GU2, etc.

\newpage
\subsection{FAIL-SAFE JUMP UNITS}

A species with a Gravitics tech level of at least 20 may build Fail-Safe Jump
Units.  The function of these devices is to provide more accurate control of
the energies involved in an interstellar jump.  They accomplish this by tuning
themselves to the ship's engines and absorbing any stray energies that would
result in a mis-jump or self-destruction.  In game terms, this is what happens:

\begin{quotation}
	If a ship is about to mis-jump or self-destruct, then the
	Fail-safe jump unit will instead be destroyed, and another
	attempt will be made automatically.  If a ship carries more
	than one unit, then the process may be repeated.  In effect,
	if a mishap occurs, a fail-safe jump unit is destroyed and
	the dice are rolled again.\end{quotation} 

Each unit costs 25 to build, and requires a carrying capacity of 1.  Use the
class abbreviation \texttt{FS} for Fail-Safe Jump Units.  For example, to build 3
units, give the production order:

\begin{verbatim}
	BUILD	3 FS\end{verbatim} 

\noindent The total cost will be $3 \times 25 = 75$.

\noindent Fail-safe jump units will be used automatically, whenever they are needed, as
long as they are carried by a ship.

\newpage
\subsection{JUMP PORTAL UNITS}

A species with a Gravitics tech level of at least 25 may build Jump Portal
Units.  These devices are carried as cargo on a starbase, and a starbase that
carries these units is called a ``jump portal''.  A jump portal can be used to
allow sub-light ships to cross interstellar distances as if they had a jump
drive of their own.  In effect, jump portals create a private wormhole for the
sub-light ship using them.  Jump portals can also be used by older FTL ships
to reduce their chances of a mishap.  There is no limit on how many times jump
portals may be used in a single turn.  Jump portals may NOT be used to move
starbases.

In order for jump portal units to be effective, they must be loaded onto a
starbase.  In effect, the starbase that carries them becomes the control center
for a portal.  A starbase may carry as many jump portal units as needed, as
long as it has sufficient cargo capacity.

A jump portal is characterized by an effective tonnage, which indicates the
maximum tonnage of a ship that may use it.  For example, a jump portal with
an effective tonnage of 80,000 tons will allow any ship to use it that is
80,000 tons or less.  The effective tonnage of a jump portal is 10,000 times
the number of jump portal units carried by the starbase.  For example, if a
starbase carries 20 jump portal units, then a ship of up to 200,000 tons may
use the portal.  Units carried by different starbases may NOT be combined to
increase their effective tonnage - the units affecting a single ship must be
on a single starbase.  Note that the tonnage of the starbase is not important.
Only the number of jump portal units carried by the starbase is important.

Jump portal units have the class abbreviation \texttt{JP}, a cost of 100 each, and
require a carrying capacity of 10 each.  You can build jump portal units
in the usual way, using the \texttt{BUILD} command:

\begin{verbatim}
Build	3 JP	; Build 3 jump portal units, total cost = 300.	\end{verbatim} 

Jump portals are used with the \texttt{PJUMP} command in the jump phase of the turn.
For example:

\begin{verbatim}
	Pjump   TR4S Willow's Helm,   PL Orion IV, BAS Seneca Portal
	PJU	CTS Hippocrates, 12 3 22, Bas Deep space 9\end{verbatim} 

Note that the last argument of the \texttt{PJUMP} command is the name of the starbase
carrying the jump portal units.  Jump portals may NOT be used to move cargo or
colonists directly.  These items may only be moved if they are being carried
as cargo by a ship that uses the portal.

Note also that a jump portal can only operate in one direction.  It can send
a ship TO a different location, but it cannot receive a ship FROM a different
location.  Thus, in order to establish two-way travel between two different
locations, you must have a jump portal at both locations.

It is also possible to use a jump portal owned by another species, if that
species gives you permission.  You may give another species permission to
use all of your jump portals with the \texttt{ALLY} command:

\begin{verbatim}
	Ally	SP Klingon\end{verbatim} 

The above order tells the computer that you consider the Klingons to be allies
and that they are allowed to use any of your jump portals.  \texttt{ALLY} orders must
appear in the pre-departure or post-arrival section of the orders and will
remain in effect until cancelled with a \texttt{NEUTRAL} or \texttt{ENEMY} order.

If you want to allow more than one species to use your portals, then provide
a separate \texttt{ALLY} order for each species:

\begin{verbatim}
	Ally	SP Klingon
	ALL	Sp Vulcans
	ALLY	sp bajorans\end{verbatim} 

A ship using a jump portal uses the age of the starbase and the gravitics
tech level of the species that owns the starbase to determine mishap
probabilities.  However, only the ship using a portal can suffer a mis-jump
or self-destruction.  Fail-safe Jump Units carried by the ship will have
their usual effect.  Fail-safe Jump Units on the starbase will have no
effect at all.

The use of jump portals by a starbase will have no effect on its offensive or
defensive combat abilities.

\newpage
\subsection{FORCED MISJUMP UNITS}

A species with a Gravitics tech level of at least 30 may build Forced Misjump
Units.  These devices are carried and used on a starbase, and are designed to
force hostile ships to leave the star system.  In effect, they are used as
weapons.  If their operation is successful, then any targeted ship will mis-
jump to a different location.  Furthermore, the ship may self-destruct.

Forced misjump units are characterized by an effective tonnage, which indicates
the maximum tonnage of a ship that can be affected by them.  The effective
tonnage is 10,000 times the number of units carried by the starbase.  Units
carried on different starbases may NOT be used in concert to increase their
effective tonnage.  The units affecting a single ship must be on a single
starbase.

The base chance of forcing a misjump is 2\% times the attacker's gravitics tech
level minus the target's gravitics tech level.  This value is increased by 2
percentage points for each forced misjump unit carried in excess of what is
actually needed to affect the target.  Note that the base chance can be
negative, but the effective chance can still be positive if a sufficient
number of units are used.

If the misjump occurs, a wormhole with totally random properties will be
created and the target ship will be forced into it.  The destination will be
a randomly chosen location within the galaxy.  (The x, y, and z co-ordinates
will be random values between 0 and 99, inclusive.)  Since this could be a
large distance from the original location, the ship may self-destruct.

Forced misjump units are used automatically whenever a starbase which carries
them takes part in combat.  However, achieving a ``lock on'' can be difficult
in a fast-paced battle, and an opportunity to use the devices may or may not
arrive in a particular round of combat.  Sometimes, though, if conditions are
exceptionally good, more than one ``lock on'' may occur in a single round.

Forced misjump units have the class abbreviation \texttt{FM}, a cost of 100 each, and
require a carrying capacity of 5 each.

Fail-safe jump units carried by a target will not prevent a forced misjump,
but they may help prevent self-destruction.

Forced misjump units will not be used against starbases.

If a ship is forced to jump during the strike phase, then the ship name will
contain the designation ``FJ'' in the status report.  These ships will jump
automatically in the jump phase of the next turn, and any explicit jump orders
will be ignored.

\newpage
\subsection{FORCED JUMP UNITS}

A species with a Gravitics tech level of at least 40 may build Forced Jump
Units.  These devices are carried and used on a starbase and are designed to
force hostile ships to leave the star system.  In effect, they are used as non-
lethal weapons, and are the non-destructive counterpart of forced mis-jump
units.  If their operation is successful, then any targeted ship will jump to a
different location.  Furthermore, unlike forced MIS-jump units, ships affected
by forced jump units will NOT normally self-destruct, since the wormhole
created by the units is controlled and set for only a very short distance
from the original location.

In all other respects, forced jump units are like forced misjump units.

Forced jump units have the class abbreviation ``FJ'', a cost of 125 each, and
require a carrying capacity of 5 each.

\newpage
\subsection{GRAVITIC TELESCOPES}

A species with a gravitics tech level of at least 50 may build Gravitic
Telescopes.  A gravitic telescope allows the user to detect alien ships and
inhabited planets at large distances from the X Y Z location of the telescope
itself.

A gravitic telescope operates on the same principle as interstellar
communication devices; that is, a small wormhole is created and used as a
conduit for the transmission of information.  However, unlike interstellar
communication devices (which are used in pairs), a gravitic telescope is only
needed on the receiving end.

Gravitic telescopes are installed and used on starbases.  In effect, the
starbase provides the control center for the operation of the telescope.  The
range in parsecs of a gravitic telescope is simply the number of telescope
units that are carried by the starbase divided by two (fractions dropped).  For
example, if a starbase is carrying seven gravitic telescope units, then it may
observe aliens up to three parsecs away.  The maximum distance is also limited
by the gravitics tech level of the species, as follows: 


\[
	\textrm{Maximum distance}  =  \dfrac{\textrm{Gravitics tech level}}{10}
\]


Fractions are dropped.  If a starbase carries more units than are allowed by
the tech level, then the excess units will not be used.

To operate a gravitic telescope, use the \texttt{TELESCOPE} command in the post-arrival
section of your orders, as follows:

\begin{verbatim}
START POST-ARRIVAL
; Place post-arrival orders here.

     Telescope	BAS Peeping Tom

END\end{verbatim} 

where the first and only argument is the name of the starbase that carries
the gravitic telescope units.  The above order will provide a list of all
detectable alien ships, starbases, and populated planets within the range
of the telescope.

The information provided by a gravitic telescope is exactly the same as the
information that appears on your status reports when one of your ships is at
the same X Y Z coordinates as an alien ship or planet;  i.e., the names of
ships in orbit or in deep space, and the names and approximate economic bases
of populated planets.  For planets, the approximate economic base will be
indicated by a single number in parentheses immediately following the planet
name.  Ships that have landed or that are under construction are not listed.
Ships and starbases that are at coordinates that do NOT contain a star system
will be listed if they are within range of the telescope.

In addition, if use of a gravitic telescope detects an alien starbase that
ITSELF contains one or more gravitic telescope units, then the number of units
on the alien starbase will also be listed.

A gravitic telescope is a very delicate and difficult instrument to operate,
and it is possible that some ships/planets may not be detected.  The chance of
detecting a particular ship or planet will depend on the gravitic tech level of
the species operating the telescope (the higher the better).  For planets, the
probability of detection will also depend on how much industry there is on the
planet and on whether or not a colony is actively hiding.  In general, home
planets and large colonies are easy to detect, while smaller colonies are
harder to detect.  Large ships are easier to detect than small ships.  Field-
distorted ships (discussed later) are impossible to detect.

A gravitic telescope requires a cargo capacity of 20, and has a cost of 500.
Use the item abbreviation \texttt{GT} for Gravitic Telescopes.

It is possible that the operation of a gravitic telescope will be detected by
orbiting ships or starbases of the observed species.  The chance of detection
is:

\[
		\textrm{Chance of Detection}  =  2  \times  (\textrm{GV2}  -  \textrm{GV1})
\]

where \texttt{GV2} is the gravitics tech level of the species being watched, and GV1 is
the gravitics tech level of the observer.  If the chance is zero or negative,
then the observer will not be detected.  For example, if SP Ferengi (\texttt{GV} = 32)
is spying on SP Human (\texttt{GV} = 39), then the Humans have a 2 x (39 - 32) = 14
percent chance that they will know that they are being observed.

The check for detection will be done once for each orbiting ship or starbase
that is observed by the gravitic telescope.  Planets and ships landed on the
surface cannot detect the operation of a gravitic telescope because of the
interference from the planet's gravity.

The observer will not know if he is detected.

If the observer is detected, then the status report of the species being
watched will indicate the location of the observer's gravitic telescope.

If operation of a gravitic telescope observes an alien starbase that ITSELF
contains a gravitic telescope, then the above chance will be increased by two
times the number of gravitic telescope units carried by the starbase:

\[
	\textrm{Chance of Detection}  =  2 \times \textrm{NGT}  +  (2  \times  (\textrm{GV2}  -  \textrm{GV1}))
\]

where \texttt{NGT} is the number of gravitic telescopes on the starbase that is being
observed.  For purposes of detection, ALL gravitic telescope units on the
starbase are counted, even if they cannot all be used to scan distant locations
because of the gravitic tech level limitation.

A \texttt{TELESCOPE} order may appear only in the post-arrival phase of the turn.  A
starbase may be given only one \texttt{TELESCOPE} order per turn.

Colonies hidden by means of the \texttt{HIDE} command MAY be detected by a gravitic
telescope.  The reason for this is that successful hiding requires the ability
to evade detection when aliens are present.  If the hiders do not know that
they are being observed, then they may fail at hiding.  And even if a colony
eventually discovers that it is being scanned by a gravitic telescope, it may
be too late for it to successfully hide.

\newpage
\subsection{AUXILIARY SHIELD GENERATORS}

A species with a Life Support tech level of at least 10 may build Auxiliary
Shield Generators.  These units are carried as cargo and add to a warship's
existing shields.  Basically, these units can be used to convert unused cargo
space into additional defensive shield capacity.

Auxilairy shield generators are designed to enhance the normal defensive
shields of warships only.  They may not be used on transports or starbases,
although they may be carried as cargo.

\begin{table}[h]
\begin{center}
\begin{tabular}{|cccrc|}
\hline
  \rowcolor{lightblue} \textbf{Unit}&  \textbf{Equivalent}  & \textbf{Needed Carrying}  &                & \textbf{Minimum Life Support}  \\
\rowcolor{lightblue} \textbf{Abbreviation} &    \textbf{Tonnage}   & \textbf{Capacity} &   \textbf{Unit Cost}&   \textbf{Tech Level}\\
\hline
        SG1     &         50,000    &       5       &      250    &    10 \\
        SG2     &        100,000    &       10      &      500    &    20 \\
        SG3     &        150,000    &       15      &      750    &    30 \\
        SG4     &        200,000    &       20      &      1000   &    40 \\
        SG5     &        250,000    &       25      &      1250   &    50 \\
        SG6     &        300,000    &       30      &      1500   &    60 \\
        SG7     &        350,000    &       35      &      1750   &    70 \\
        SG8     &        400,000    &       40      &      2000   &    80 \\
        SG9     &        450,000    &       45      &      2250   &    90 \\
\hline
\end{tabular}
\caption{Shield generator sizes}
\label{tab:shields}
\end{center}
\end{table}

Shield generators come in the sizes listed in Table~\ref{tab:shields}.

The `Equivalent tonnage' indicates the amount of additional defensive capacity
that the unit provides.  For example, a 300,000 ton heavy cruiser carrying two
SG3s will have the normal shields of a heavy cruiser PLUS the shields of two
150,000 ton destroyers.

\begin{informationnote}
Note that the SHIELD POWER is additive - NOT the tonnage.  Thus, the above
heavy cruiser does NOT have the defensive shields of a single 300,000 ton
ship!  Keep in mind that the defensive and offensive power of a ship is NOT
proportional to the tonnage.  Larger ships are more powerful than several
smaller ships of the same total tonnage.  For example, the shield power of a
single 400,000 ton battlecruiser is significantly greater than the combined
shield power of four 100,000 ton frigates.
\end{informationnote}

Use the BUILD command to build auxiliary shield units.  For example, to build
3-100,000 ton auxiliary shield generators, give the production order:

\begin{verbatim}
	BUILD 3 SG2\end{verbatim} 

\noindent The total cost will be 1500.

\noindent On the status reports, auxiliary shield generators will be referred to as
``Mark-1 Shield Generators'', ``Mark-2 Shield Generators'', etc. corresponding
respectively to SG1, SG2, etc.

\newpage
\subsection{FIELD DISTORTION UNITS}

A species with a Life Support tech level of at least 20 may build Field
Distortion Units.  These devices are carried only on ships and starbases, and
distort sensor data and visual data obtained by others.  In effect, a ship or
starbase that carries field distortion units can be seen and its size can be
determined, but the name of the ship and the species that owns it can NOT be
determined.  It is also not possible to determine if the ship is sub-light or
FTL.

In order for field distortion to take place, the ship or starbase must carry a
number of units EXACTLY equal to its tonnage divided by 10,000.  For example,
since the tonnage of a battleship is 450,000, field distortion will only take
place if the ship carries EXACTLY 45 units.  A TR17S must carry EXACTLY 17
field distortion units, a 250,000 ton starbase must carry EXACTLY 25 units,
and so on.  If a number of units other than the required amount is carried,
then they will simply be carried as cargo and will not be put into operation.

Field distortion units have the class abbreviation \texttt{FD}, a cost of 50 each,
and require a carrying capacity of 1 each.

A ship using field distortion will appear to others as simply a class
abbreviation and the name ``???''.  The species name will simply be a number. 
Here is an example of three alien ships that are using field distortion units
as listed on the status reports of the owners:


The owner of the first ship will see this:

\begin{verbatim}
Ships at x = 19, y = 2, z = 36:
  Name                                           Cap. Cargo
 ----------------------------------------------------------------------------
  CT Thieve's Cant (A8,O2)                         2  2 FD\end{verbatim} 


The owner of the second and third ships will see this:

\begin{verbatim}
Ships at x = 19, y = 2, z = 36:
  Name                                           Cap. Cargo
 ----------------------------------------------------------------------------
  BCS Bandicoot (A3,D)                            40  40 FD
  BS Jabberwocky (A1,D)                           45  45 FD\end{verbatim} 


And here is what other species at the same location would see:

\begin{verbatim}
Aliens at x = 19, y = 2, z = 36:
  Name                                               Species
 ----------------------------------------------------------------------------
  CT ??? (O2)                                        SP 171
  BC ??? (D)                                         SP 92
  BS ??? (D)                                         SP 92\end{verbatim} 


The battlecruiser and the battleship are both owned by the same species,
and the corvette is owned by a different species.  The numbers are randomly
generated, but a particular number will always apply to the same species as
long as it stays at the same life support tech level.  If the tech level
changes, then a new random number is generated.  To add to the confusion,
it is possible, although highly unlikely, that two species will have the
same number at the same time.

For example, if SP 92 corresponds to SP Klingon in the above example, then
ALL field-distorted ships owned by the Klingons, wherever they may be, will be
listed as being owned by SP 92 as long as the Klingon life support tech level
does not change.  However, it is also possible, but very unlikely, that another
species will also be assigned the number 92 at the same time.

If you wish to attack a species that is using field distortion units, but you
don't know their real name, use the name that is listed in the status report.
For example, if you want to attack just the corvette shown above, you should
issue the combat order:

\begin{verbatim}
	ATTACK	SP 171\end{verbatim} 

Field distortion units will be used in combat automatically, only if ALL ships
and starbases of a species at the battle location are using the units, AND if
the species has no populated planets at the battle location.  Otherwise the
units will not be used but will simply be carried as cargo.  This is necessary
because ships and planets of the same species must collaborate, and doing so
could easily betray the real identity and owner of the ships.

In combat, field-distorted ships will be harder to hit.  Specifically, the
chance-to-hit a field-distorted ship will be 25\% less than the chance to hit
an undistorted ship.

If one or more field distortion units are destroyed in combat by damage that
passes through the shields, then the units will stop functioning, and other
participants in the battle will learn the real name of the ship and of the
species that owns the ship.  Note that this will only occur if the units were
functioning when damage passed through the shields.  It will not occur if the
units were simply being carried as cargo.  This also applies to ships that are
destroyed in an intercept or in an ambush.

For the purpose of INTERCEPTs only, a field-distorted ship will be considered
an enemy ship, even if the owner is not actually a declared enemy.  In other
words, a ship that is ``disguised'' is assumed to be inherently hostile.

Field distortion units should never be carried by an attacker if the attacker
plans to besiege a planet.  It is impossible to conduct a siege with field-
distorted ships, and any attempt to do so will reveal the true identity of the
species using the units.

Field-distorted ships and starbases are completely invisible to operators of
gravitic telescopes.

Field distortion units can not be used by a ship that is landed on the surface
of a planet.  Thus, the true name and species of a landed ship will be known
to another species if it has population on the same planet.

Finally, note that by combining the use of field distortion units with the
HIJACK command, a species can raid its enemies and commit other acts of piracy
without giving away the attacker's identity.

\newpage
\subsection{POPULATION GROWTH}

A species with a Biology tech level of at least 20 will experience enhanced
population growth on colonies.

Advanced biological knowledge will increase the fertility and survival rate of
people on colony worlds.  In Far Horizons, we will simulate this reality by
adding a bonus to population growth.  Note that this bonus will apply only to
colonies --- it will NOT apply to the homeworld.

Population growth will receive a bonus of N percentage points, where N is the
Biology tech level divided by 20, fractions dropped.  For example, if the
normal population growth on a colony is 9 percent, and the Biology tech level
is 59, then the actual growth rate will be 9 + 59/20 = 11 percent.

\newpage
\subsection{TERRAFORMING}

A species with a Biology tech level of at least 40 may terraform a planet,
making it more suitable for habitation.  Terraforming has the effect of
modifying the planet's temperature class, pressure class, and gaseous
composition.

Terraforming is accomplished by operating special atmospheric processing
plants on the surface which use biological and chemical methods to modify
the atmosphere.  Terraforming can also be used to add an atmosphere to a planet
that doesn't have one.

The number of plants needed to completely terraform a planet is exactly the
same as the life support requirement (\texttt{LSN}) for the planet.  For example, if
your species requires a minimum Life Support tech level of 9 to establish
a normal colony on the planet, then you must use 9 terraforming plants to
completely terraform the planet.

The cost of each processing plant is 50,000 divided by the Biology tech level
of the species.  For example, if the Biology tech level is 48, then the amount
needed to build a single plant is $50,000/48 = 1041$ (drop fractions).

To build terraforming plants, use the \texttt{BUILD} command and the abbreviation \texttt{TP}:

\begin{verbatim}
	Build	6	TP	; Build 6 terraforming plants.\end{verbatim} 

Terraforming plants can then be transported to the colony where they must be
installed.  Each terraforming plant requires a cargo capacity of 100.

Alternatively, if the colony has sufficient production capacity, you can build
the terraforming plants on the colony where they will be used.

To perform terraforming on a colony, you must issue a \texttt{TERRAFORM} command in
the post-arrival section of your orders, after \texttt{TRANSFER}ing the plants to the
colony:

\begin{verbatim}
START POST-ARRIVAL
; Place post-arrival orders here.

        Tran    6 tp    TR30 Big Boy, PL Epsilon Eridani IV
        Terraform 6     PL Epsilon Eridani IV
        SCAN            TR30 Big Boy

END\end{verbatim} 


The above commands will transfer six terraforming plants from the transport
to the colony and will then install and activate them.  The results of the
terraforming take effect immediately.  The \texttt{SCAN} order will show you what the
results are.

If the number of \texttt{TP}s is not specified or is 0, then all available TPs will be
used, but not more than are actually needed to completely terraform the colony.

When the \texttt{TERRAFORM} order is processed, the gamemaster's computer will change
the planet's temperature class, pressure class, and atmospheric composition to
bring them closer to the home planet's.  If the colony does not have exactly
the number needed to completely terraform the planet, then a partial
transformation will take place, in the following order:

\begin{quotation}
	\noindent 1. Eliminate poisonous gases \\
	2. Add required gas \\
	3. Modify temperature class \\
	4. Modify pressure class
\end{quotation} 

After installation, terraforming plants will not appear in the inventory of
the colony.  In effect, they are ``used up''.  Terraforming plants may not be
recycled after installation.

Although terraforming takes effect immediately in game terms, you can assume
that it actually takes a few years.  Consequently, if a species were to attempt
to terraform, say, the planet of an enemy, the residents would have plenty of
time to prevent the installation and operation of the plants.  Because of this,
terraforming plants may NOT be operated on a planet if one or more other
species also live on the planet, unless the other species allow it.  The
gamemaster's computer will not be able to check for this, but a player
that violates this rule will be evicted from the game.

The effects of terraforming will last until the end of a game, unless
counteracted by terraforming by another species.

\newpage
\subsection{GERM WARFARE BOMBS}

A species with a Biology tech level of at least 50 may build Germ Warfare
Bombs.  When used successfully, a bomb will wipe out an entire species on a
particular planet.

Germ warfare is accomplished by seeding an inhabited planet with specially
designed micro-organisms which are delivered in a device called a Germ Warfare
Bomb.  The effectiveness of the bomb is a function of the relative Biology tech
levels of the attacking species and target species.  Specifically, the base
chance of success is 50\%, and this value is modified up or down by 2\% per point
of difference in Biology tech levels of the two species.

A Germ Warfare Bomb has a cost of 1000, requires a carrying capacity of 100,
and has the class abbreviation ``GW''.  It should be built using the BUILD
command, as in the following example:

\begin{verbatim}
;Build 2 germ warfare bombs.
     BUILD	2 GW\end{verbatim} 

To use germ warfare bombs in combat, issue the appropriate \texttt{ENGAGE} and \texttt{ATTACK}
order as discussed earlier in the section on combat.  All bombs that are
present at the battle will be used.  A separate check is made for each bomb
that the attacker takes to the battle.  If at least one bomb succeeds, then the
defending species will be destroyed.  Thus, the more bombs an attacker brings,
the greater will be the chance of success.

Germ warfare bombs do their dirty work very quickly and an affected planet may
be safely colonized in the next turn.  Remember, ALL mining and manufacturing
base that was originally on the planet will have been destroyed.  However, you
will gain some economic units from looting.

\newpage
\section{CLASS ABBREVIATIONS}

The class abbreviations currently being used in FAR HORIZONS are listed in Table~\ref{tab:classabrvs}. 
Remember, all ships may be built in sub-light versions, which have an ``S''
suffixed to the abbreviation.  For example, a sub-light battleship has the
class abbreviation ``BSS'', a sub-light 150,000 ton transport has the
abbreviation ``TR15S'', etc.

\begin{table}[ht]
 \caption{Class Abbreviations} \label{tab:classabrvs}
\begin{tabularx}{\textwidth}{|lX|lX|}
\hline
\rowcolor{lightblue} \textbf{Abbr} &    \textbf{Class Name} & \textbf{Abbreviation} &    \textbf{Class Name}  \\
\hline
AU & Colonial Manufacturing Unit &	FS & Fail-Safe Jump Unit\\ 
BAS & Starbase& 	GT & Gravitic Telescope Unit\\ 
BC & Battlecruiser& 	GUn & Auxiliary Gun Unit, Mark-n\\ 
BI & Biology tech level& 	GV & Gravitics tech level\\ 
BM & Battlemoon& 	GW & Germ Warfare Bomb\\ 
BR & Battlestar& 	IU & Colonial Mining Unit\\ 
BS & Battleship& 	JP & Jump Portal Units\\ 
BW & Battleworld& 	LS & Life Support tech level\\ 
CA & Heavy Cruiser& 	MA & Manufacturing tech level\\ 
CC & Command Cruiser & MI & Mining tech level\\ 
CL & Light Cruiser & ML & Military tech level\\ 
CS & Strike Cruiser & PB & Picketboat\\ 
CT & Corvette & PD & Planetary Defense Unit\\ 
CU & Colonist Unit & PL & Planet\\ 
DD & Destroyer & RM & Raw Material Unit\\ 
DN & Dreadnought & SD & Super Dreadnought\\ 
DR & Damage Repair Unit & SGn & Auxiliary Shield Generator, Mark-n\\ 
ES & Escort & SP & Species\\ 
FD & Field Distortion Unit & SU & Starbase Unit\\ 
FF & Frigate & TP & Terraforming Plant\\ 
FJ & Forced Jump Unit & &\\
FM & Forced Mis-jump Unit  & TRn & Transport, eg. TR7 for 70,000 tons, TR14 for 140,000 tons, etc. \\ 	

\hline
\end{tabularx}
\end{table}
\newpage
\section{COMMAND SUMMARY}

The commands currently being used in FAR HORIZONS are listed in Table~\ref{tab:commands}.
 
\begin{longtable}{|llp{9cm}|}
\caption{Command Summary} \label{tab:commands} \\
\hline
\rowcolor{lightblue} \textbf{Command} &    \textbf{Arguments} & \textbf{Explanation} \\
\hline
ALLY           & sp          &    Declare species ``sp'' to be an ally \\
AMBUSH       &   n           &    Spend ``n'' in preparation for ambush \\
ATTACK       &   sp          &    Attack opponent ``sp'' \\
ATTACK       &   SP n        &    Attack field-distorted species number ``n'' \\
ATTACK       &   0           &    Attack all declared enemies \\
AUTO         &               &    Automatically generate sensible orders for  next turn \\
BASE        &    [n] s,base &     Build or increase size of starbase ``base'' using ``n'' starbase units from ``s''  \\
BATTLE      &    x y z     &      Set the location for a battle  \\
BUILD       &    n ab      &      Build ``n'' items of class ``ab''  \\
BUILD       &    ship      &      Build ``ship''  \\
BUILD       &    ship,n    &      Start building ``ship'', spend only ``n'' \\
BUILD       &    base,n    &      Start building starbase ``base'', spend ``n''  \\
CONTINUE    &    ship      &      Finish construction of ``ship''  \\
CONTINUE    &    ship,n    &      Continue construction on ``ship'', spend only ``n''  \\
CONTINUE    &    base,n    &      Increase size of starbase ``base'', spend ``n''  \\
DESTROY     &    ship      &      Destroy ``ship''  \\
DESTROY     &    base      &      Destroy starbase ``base''  \\
DEVELOP     &    [n]       &      Build \texttt{CU}s, \texttt{IU}s, and \texttt{AU}s for producing planet but do not spend more than ``n'' \\
DEVELOP    &     [n] pl   &       Build \texttt{CU}s, \texttt{IU}s, and \texttt{AU}s for colony planet ``pl'' in same sector but do not spend more than ``n''  \\
DEVELOP    &     [n] pl, ship &    Build \texttt{CU}s, \texttt{IU}s, and \texttt{AU}s for colony planet ``pl''  and load units onto ``ship'' but do not spend more than ``n'' \\
DISBAND   &      pl      &        Disband colony ``pl'' \\
END       &              &        End current section of the order form \\
ENEMY     &      sp      &        Declare species ``sp'' to be an enemy \\
ENEMY     &      n       &        Declare all species to be enemies \\
ENGAGE    &      n [p]      &     Specify combat engagement option ``n'' and optional planet number ``p'' \\
ESTIMATE  &      sp         &     Estimate tech levels of species ``sp''  \\
HAVEN    &       x y x      &     Set rendezvous point for ships that withdraw from combat \\
HIDE   &                    &     Actively hide this planet from alien observation \\
HIDE   &         ship       &     Keep ``ship'' out of combat unless you start to lose the battle \\
HIJACK  &        sp         &     Hijack opponent ``sp'' \\
HIJACK  &        SP n       &     Hijack field-distorted species number ``n'' \\
HIJACK  &        0          &     Hijack all declared enemies \\
IBUILD  &        sp,n ab    &     Build ``n'' items of class ``ab'' for species ``sp'' \\
IBUILD  &        sp,ship    &     Build ``ship'' for species ``sp'' \\
IBUILD  &        sp,base,n   &    Build starbase ``base'' for species ``sp'', spend ``n'' \\
ICONTINUE &      sp ship     &    Finish construction of ``ship'' for species ``sp'' \\
ICONTINUE &      sp base,n   &    Increase size of starbase ``base'' for species ``sp'', spend ``n'' \\
INSTALL   &      n ab pl      &   Install ``n'' \texttt{IU}s or \texttt{AU}s on planet ``pl'' \\
INSTALL   &      pl           &   Install all available \texttt{IU}s and \texttt{AU}s on planet ``pl'' \\
INTERCEPT &      n            &   Spend ``n'' in preparation for interception \\
JUMP     &       ship,loc     &   Have ``ship'' jump to destination ``loc'' \\
LAND     &       ship,pl      &   Have ``ship'' land on planet in same star system \\
MESSAGE  &       sp           &   Send a message to species ``sp'' \\
MOVE     &       ship, x y z  &   Move ``ship'' up to one parsec \\
MOVE     &       base, x y z  &   Tow starbase ``base'' up to one parsec \\
NAME     &       x y z p PL name &  Give ``name'' to planet ``p'' at location "x y z" \\
NEUTRAL  &       sp           &   Declare neutrality towards species ``sp'' \\
NEUTRAL  &       n            &   Declare neutrality towards all species \\
ORBIT    &       ship,pl      &   Have ``ship'' orbit planet in same star system \\
PJUMP    &       ship,loc,bas &   Have ``ship'' jump to destination ``loc'' via jump portals on starbase ``bas'' \\
PRODUCTION &     PL name   &      Start production on planet ``name'' \\
RECYCLE    &     n ab      &      Recycle ``n'' items of class ``ab'' \\
RECYCLE    &     ship      &      Recycle ``ship'' \\
RECYCLE    &     base      &      Recycle starbase ``base'' \\
REPAIR     &     ship,n    &      Repair ``ship'' using ``n'' onboard damage repair units \\
REPAIR    &      base,n     &     Repair ``base'' using ``n'' onboard damage repair units \\
REPAIR    &      x y z [age]&     Repair as many ships/starbases as possible in sector x y z, pooling damage repair units but do not reduce age below ``age'' \\
RESEARCH  &      n tech    &      Spend ``n'' on research in technology ``tech'' \\
SCAN      &      ship      &      Have ``ship'' do a scan of its current location \\
SEND      &      n sp      &      Send ``n'' economic units to species ``sp'' \\
SHIPYARD  &                &      Increase shipyard capacity by one. \\
START     &      section     &    Start processing ``section'' of the order form \\
SUMMARY   &                  &    Provide only a brief summary of combat results, instead of listing every single hit and miss \\
TARGET   &       n           &    Concentrate fire on target type ``n'' during combat \\
TEACH   &        tech [n] sp &    Transfer knowledge of technology ``tech'' to species ``sp'' to maximum tech level ``n'' \\
TELESCOPE  &     base       &     Operate gravitic telescope on starbase ``base'' \\
TERRAFORM  &     [n] pl     &     Terraform planet ``pl'' using ``n'' TPs \\
TRANSFER   &     n ab s,d   &     Transfer ``n'' items of class ``ab'' from ``s'' to ``d'' \\
UNLOAD    &      ship       &     Transfer all \texttt{CU}s, \texttt{IU}s, and \texttt{AU}s from ``ship'' or \\
UNLOAD    &      base       &     starbase ``base'' to the planet it is at and install as many \texttt{IU}s and \texttt{AU}s as possible \\
UPGRADE   &      ship       &     Upgrade ``ship'' to age zero \\
UPGRADE   &      base       &     Upgrade starbase ``base'' to age zero \\
UPGRADE   &      ship,n     &     Upgrade ``ship'', spend ``n'' \\
UPGRADE   &      base,n     &     Upgrade starbase ``base'', spend ``n'' \\
VISITED   &      x y z      &     Mark a star system as having been visited, even if you have not actually been there. \\
WITHDRAW   &     n1 n2 n3    &    Set conditions for withdrawing from combat \\
WORMHOLE   &     ship [,pl]  &    Have ``ship'' jump to opposite end of wormhole and orbit planet ``pl'' on arrival \\
WORMHOLE   &     base [,pl] &     Have starbase ``base'' jump to opposite end of wormhole and orbit planet ``pl'' on arrival \\
ZZZ       &                  &    Terminate a MESSAGE \\
\hline
\end{longtable}

where:
\begin{tabular}{r@{ = }l}
	\noindent ab & class abbreviation \\
	base & name of a starbase, including ``BAS'' abbreviation \\
	d & name of a ship, starbase, or planet, including class abbreviation \\
	loc & jump destination, either ``x y z'' or ``PL name''  \\
	n & a whole number, 0 or more  \\
	$\left[n\right]$ & an optional whole number, 1 or more  \\
	name & name string, including any embedded spaces. May not start with a digit! \\
	p & planet number \\
	pl & planet name, including abbreviation ``PL'' \\
	s & name of a ship, starbase, or planet, including class abbreviation \\
	section & COMBAT, PRE-DEPARTURE, JUMPS, PRODUCTION or POST-ARRIVAL \\
	ship & name of a ship, including class abbreviation \\
	sp & species name, including ``SP'' abbreviation \\
	tech & technology abbreviation: MI, MA, ML, GV, LS, or BI \\
	x y z & galactic coordinates of a sector \\
\end{tabular}



\newpage

\section{SET-UP FORM FOR ENTERING A GAME}

The following form must be filled out at the start of a game and sent to the
gamemaster.  Refer to Chapter 3 for detailed information on how to fill out
the form.

\begin{verbatim}
Allocate points to Military, Gravitics, Life Support, and Biology tech levels.
You have a total of 15 points to allocate.
     Military:
     Gravitics:
     Life Support:
     Biology:

[REMINDER: If a tech level is zero, you will not be able to raise it unless
another species transfers the knowledge or technology to you.]

Next, enter the names for your species, home planet, and government.
You may use up to 31 characters each.

     Species name (MUST contain 7 or more characters):
     Home planet name:
     Government name:
     Government type:\end{verbatim} 

\begin{importantnote}
	If your gamemaster is running a list server for use
	by the players, then your species name may not contain any of
	the special characters `\$' (dollar sign), `!' (exclamation
	point), or `"' (double quote).  In general, it is a good idea
	not to use any of these characters when naming your species.
	Also, make sure that the name of your species contains at
	least seven characters, including embedded spaces.
\end{importantnote}

\noindent Fill out the above items and send them to the gamemaster before the game
starts.


\begin{center}
\textbf{End of Rules for FAR HORIZONS}
\end{center} 

\end{document}
